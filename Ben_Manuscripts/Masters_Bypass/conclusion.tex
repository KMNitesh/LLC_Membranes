\section*{Conclusion}

In this work, we have illustrated a more detailed picture of the structure 
of a self-assembled thermotropic liquid crystal membrane using an atomistic 
molecular model. Using the model we have learned that these liquid crystal
monomers prefer to stack in a sandwiched configuration with five monomers
in each layer. By a thorough analysis of the membrane structure, we have
discovered the existence of two metastable basins that both persist at room
temperature. Structurally, Basin A matches closest to experiment, while 
Basin B has not been seen experimentally. We hypothesize that Basin B will
be dominant at higher temperatures. Our model's inability to capture aromatic
interactions likely destabilizes Basin A at room temperature and enhances 
the stability of Basin B. Basin B is likely a part of the transition towards
the isotropic phase which exists at high temperatures. This hypothesis will
be tested in future work. In addition to structural accuracy, our model's
physical properties are consistent with experimental measurements. Ionic
conductivity, calculated
in two ways, is in reasonable agreement with experiment. Using the design
framework and analysis methods applied herin, we have the ability to 
understand structures of new and unsynthesized LLC membranes.
