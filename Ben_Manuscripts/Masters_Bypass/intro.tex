\section*{Introduction}

Nanostructured membrane materials have become increasingly popular for 
aqueous separations applications such as desalination and biorefinement
because they offer the ability to control membrane architecture at the
atomic scale allowing the design of solute-specific separation membranes 
\cite{humplik_nanostructured_2011}. Most membrane-based aqueous 
separations of small molecules can be achieved using reverse osmosis 
(RO) or nanofiltration (NF) \cite{van_der_bruggen_review_2003}. 

While RO and NF have seen many advances in the past few decades, they 
are far from perfect separation technologies. Current state-of-the-art
RO membranes are unstructured with tortuous and polydisperse diffusion
pathways which leads to inconsistent performance \cite{song_nano_2011}.
Necessarily high feed pressures drive up energy requirements which 
strain developing regions and contribute strongly to CO\textsubscript{2} 
emissions \cite{mcginnis_global_2008}. Moreover, designing RO 
membranes to achieve targeted separations of specific solutes is nearly
impossible because various solutes dissolve into and diffuse through the polymer
matrix at different rates \cite{wijmans_solution-diffusion_1995}. At best,
one can exploit these differences to create a functional selective 
barrier. NF was introduced as an intermediate between RO and 
ultrafiltration, having the ability to separate organic matter and salts
on the order of one nanometer in size. Larger and well-defined pores 
drive down energy requirements while still affording separation of 
solutes as small as ions to some degree \cite{van_der_bruggen_review_2003}.
This is why NF is often used as a precursor to reverse osmosis. 
Unfortunately, NF membranes, like RO, are produced with a pore size 
distribution which limits their ability to perform precise separations
\cite{bowen_modelling_2002}.
% Thesis: RO and NF are imperfect
% RO membranes are tortuous and polydisperse, energy intensive, can't do targeted separations well
% NF lower energy but has a pore size distribution

Nanostructured membranes can bypass many of the performance issues which
plague traditional NF and RO membranes. One can accomplish targeted
separations by tuning size and functionality of the molecular building
blocks which form these materials. Solute rejecting pores can have their  % BJC: "these materials" --> "nanostructured membranes", or is that redundant?
size tuned uniformly, resulting in sharp size cut-offs. Entirely different
mechanisms may govern transport in a given nanostructured material which
can inspire novel separation techniques.
% Thesis: Nanostructured materials can overcome issues seen in RO and NF
% We can achieve targeted separations by controlling the system arhitecture
% We can get uniform pore sizes 
% we can create new separation techniques by learning about these systems <-- This point might be better elsewhere

Development of nanostructured materials has been limited by the ability
to synthesize and scale various fundamentally sound technologies.
Graphene sheets are atomically thick which results in excellent permeability
but defects during manufacturing severely impact selectivity 
\cite{cohen-tanugi_multilayer_2016}. Molecular dynamics simulations of
carbon nanotubes show promise \cite{humplik_nanostructured_2011} but 
synthetic techniques are unable to achieve scalable alignment and pore
monodispersity \cite{hata_water-assisted_2004,maruyama_growth_2005}.
Zeolites have sub-nm pores with a narrow pore size distribution and MD simulations
exhibit complete rejection of solvated ions \cite{murad_molecular_1998},
however, experimental rejection was low and attributed to interstitial
defects formed during membrane synthesis \cite{li_desalination_2004}.
% Thesis: Past development has been limited by the ability to create working membranes at a useful scale	  
% Graphene sheets have defects
% Carbon nanotubes can't be made uniform or aligned
% Zeolites may have interstitial defects hurting performance

Self assembling lyotropic liquid crystals (LLCs) are a suitable candidate
for aqueous separation applications. They share the characteristic
ability of nanostructured membrane materials to create highly ordered 
structures with the added benefits of low cost and synthetic techniques
feasible for large scale production \cite{feng_scalable_2014}. LLCs are
versatile and controllable with a large chemical design space available
for membrane design. We will be studying assemblies formed by Na-GA3C11 
(Fig~\ref{fig:python}a), a system which
has been extensively studied experimentally \cite{smith_ordered_1997,
zhou_supported_2005,resel_h2-phase_2000,feng_scalable_2014,feng_thin_2016}. 
Neat monomer forms the thermotropic, Col\textsubscript{h} phase. The 
presence of small amounts of water results in the H\textsubscript{II} 
phase. In both cases, monomers assemble into mesophases made of 
hexagonally packed, uniform size, cylinders with hydrophilic groups oriented inward
towards the pore center and hydrophobic groups facing outward. Until
% BJC: reproduce the figure from ACS nano? It does a good job
% BJC: I can also color in the atomistic layer a bit 
recently, the mesophases could not be macroscopically aligned, resulting
in a low flux membrane, slowing research in the field. In 2014, 
Feng et al. showed that the mesophases could be aligned using a magnetic
field with subsequent crosslinking to lock the structure in place
\cite{feng_scalable_2014}. In 2016, Feng et al. showed that the same 
result could be obtained using a technique termed soft confinement
\cite{feng_thin_2016}. Following this breakthrough, research into LLC
membranes has been reinvigorated. We have chosen to focus our initial efforts 
on the development of a model of the Col\textsubscript{h} phase membrane
in order to accelerate development of LLC membranes.  Compared to the 
H\textsubscript{II} phase, the Col\textsubscript{h} phase is a simpler
starting point, due to the absence of water, and has equivalent 
experimental structural data. 
. 
% Thesis: LLCs would make good filtration membranes
% They share the same properties as other nanostructured materials
% They can be scaled and are cheap to make
% LLCs have a large chemical design space
% It has been studied extensively
% Ordered - hexagonally packed cylinders with uniform sized pore regions
% Scalable - alignment gets rid of some performance issues associated with misalignment

A molecular level understanding of LLC membrane structure will elucidate
small molecule transport mechanisms, providing guidelines to reduce the
chemical space for the design of monomers used to create separation-specific
membranes. We do not yet understand how to reduce the effective pore
size or how to tune the chemical environment in the 
nanopores for effective water desalination or small organic separations.
Rejection studies show that this membrane can not perform separations of solutes less than 1.2 nm  
because the pores are too large \cite{zhou_supported_2005}. Over the past
20 years, LLC membrane studies have been limited primarily 
to Na-GA3C11 with some characterization done after minor structural 
modifications \cite{resel_structural_2000}. Optimization has been 
performed through trial and error. The only source of predictive modeling  % BJC: Reference. I think it is just w.r.t. bicontinuous cubic
has been macroscopic models which likely do not adequately describe 
transport at these length scales. A good model should incorporate a 
detailed picture of the nanoscopic pore structure. Pore components may 
play a role in the rejection of charged or uncharged solutes. Choice of 
counterion may influence the establishment of a Donnan potential
affecting the degree to which the membrane can exclude charged species.
Moieties inside the pores may interact with neutral solutes, rejecting
% BJC: (so I can remember this theory in the future) A donnan potential
% is established when the same ions are present on either side of a barrier.
% Pretend you have a bath of NaCl separated by a semipermeable membrane from
% a bath of NaPr where Pr is a protein. Sodium will first cross to the side
% with less sodium down a concentration gradient. This will cause a
% charge imbalance which will cause sodium ions to cross back to the other 
% side down an electric potential gradient. Eventually this movement 
% reaches an equilibrium and a Donnan potential is established since the 
% charge will likely be unbalanced. In the context of the HII membrane, we
% will have a bunch of sodium counterions in the pores and a bath of NaCl
% above the membrane. In most cases, except in high concentrations of bulk
% NaCl, sodium ions from the pore will leave the membrane down a conc gradient.
% Some will go back in, but the resulting system will be a net negatively charged
% membrane with a net positive charge in the bulk solution (at least close to
% the membrane-salt water interface). Sodium ions will be rejected from the 
% bulk due to charge exclusion despite a concentration gradient on either 
% side of the membrane.
on the basis of shape and size, rather than just hydrodynamic
radius. An atomistic understanding of pore structure and its influence on
% BJC: probably leave this point out since it involves a solvated system: 
% Is water structured inside the pores, restricting low energy pathways
% for solutes to follow? 
transport can help identify performance bottle necks and direct design of
future membranes. 
%Thesis: A molecular model will help us understand solute transport in these
% systems and aid in the selection of monomers for creating new membranes
% We can figure out a way to make desalination feasible
% Most attempts at improvement have been based on trial and error
% Past modeling hasn't been nanoscopically detailed and can't tell us what
% we need to know to make improved membranes 
% An atomistic model can actually help direct design

%BJC: Can't decide which thesis to use
% Thesis 1
% A clear picture of the nanoscopic structure of LLC membranes, gained by
% building a molecular model will provide evidence to support or call
% into question past drawn conclusions that have largely guided our 
% understanding of the system. 
% Thesis 2
A clear picture of the nanoscopic LLC membrane structure, gained by building 
a molecular model will provide evidence to answer existing and newly proposed
questions.
The arrangement of monomers
in the channels is thought to be confined to the pore walls. It is 
possible they are arranged more randomly. This would change the way
one thinks about molecules diffusing through the membrane. The
Col\textsubscript{h} phase is described as having pores made of disks or
layers stacked
on top of one another, each containing a set number of monomers. A 
simple simulation study of a similar molecule suggests that there are 4
monomers in each disk~\cite{zhu_methacrylated_2006}. A separate calculation
based on the volume of the liquid crystal monomers proposes that there 
are seven monomers in each layer~\cite{resel_structural_2000}. 
A molecular model has the best chance of directly answering this question.
Once we know the number of monomers in each layer, we still do not know how 
monomers in each layer are positioned with respect to other layers. 
One of the driving forces for self % reference something with self assembly of aromatic compounds
assembly in this system is thought to be pi-pi stacking interactions
between aromatic headgroups \cite{gazit_possible_2002}. Gas phase ab
initio studies of benzene dimers have shown a clear energetic advantage
for parallel displaced and T-shaped pi-pi stacking conformations versus a
sandwiched conformation~\cite{sinnokrot_estimates_2002}.
Substituted benzene rings exhibit an even stronger pi-pi stacking 
attraction which favors the parallel displaced configuration in all cases
except where the substitutions are extremely electron withdrawing
\cite{waller_hybrid_2006,ringer_effect_2006}. While we might be able to
provide answers to these questions using a molecular model, there remains
the possibility that there is more than one metastable state associated
with a given LLC system. We must be able to identify which states can
%BJC: might get rid of this last point since we don't address experimental viability of metastable states
%BJC2: well let's see what results Xunda can pull out
be produced experimentally and what implications each state might have
regarding transport properties.
% Thesis: Knowing what the molecular structure can validate and explain work that has been done in the past
% We might construct a different picture of the arrangement of sodium ions 
% We might be able to answer how many monomers are in each layer
% How do benzene rings stack on top of each other?
% Is it possible there are more metastable states?

We must show that the developed molecular model is consistent with
physical observations so that we can rely on conclusions drawn about % better word for trust?
structural features characteristic of the system. This article will
illustrate the development of a predictive molecular model and the steps
taken to ensure it mimics the real system within the constraints 
inherent to MD. To understand how physically realistic the model is,
validation by comparison to experiment is necessary. We are primarily
interested in reproducing the conclusions about structure which have been
made from X-ray diffraction (XRD) experiments and in matching ionic conductivity measurements
\cite{feng_thin_2016}. We have comparied simulated X-ray diffraction
patterns to experiment in order to match major features present in the
2D patterns. We calculated ionic conductivity using two agreeing methods.
We examined the the influence of crosslinking on membrane structure. 
The analysis used in this paper can be readily extended to the 
H\textsubscript{II} phase and other similar LC systems.
% Thesis: Our model needs to be consistent with experimental observations
% A bunch of things we can compare to 
