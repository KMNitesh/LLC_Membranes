\documentclass{article}
\title{Predicting Transport in Lyotropic Liquid Crystal Membranes with Molecular Dynamics Simulations -- Outline}
\author{Benjamin J. Coscia \and Douglas L. Gin \and Richard D. Noble \and Michael R. Shirts} 
\begin{document}
	\bibliographystyle{ieeetr}
	\maketitle
	\section{Introduction}
	Nanostructured membrane materials have become increasingly popular for aqueous separations applications because they offer the ability to control membrane architecture at the atomic scale allowing design of solute-specific separation membranes.
	\begin{itemize}
		\item Current state-of-the-art reverse osmosis membranes are dense and unstructured with tortuous and polydisperse pores which lead to inconsistent performance
		\item Tortuosity and polydispersity drive up energy requirements which strain developing regions and contribute strongly to CO2 emissions
		\item Designing RO membranes to achieve targeted separations of specific solutes is nearly impossible due to the separation (hypothesized to be) controlled by fluctuating polymer voids
		\item RO has difficulty separating neutral organics because they tend to dissolve in polymer matrix
		\item Many current RO membranes degrade in typical chlorine filled municipal water supplies (debating this point because there aren't any studies of LLC membrane fouling resistance)  
		\item With nanostructured materials, solute rejecting pores can be tuned uniformly -- drives down energy requirements
		\item Targeted separations can be accomplished by tuning the molecular building blocks which form these materials
		\item Entirely different mechanisms govern the separation processes in various nanostructured materials which can inspire novel separation techniques
	\end{itemize}
	
	Development of nanostructured materials has been limited by the ability to synthesize and scale various fundamentally sound technologies.
	\begin{itemize}
		\item Leading technologies and their limitations:
		\begin{itemize}
			\item Graphene sheets -atomically thick which gives excellent permeability but defects during manufacturing severely impact selectivity
			\item Carbon Nanotubes - MD studies are promising but synthetic techniques unable to achieve necessary alignment and pore monodispersity
			%BJC: The following are ideas pulled from the grant. I need to do some reserach and come up with some one liners as to why each have not been successful
			\item Track-etch membranes
		    	\item Molecular squares
			\item Macrocyclic surfactants 
		\end{itemize}
	\end{itemize}
	  
	Self assembling lyotropic liquid crystals (LLCs) share the characteristic ability of nanostructured membrane materials to create highly ordered structures with the benefits of low cost and synthetic techniques feasible for large scale production.
	\begin{itemize}
		\item LLCs are versatile and controllable with a large chemical design space available for membrane design
		\item Synthetic techniques are cheap and amenable to creating any monomer in this large design space
		\item LLCs forms lamellar, bicontinuous cubic and hexagonal phase based on solution composition
		\item Na-GA3C11 forms two types of self assembled LLCs - thermotropic (Colh) and lyotropic (HII)
		%BJC: this is making me scratch my head - I'm afraid to talk a lot about this because of Mike's new data. But I think it's important to distinguish between what has been reported in past literature 
		\item The thermotropic, Colh, is formed by the self assembly of neat monomer
		\item The lyotropic, HII phase is formed in the presence of small amounts of water
		\item Both assemble into cylinders with hydrophilic groups oriented inward towards the pore center and hydrophobic group facing outward. The only difference is the inclusion of water in the structure which leads to minor variations in the structure with potentially different filtration properties (although no filtration experiments have been done on Colh)  
		\item Hydrophilic regions point towards pore centers
		\item Until recently, they could not be aligned - hindered progress
		\item Yale aligns them, then crosslinks them to lock in the structure - reference 2014 and 2016 papers
		\item LLC HII phase membranes offer potential for high permeability and selectivity which equals low energy consumption
		\item The Colh phase shares the same structural features with the HII phase with the exception of the presence of water. This paper will focus on the development of a model of the Colh phase since it is a simpler starting point and has just as much experimental data. The analysis used in this paper can be readily extended to the HII phase. 
	\end{itemize}
	
	A molecular level understanding of LLC membrane structure will elucidate small molecule transport mechanisms, providing guidelines to reduce the chemical space for design of monomers used to create separation-specific membranes.
	\begin{itemize}
                \item We do not yet understand how to reduce the effective pore size and/or tune the chemical environment in the HII nanopores for effective water desalination and small organic separations. Rejection studies show that this membrane can not do desalination yet
		\item Colh phase studies currently limited to one monomer
                \item Optimization efforts performed through trial and error over the past 20 years
                \item Macroscopic models are the only source of predictive modeling and existing theories do not adequately describe transport at these length scales
                \begin{itemize}
                        \item Do ions have trouble getting through because of interactions with other things in the pores (e.g. ions, carbonyl groups, benzene rings) -- related to ionic conduction
                        \item Does concentration of ions in the pore repel incoming ions?
                        \item Do neutral solutes get rejected based solely on size rejection, or do interactions within the pore lead to selective rejection
                        \item Is water structured inside the pores leading to transport barriers?
			\item What does the microscopic pore structure look like and how can we relate that to monomer design and transport?
		\item An atomistic understanding of the mechanism of solute transport can identify performance bottle necks and direct design of future monomers/membranes
		\item We can use molecular dynamics simulations to enhance our understanding 
	\end{itemize}
	 
	A clear picture of the microscopic structure of LLC membranes, gained by building a molecular model, will confirm or deny past drawn conclusions that have largely guided our understanding of separation mechanisms. 
	\begin{itemize}
		\item The arrangement of sodium ions in the channels is thought to be confined to the pore walls. It is possible they are arranged more randomly
		\begin{itemize}
			\item This could change how one thinks about molecules diffusing through membrane
			\item Could also be a difference between lyotropic and thermotropic phases
		\end{itemize}
		\item How do the monomer head groups prefer to be oriented relative to each other?
		\begin{itemize}		
			\item The Colh phase is described as having pores made of disks or layers stacked on top of one another, each containing a set number of monomers. 
			\begin{itemize}
				\item Do the benzene rings prefer to be stacked on top of each other or in another pi-stacking mode.
				\item Gas phase ab initio studies of benzene dimers have shown a clear energetic advantage for a parallel displaced or T-shaped conformation versus a stacked conformation. %BJC: Although the system is made of bulky substituted benzenes, I think it's easy to see how one can hypothesize for our system based on the simpler system.   	
		\item A simple simulation study of a similar molecule (Head group is a sulfonate in the meta position) suggests that there are 4 monomers in each disk
		\item Calculations based on the volume of the liquid crystal suggest that there are seven monomers in each 
		\item We can use fourier analysis to determine if there are distinct layers, and if so, how many monomers are situated in each layer.
	\end{itemize}
	
	We must show that the developed molecular model is consistent with physical observations so that we can trust conclusions drawn about structural features characteristic of the system.
	\begin{itemize}
		\item This paper will illustrate the careful development of a predictive molecular model and the steps taken to ensure it mimics the real system as best we can 
		\item To understand how physically realistic the model is, validation by comparison to experiment is necessary
		\item We are primarily interested in reproducing the conclusions about structure which can be made from SAXS experiments, predicting ionic conductivity with a reasonable comparison to experiment, and reproducing experimental density measurements.    
			\item We can simulate x ray diffraction patterns based on atomic coordinates
			\item We can predict ionic conductivity using two agreeing methods
			\item Density? - I will ask Mike to take more density measurements. It seems worthwhile     
	\end{itemize}
	
	\section{Methods}
	
	HII monomers were parameterized using the Generalized Amber Forcefield with the Antechamber package provided with AmberTools16. All molecular dynamics simulations were run using Gromacs version 5.1.2.
	
	An ensemble of characteristic, low-energy vacuum monomer configurations were constructed by applying a simulated annealing process to a parameterized monomer.
	\begin{itemize}
		\item Structure cooled from 1000 to 50 K over 10 nanoseconds
		\item Result not global minimum but close enough for structure building
		\item Antechamber used for atomtyping with gaff forcefield
		\item Used Openeye Quacpac molcharge.py to assign charges
		\item Anneal again 
		\item Multiple configurations saved from annealing trajectory to prove independence of starting config
		\item Manual modifications to the structure were made to create specific geometries
	\end{itemize}
	
	The timescale for self assembly of monomers into the hexagonal phase is unknown and likely outside of a reasonable length for an atomistic simulation, calling for a more efficient way to build the system. 
	\begin{itemize}
		\item Work done shows coarse grain model self assembly in ~1000 ns , Citation: J. Phys. Chem. B 2013, 117, 4254-4262
		\item Attempts with Colh system not fruitful  
		\begin{itemize}
			\item Packed monomers into box
			\item Simulated for ~100 ns with no progress shown towards self assembly
		\end{itemize}  
		\item Wrote code to assemble monomers into Colh configuration close to what is expected 
		\item Equilibration simulations allow structure to relax into expected configuration 
	\end{itemize}
	
% 	Each pore is made of twenty stacked monomer layers, the smallest number of layers possible without sacrificing the expected aligned hexagonal morphology.
%         %MRS4: OK, but note that we will need to expand this to be enough convincing qualitative information.  Could also use quantitaitve information about how far from the surface you see the effects of the surface.
% 	\begin{itemize}
% 		\item Space between membrane layers in z direction - semi-isotropic sims fix z box dimension
% 		\item 5 layers create micellar structure
% 		\item Beyond 20 layers is unnecessary to get expected configuration
% 		\item layers spaced far enough apart to avoid large energy repulsions but close enough to prevent disks slipping between one another ~ 5 angstroms works.
% 		\item How far into membrane do we have to go to avoid effects of the surface (this is a TODO)
% 	\end{itemize}
%MRS3: what is the thesis? decide what each figure is attempting to communicate to the reader.
%BJC3: Maybe i include this in the supplemental information. I don't think a picture of the initial configuration is actually that interesting or important. I can't think of a thesis for a figure that is purely illustrative 
	Each pore is made of twenty stacked monomer layers with continuity in all directions, avoiding any edge effects and creating an infinite length pore ideal for studying transport. 
	\begin{itemize}
		\item A thinner system is better to reduce the computational cost and allow us to look at longer timescales
		\item Number of layers chosen to give sufficient resolution when simulating XRD patterns
	FIGURE: initial configuration \\
	caption: A representative initial configuration. Each pore is rotated about the z axis rotated to prevent overlaps between pores \\
	\\
	\noindent Initial guesses for the remaining structural parameters were chosen based on experimental data and treated as variables during model development 
	\begin{itemize}
		\item XRD gives Pore-to-Pore distances of ~4.1 nm and indicates possible pi-stacking at ~3.7 Angstroms
		\item Pi-stacking exists in multiple stable configurations: sandwiched, T-shaped and parallel-displaced
		\item T-shaped and parallel-displaced are nearly isoenergetic and more stable than the sandwiched configuration.
		\item T-shaped configuration is most stable when benzene centers are ~ 5 A apart which is not consistent with WAXS. http://www.jbc.org/content/273/25/15458.full 
		\item System made with stacked and parallel-displaced benzene rings to see what is favored and matches XRD
		\item TEM images and rejection studies give a pore size estimate
		\item To prevent unrealistic jumps during initial equilibration steps, the following equilibration scheme was adopted
		\begin{itemize}
			\item Apply position restraints to monomer head groups during energy minimization 
			\item Leave position restraints for npt simulations to allow tails to intermingle (this also helps ensure independence of starting configuration)
			\item Gradually reduce force constants from 1000000 (by square root every 50 ps) until they are completely off
			\item Run long NPT simulations at 300 K and 1 bar ( $>$200 ns ) to fully equilibrate
		\end{itemize}
	\end{itemize}
	
	Using an equilibrated structure, a crosslinking procedure was performed in order to better parallel synthetic procedures. 
	\begin{itemize}
		\item Crosslinking maintains alignment of cylindrical mesophases - emphasize that replicating the mechanism/kinetics is not important 
		\item head to tail addition dominates so I only implemented that
		\item racemic mixture - don't have to be too concerned about direction of attack 
		\item Details of crosslinking algorithm (refer to appendix or supplemental info but give a brief overview here)
	\end{itemize}  
	
	\section{Results and Discussion}
	
	Appropriate selection of initial structural parameters including interlayer spacing, relative interlayer orientation, and pore spacing results in an equilibrated model that is experimentally consistent.
	\begin{itemize}
		\item Visual perspective \\
%MRS3: what is the thesis of the figure -- what is it trying to show.  Important when analyzing the outline and deciding what it means.  Make that the caption.
		FIGURE: shows top view and cross section (with pbcs on)\\
		caption: \\ 
		(a) The expected hexagonal membrane morphology is preserved upon equilibration \\
		(b) Straight cylindrical pore regions are present \\
		\item System dimensions and properties are consistent upon equilibration using multiple monomer configurations 
		\item Pore to pore distance
		\begin{itemize}
			\item We know from SAXS data what the distance between pores should be
			\item We require long simulation times to reach an equilibrated structure with the correct dimensions
			\item Crosslinking locks in structure, maintaining pore distances even after long simulations
		\end{itemize}

		FIGURE - 2 panels \\
		(a) Measuring the distance between pores over time indicates equilibration when the distances stop changing \\
		(b) A crosslinked system run for 100 ns shows that the system maintains itself close to its initial position \\
 
		\item Simulated X-ray Diffraction
		\begin{itemize}
			\item We used the 3 dimensional fourier transformed electron density to generate simulated 1D and 2D diffraction patterns
			\item The 1D patterns are generated by spherical integration of the FT
			\item 2D patterns are generated by taking cross sections of the FT in the qx, qy and qz planes
			\item We matched experiments based on iterative improvement of our choice in initial structure and equilibration procedure
		\end{itemize}
	        
		FIGURE - 4 panels \\
        	(a) Experimental Small Angle X-ray Scattering data confirms a hexagonally packed morphology and tells us the average distance between pores \\
        	(b) A simulated 1D x-ray diffraction pattern generated from atomic coordinates matches experimental measurements \\
        	(c) Experimental Wide Angle X-ray Scattering patterns provide details of structure on a length scale less than 1 nm. The diffuse ring at q = 1.4 A\textsuperscript{-1} represents the average distance between carbons in packed alkyl chains. Meridional reflections at q = TBD A\textsuperscript{-1} are caused by pi-stacking. The weak meridional reflections at q = TBD A\textsuperscript{-1} are hypothesized to be a consequence of parallel-displaced stacking of benzene rings \\
		(d) Simulated X-ray diffraction pattern show the same dominant reflections which appear experimentally

		\item Pore Radius - a less reliable validation because we don't have an agreed upon way to measure this parameter experimentally
		\item Density - there are crude lab measurements which my model is in agreement with (no one has reported any values since it wasn't relevant to them)
	\end{itemize}

        Experiments to be run!
%MRS4: more generally, we may want to explore the fact that there are other stable configurations (like the disordered ones we have found).  We should brainstorm ways of calculating the free energy difference, BUT we should avoid it in this paper.
        \begin{itemize}
                   \item Layered vs. offset configuration
                   \begin{itemize}
                        \item All simulations involving 'offset' configurations (which seems to be the correct configuration) will have a 'layered' counterpart in order to highlight difference between the two configurations.
%MRS4: Which, hopefully, will show that that is not consistent with experiment.
                        \item The comparison can be a figure in supplemental information
                   \end{itemize}
                   \item Measures of stability / analysis (These methods will be applied to all simulations run)
                   \begin{itemize}
                        \item Pore-to-pore distances
                        \item Tilt angle vs. time  % it seems more likely that the circular anisotropy is NOT tilt angle but it is worth monitoring for now.
                        \item Radial distribution functions  %MRS: this may be some thing that we present as an analysis AFTER we look at stability, since the other things are more things that can be verified vs. experiment.
                        \item Simulated diffraction patterns
                   \end{itemize}
                   %MRS4: so this is something that we want to be careful about how we present the rationale.  THe original rationale is that if the interactions were effectively too weak compared to experiemnt, then the 'effective' temperature would be a bit higher than we the 'thermostatted' temperature.  I think that  
                   \item Effect of temperature on stability of configurations.  15 degrees off (room temperature is 295) is much more likely than 260, though for our purposes, we probably want to run lower just to be sure.  Would be interesting to know the temperature stability of the experimental structure before cross-linking . . .  
                   \begin{itemize}
                        \item Three different temperature systems have been set up and simulated for 20 ns (All with 20 layers and no vapor gap): 260K, 280K and 300K
                          %MRS4: we will need longer than 20 ns to make sure they have stabilized.  We know there are structures that are at least metastable (our somewhat more disordered phase)
                        \item I have systems with no restraints on at all three temperatures
                        \item We can look at them with weak restraints in case the systems misbehave as time goes on. As of now, the tilt angles look to stabilize after 10 ns
%MRS4: yea, the reason for weak restraints would be to kind of replace the force field errors -- i.e. what would it look like if it HAPPENED to have a slightly larger bias toward that configuration.  Ideally, we won't need restraints.
%MRS4: and maybe we do it with just the 3 temperatures first, before adding restraints.  THere are so many variables, we want to be sure we are running simulations which are more likely to give results.
                        \item If we get stable configurations with correct XRD patterns at lower temps but not higher, it may be a consequence of the forcefield %MRS: correct, though it's better if it works ok at 280 vs 260.  We don't want it to start to crystallize.
                        \end{itemize}
                   \item Effect of restraints on stability
		   \begin{itemize}
                        \item Run out simulations with weak restraints (f = 0.1 kJ/mol*nm2) %MRS4: or possibly weaker.  Should work out the RMSD allowed by the restraints, and possibly let them move as much as 1/6 of the way around the circle.  
                        \item Higher restraints may cause the system to be too ordered, but it may be worth running in the case of long term movement
                        \item Run out simulations with no restraints (same simulation as 300K, no restraints from temperature stability sims)
                        \item Hopefully this will let us abandon restraints after equilibration
                   \end{itemize}
		   \item Other simulations
                   \begin{itemize}
                     %MRS4: OK, so this is the same as above, except with no tilt in the monomers?
                        \item Offset, no vapor gap, 20 layer : No tilt (i.e. use monomers that I have previously used). I want to see if the diagonal spots still show up. The concern here is that I wont be able to get the monomers close enough together without causing some unreconcilable energy minimization errors.
%MRS4: it is good to see if spots an appear from the tilt, though it looks like we can get them with layered configurations even w/o the tilt.
                        \item Offset, higher tilt. I'm working on a monomer that tilted more than the one I previously used. Can I make spots appear based purely on tilt? Also, what implications does this have for the pore spacing.
                   \end{itemize}
        \end{itemize}

	The monomers are arranged into disks containing 5 monomers in each layer.
	%BJC: So, it looks like there is some kind of z directional order 
       	%BJC5: This will require some re-evaluation once we come to a reliable starting configuration and equilibration procedure based on the diffraction results
 
	\begin{itemize}
		\item It has been suggested that there are 4 monomers in a each disk/layer, however, simulations have shown that this leads to unstable configurations with dimensions that are too small compared to experiment.
		\item Stable simulations with 6 and 7 monomers have been performed which give structural characteristics consistent with experiment
		\item Stable systems have also been simulated consisting of varying numbers of monomers per layer
		\item Calculating the number density of components in the z direction has shown defined groups % will do for 7 monomers / layer and alternating monomers / layer
		\\ Figure \\
		(a) The number density of benzene ring centers in the z direction shows regions with higher concentrations of the component \\
		(b) A power spectrum of the data in (a) indicates that a new layer is defined every 4 angstroms implying that there are 5 monomers per layer.\\
%MRS4: make it clear that this is calculated from the equilbrium distance. 
%MRS5: in general, there should be some information we can see looking directly at the density, not just the power spectrum.  We will need to convince people there is something there.
		\item Even though there are defined regions of higher component density, there is still considerable variation in the position of the components (shown by the extra noise in the power spectrum)
		\item This suggests that the arrangement of monomers is more complicated than simple layers stacked on top of each other
		\item Density of components along Z axis \\
%MRS4: we need to reconcile the idea ``there are 5 monomers per layer'' and ``there are not clear layers''.  It will be hard for me to reconcile them without knowing a bit more information.
%BJC5: Hopefully I'll have an answer to this soon. For now I'm leaving it untouched
		\item There are not clear layers despite the defined layers in the starting configuration.
		
	\end{itemize}
	
	The arrangement of sodium counterions within the hydrophilic pore regions can be fit to a gaussian distribution. 
	\begin{itemize}
		\item Past literature reports ions arranged in a circle about pore walls implying some kind of void space as a pore. Simulations under a variety of conditions suggest that ions prefer disorder within the pores. 
		\item A size exclusion mechanism has been proposed in the past, however that might no be the only force at play
		\item Ions may play a role in transport, hindering flux of solute and solvent by slowing their diffusion
		\item The membrane may exhibit the permeability-selectivity tradeoff inherent to solution-diffusion type membranes.
		\\FIGURE - plot from grant - sodium, benzene, tail densities in pores
                %MRS3: again, thesis.
%MRS4: we should think a little bit about whether which parts of the alkyl tail we plot, and whether to just focus on the core (ignoring the first 5 A or so from each end).  Depends on what we are trying to show.
		caption: The densities of sodium (blue), benzene (red), and alkyl tail carbons (cyan) surrounding the membrane pore centers indicate that each the pore region is a soup of the three components opposing the previously assumed hollow region. 
	\end{itemize}	
	
	%BJC: New x-ray data here. Need higher resolution simulated data (working on it) and supporting experimental results first.
        %MRS4 got it.  Will need to do some work on making this clear.
	
	The model gives reasonable estimates of ionic conductivity.
	\begin{itemize}
		\item There are a few ways to estimate ionic conductivity as seen in literature. We prefer a method which can extract an estimate based purely on an equilibrium trajectory
		\item We must also be sure that our analysis of results is consistent with the method use for experimental evaluation (i.e. AC impedance spectroscopy)
		\item We must also link our perfectly straight microscopic system to the not-so-straight macroscopic system
                %MRS4: not sure you describe how to do that here.
		%BJC5: We should discuss how we will actually make the link. We talked about it once but kind of left it hanging. It will likely be just some constant that we multiply our number by. To calculate the constant, I think using the azimuthal distribution is a good way to go about it. Although for that we'd need more data points telling us ionic conductivity as a function of azimuthal angle. That data will definitely be hard to come by in the near future so maybe a linear model is an okay estimate for now -- just to have something. For that, we know ionic conductivity with randomly oriented crystalline domains and we know ionic conductivity of a 'perfectly' aligned system (from our simulations). So we can make ionic conductivity as a linear function of alignment, I(x) (Actually it'd be piecewise with opposite slopes on either side of the an azimuthal angle of 90). Then we can integrate the azimuthal distribution weighted by I(x) to get a number to multiply our measurements by. 
		\item Two methods used to for prediction
		\item Nernst Einstein Relation:
		\begin{itemize}
			\item Widely used equation for estimating ionic conductivity
			\item Estimates DC ionic conductivity -- Frequency used during AC impedance slow enough to be approximated by dc at short enough timescales
			\item Relates the diffusive motion of ions in the membrane to the membrane's ionic conductivity
			\item Concentration is concentration of ions in the whole membrane, not just channels
		\end{itemize}
		\item Collective Diffusion:
		\begin{itemize}
			\item Defines a collective coordinate, Q (charge), to quantify the amount of charge transfer through the system
			\item In the limit of infinite time, the MSD of Q can be used to formulate a diffusion coefficient of Q that can be related to ionic conductivity
			\item The model is valid for non-equilibrium and equilibrium simulations. Our analysis is based on the latter
			\item A similar model has been derived and validate to predict water permeability using equilibrium simulations
			\item The pore region is defined as the entire membrane system since lab IC measurements are done on bulk membrane rather than on individual pores. One would expect single channel IC to be much larger than the bulk membrane
		\end{itemize}                    
	\end{itemize}
	TABLE: Calculated ionic conductivity using Nernst-Einsten and Collective Diffusion agree within error
	\\

	The procedure used to create and validate our model can be used to evaluate other liquid crystalline assemblies. Using the design framework and analysis methods applied herin, we have the ability to reliably predict properties of new nanoporous membranes.
	
	\section{Conclusion}
	
	In this work, we have suggested a more detailed picture of the structure of a self assembled thermotropic liquid crystal membrane using an atomistic molecular model.
	\begin{itemize}
		\item The model's physical properties are consistent with experimental measurements
		\item Channels are more disordered than previously thought and are filled with organic matter rather than hollow 
		\item There are likely no defined layers
		\item Results presented for Colh phase monomer but methods are readily adapted to other LCs
	\end{itemize}
	
	\section{Supplemental Information}
	
	Monomer configurations
	\begin{itemize}
		\item show all pore-to-pore equilibration plots used to prove independence of starting conifiguration
		\item 3D visualizations of different configurations tested
		\item 7 monomer per layer configurations and others if needed.
	\end{itemize}
	
	\noindent Crosslinking details
	\begin{itemize}
		\item Algorithm description. Link to full algorithm in git repository
		\item A figure showing the new crosslinks
	\end{itemize}
	
	\noindent Ionic conductivity % plus more details on implementation.
	\begin{itemize}
		\item MSD plots
	\end{itemize}
\end{document}
