\documentclass{article}
\usepackage{graphicx}
\usepackage{wrapfig}
\usepackage{subcaption}
\usepackage[margin=1in]{geometry}
\usepackage{amsmath} % or simply amstext
\usepackage{amssymb}
\usepackage{siunitx}
\usepackage{booktabs}
\usepackage[export]{adjustbox}
\newcommand{\angstrom}{\textup{\AA}}
\newcommand{\colormap}{jet}  % colorbar to use
\usepackage{cleveref}
\usepackage{booktabs}
\usepackage{gensymb}
\usepackage{float}

%BJC: potential titles (uncommented is current favorite)
\title{Understanding the nanoscopic structure of Q\textsubscript{I} phase LLC membranes}
\author{Benjamin J. Coscia \and Michael R. Shirts} 

\begin{document}

  \graphicspath{{./figures/}}
  \maketitle

  \section{Introduction}

  We need highly selective membranes in order to perform efficient separations.

  Amphiphilic molecules are capable of self-assembling into ordered nanostructures.

  Lytropic liquid crystals are a class of amphiphilic molecules that can be cross-linked
  into mechanically strong membranes.
  \begin{itemize}
  	\item H\textsubscript{II} phase lyotropic liquid crystals have densely packed, uniform
	sized pores and have the potential to disrupt conventional membrane separation
	techniques by being selective based not only on size and charge, but on chemical
	functionality as well.
	\item Q\textsubscript{I} phase LLCs consist of a tortuous network of 3D interconnected
	pores.
  \end{itemize}

  We can only learn so much from experiment. MD can give us mechanistic insights with
  atomistic resolution so that we can intelligently design new membranes for 
  solute-specific separations.

  In our previous work, we studied the transport of 20 small polar molecules
  in an H\textsubscript{II} phase LLC membrane. 

  Unfortunately, the timescales that we can simulate with MD are insufficient to be
  able to predict macroscopic transport properties traditionally used to characterize
  membranes in the lab.

  % review of techniques for predicting long time scale behavior

  In this work, we have determined the transport mechanisms and macroscopic
  transport properties exhibited by a number of polar solutes with varying size,
  chemical functionality and hydrophilic character.
  \begin{itemize}
	\item Many of the separations we are interested in involve polar organic 
	compounds.
  \end{itemize} 

  \section{Methods}

  \subsection{Molecular Dynamics Simulations}
  
  \subsubsection{Unit Cell Construction}\label{method:unit_cell}
  
  % BJC: These first 3 paragraphs might be better suited for introduction.
  The space group of the Q\textsubscript{I} phase configuration formed by
  the monomer 1 is thought to be either Pn3m or Ia3d.
  \begin{itemize}
    \item 6 Q phase architectures have been identified in small molecule
    amphiphile systems.~\cite{mariani_cubic_1988}
    \item Of these, only 4 are consistent with diffraction data generated
    by the gemini surfactant used here.~\cite{pindzola_cross-linked_2003}
    %BJC: excess detail for now
    \item Q$^{230}$ (Ia3d), Q$^{224}$ (Pn3m), Q$^{229}$ (Im3m) or Q$^{212}$ (P4\textsubscript{3}32) phase of type I configuration
    \item The presence of $1 / \sqrt{6}$ and $1 / \sqrt{8}$ peaks rules out the
    Q$^{227}$ (Fd3m) and Q$^{223}$ (Pm3n) configurations.
    \item The most likely phase is the type I Q$^{230}$ (Ia3d) because it is the
    most common phase observed between lamellar and hexagonal phases and the
    monomeric alkyltrimethylammonium salts used to synthesize the gemini LCs
    exhibit clear Ia3d symmetry. 
  \end{itemize}
  
  Molecular simulation studies of systems exhibiting bicontinuous cubic 
  phase behavior are typically centered around phase formation
  ~\cite{mondal_self-assembly_2013,ellison_entropy-driven_2006,gonzalez-segredo_coarsening_2004}
  and the few which study transport rely on a previously self-assembled unit cell.
  ~\cite{roy_water_2016}
  \begin{itemize}
    \item Self-assembly is a slow process which usually requires united atom
    or coarser molecular representations.~\cite{klein_large-scale_2008}
    \item Self-assembly also prevents us from creating unit cells with arbitrary
    space group symmetry since one space group is likely to be most energetically
    favorable for a given interaction potential.
    \item Therefore, we developed an initial configuration build procedure that allowed 
    us to place LLC monomers with Ia3d, Pn3m and any other symmetry. 
  \end{itemize}
  
  The interface between the hydrophilic and hydrophobic regions of Pn3m
  and Ia3d Q\textsubscript{I} unit cells can be described using triply-periodic
  minimal surfaces.~\cite{marrink_molecular_2001}  % This ref does what I do for monomer placement
  \begin{itemize}
	\item Schwarz' D surface describes the Pn3m interface while Schoen's gyroid
	describes the Ia3d interface.~\cite{hyde_bicontinuous_1996, andersson_minimal_1988}
	\item Both surfaces can be approximated using implicit trigonometric expressions:~\cite{von_schnering_nodal_1991}
	\item \begin{equation} 
	          sin(x)sin(y)sin(z) + sin(x)cos(y)cos(z) + cos(x)sin(y)cos(z) + cos(x)cos(y)sin(z) = C 
	      \label{eqn:schwarzD}
	      \end{equation}
	\item \begin{equation} 
	          sin(x)cos(y) + sin(y)cos(z) + sin(z)cos(x) = C 
	      \label{eqn:gyroid}
	      \end{equation}
	\item When $C$ = 0, the surface splits the unit cell into two equal volumes. % need reference if I'm actually going to say this
  \end{itemize}
  
  We find points that lie on the implicit surface and then place monomers
  perpendicular to the surface at those points.
  \begin{itemize}
    \item We placed 582 monomers at points on the desired surface of our unit
    cell based on the known experimental density, unit cell dimensions and 
    water content. % glycerol content? need to look this up again
    \item We identified points which lie close to the surface by creating a 
    cubic grid of points and only keeping those whose $(x, y, z)$ coordinates
    lie within a specified tolerance of C when substituted into 
    equation~ref{eqn:schwarD} or~\ref{eqn:gyroid}.
    \item We constructed a vector, $\mathbf{v}$, based on the monomer's constituent
    atom positions which points from the tails to the head groups.
    \item To place each monomer, we randomly chose one of the points on the surface, 
    translated a reference atom on the monomer to that point,
    and rotated the monomer so that $\mathbf{v}$ coincided with the vector normal
    to the surface, calculated based on the gradient of equations~\ref{eqn:schwarzD}
    and~\ref{eqn:gyroid}.
    \item After each monomer placement, we removed all grid points within a specified
    distance of the monomer in order to reduce overlaps in the initial configuration.
    \item This procedure shares similarities with that of Marrink and Tieleman.~\cite{marrink_molecular_2001} % BJC: Just found this reference :( I came up with this independently
  \end{itemize}
  
  Despite our best efforts, overlaps in our initial configuration were unavoidable
  since we used an all-atom model. 
  \begin{itemize}
    \item The simplest way to overcome this problem was to isotropically expand 
    the unit cell by doubling the unit cell vectors in all directions, increasing the unit
    cell volume by a factor of 8. 
    \item In a series of energy minimizations, we slowly shrank the unit cell
    back down to the desired size.
  \end{itemize}
  
  In accordance with experimental procedure, we solvated the unit cell with glycerol at
  70$^{\circ}$ C.
  \begin{itemize}
    \item We used the GROMACS command \texttt{gmx insert-molecules} to add the desired
    number of glycerol molecules to the unit cell.
    \item All of which were placed in the vacant pore region.
    \item In general, we could not insert all of the glycerol molecules intended in one try. 
    \item We performed a series of NVT simulations, inserting more glycerol molecules between
    each.
    \item Once we could no longer fit more glycerol molecules, we switched to the NPT
    ensemble and continued inserting glycerol molecules.
    \item We then allowed the system to equilibrate for an additional 5 ns.
  \end{itemize}
  
  % add glycerol to tails?
  
  Also in accordance with experimental preparation procedure, we cross-linked the 
  monoemr's diene tails at 70$^{\circ}$ C. 
  \begin{itemize}
    \item Practically, cross-linking is necessary for the membrane's mechanical
    integrity which likely will not have a significant influence on the timescales
    which we study. We include it anyways for consistency.
    \item Our cross-linking procedure is iterative. 
    \item Each iteration, we identify carbon atoms that are a part of the diene 
    tails and in close proximity.
    \item We update the topology to include new bonds and update atom types and
    other topological information accordingly.
    \item See Supporting Information for more details.
  \end{itemize}
   
  \section{Results and Discussion}

  \section{Conclusion}
 
  \section*{Supporting Information}

  Detailed explanations and expansions upon the results and procedures mentioned in
  the main text are described in the Supporting Information. This information is
  available free of charge via the Internet at http://pubs.acs.org.

  \section*{Acknowledgements}

  Molecular simulations were performed using the Extreme Science and
  Engineering Discovery Environment (XSEDE), which is supported by National
  Science Foundation grant number ACI-1548562. Specifically, it used the Bridges
  system, which is supported by NSF award number ACI-1445606, at the Pittsburgh
  Supercomputing Center (PSC). This work also utilized the RMACC Summit supercomputer,
  which is supported by the National Science Foundation (awards ACI-1532235 and
  ACI-1532236), the University of Colorado Boulder, and Colorado State
  University. The Summit supercomputer is a joint effort of the University of
  Colorado Boulder and Colorado State University.

  \clearpage

  \bibliographystyle{ieeetr}
  \bibliography{BCC}

  \newpage

  \section*{TOC Graphic}

\end{document}
