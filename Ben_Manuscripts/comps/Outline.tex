\documentclass{article}
\usepackage{graphicx}
\usepackage{wrapfig}
\usepackage{subcaption}
\usepackage[margin=1in]{geometry}
\usepackage{amsmath} % or simply amstext
\usepackage{amssymb}
\usepackage{siunitx}
\usepackage{booktabs}
\usepackage[export]{adjustbox}
\usepackage{cleveref}
\usepackage{booktabs}
\usepackage{gensymb}
\usepackage{float}
\usepackage[x11names,table]{xcolor}
\renewcommand{\baselinestretch}{1.5}
\usepackage{url}
\newcommand{\foo}{\color{LightSteelBlue3}\makebox[0pt]{\textbullet}\hskip-0.5pt\vrule width 1pt\hspace{\labelsep}}
\DeclareCaptionFont{blue}{\color{LightSteelBlue3}}
\usepackage{array,booktabs}

\title{PhD Research Proposal: Molecular Level Design of Nanoporous Lyotropic Liquid
Crystal Membranes for Aqueous Separations \\ \vspace{0.5cm}
\large Advisors: Michael Shirts and Richard Noble}

\author{Benjamin J. Coscia} 

\begin{document}

  \maketitle
  \thispagestyle{empty}
  \clearpage
  \setcounter{page}{1} % don't count title page as a page 

  \section{State of the Art}\label{section:state-of-the-art}

  \subsection*{Commercial Membranes for Small Molecule Separations}
  
  % perhaps a word or two about micro and ultra filtration  
  
  % re-word. This paragraph is verbatim from structure paper
  More highly selective nanoporous membranes would be extremely useful
  in performing complex aqueous separations with seawater and various
  types of wastewater.
  \begin{itemize}
    \item For example, Sodium chloride and boron in seawater 
    \cite{fritzmann_state---art_2007} and organic micropollutants found in
    municipal and industrial wastewaters \cite{schwarzenbach_challenge_2006}
    represent just a few of the diverse contaminants of water sources. 
    \item By efficiently separating contaminants from feed solutions with
    highly selective membranes, it is possible to reduce the number of 
    required membrane passes and post-treatment steps needed for a given 
    filtration process \cite{werber_materials_2016}, thus lowering energy
    requirements. 
    \item Additionally, one can also extract valuable resources from the 
    feed streams. For example, flowback water produced during hydraulic
    fracturing of shale formations contains dissolved species such as acetate
    whose extraction has economic value \cite{dischinger_application_2017}.
  \end{itemize}

  Reverse osmosis (RO) and nanofiltration (NF) are two prevailing membrane
  filtration processes that can be used to separate solutes on the order of
  1 nm in size and smaller, including ions.
  \begin{itemize}  
    \item Both apply hydraulic pressure to the feed solution in order to 
    overcome osmotic pressure and force water and unfiltered components 
    through the membrane.
    \item RO membranes are typically thin film composite with a porous mechanical 
    support layer and a thin but dense polymer matrix active layer where separations
    occur.\cite{jeong_interfacial_2007}
    \item RO separates solutes based on the solute's ability to dissolve into
    and diffuse through the tortuous pathways available in dense active layer.
    \item RO offers high selectivity at the cost of relatively high energy 
    requirements since one must apply a large hydraulic pressure in order to 
    achieve an economical flux.  % get number
    \item In contrast to RO membranes, NF membranes have explicit pores on the
    order of 1 nm in size. 
    \item Typically, separations are achieved based on size exclusion and
    charge exclusion if the membrane surface is charged.
    \item NF membranes require significantly less applied pressure in order
    to achieve solute flux comparable to RO.
    \item Unfortunately, conventional synthesis processes, such as 
    phase-inversion\cite{smolders_microstructures_1992} are stochastic in
    nature which yields pores that are polydisperse in size.\cite{werber_materials_2016} % name some synthesis processes
    \item Pore size polydispersity is detrimental to membrane selectivity
  \end{itemize}
  
  The downfall of RO and NF membranes can be summarized by the well-known
  permeability-selectivity tradeoff. Namely, it is difficult to increase the
  permeability of a desired molecular or atomic species, while maintaining
  the same retention of an undesired species.\cite{werber_materials_2016}  
  
  \subsection*{Nanostructured Membranes}
  
  Nanostructured membranes attempt to overcome the permeability-selectivity 
  tradeoff through molecular level design.
  \begin{itemize}
	\item Graphene sheets, carbon nanotubes (CNTs) and zeolites are three
	highly studied nanostructured technologies.
  \end{itemize}
  
  Graphene membranes are an extremely active are of research because 
  they offer potential for extremely high permeability membranes.
  \begin{itemize}
    \item 2D materials
    \item Functionalization?
    \item Easy to introduce defects
    \item Multi-layered
  \end{itemize}
  
  Carbon nanotubes have shown promise due to unprecedentedly fast
  water transport. 
  \begin{itemize}
    \item Functionalization of nanotubes
    \item Aligned nanotubes difficult to synthesize
  \end{itemize}
   
  Zeolite-coated ceramic membranes offer the potential for permeabilities
  comparable to ultrafiltration with selectivities as good as NF and RO. 
  \begin{itemize}
    \item Zeolites have highly uniform nm-sized crystalline structures with
    cage-like cavities that allow movement and trapping of small solutes. \cite{pendergast_review_2011}
    \item The crystalline frameworks are typically formed by networks of silicon
    and aluminum each attached to 4 oxygen atoms in a tetrahedral arrangement. 
    \item One can replace the silicon and aluminum atoms via ion exchange 
    in order to control the size of the cavities and hence its molecular-seiving properties.
    \item A number of studies have tested the permeability and sodium salt
    rejection of various zeolite membranes, however none have fully
    overcome the permeability-selectivity tradeoff.
    \item Most are prone to defects in the crystalline structure.
  \end{itemize}

  \subsection*{Lyotropic Liquid Crystal Membranes}
  
  Preliminary evidence has shown that cross-linked lyotropic liquid crystal
  (LLC) membranes can be produced at moderate scale and may be capable of 
  performing highly selective separations. 
  \begin{itemize}  
    \item LLCs are amphiphilic molecules that have the ability to self-assemble
    into porous nanostructures \cite{smith_ordered_1997} and can be cross-linked
    to create mechanically strong membrane films with periodic pores on the
    order of 1 nm in diameter \cite{zhou_supported_2005}. 
    \item LLC membrane pores are uniform in size because they are formed by 
    self-assembly. 
    \item Since LLC polymer membranes lack an appreciable pore size distribution,
    they inherently exhibit high selectivity due to their strict molecular weight
    cut-off (MWCO)~\cite{zhou_supported_2005}. 
    \item Additionally, LLC monomers can be salts, and therefore lead to Donnan
    exclusion of ions in solution.\cite{donnan_theory_1995}
%    The membrane gains a net surface charge when counterions from
%  the head groups that line the pore walls escape into the feed solution in an
%  effort to balance the gradients of concentration and electric potential
%  \cite{donnan_theory_1995}.
  \end{itemize}

  The feasibility of nanostructured LLC polymer membranes for selective separations
  has been demonstrated using LLC monomers that form the type 1 bicontinuous cubic
  (Q\textsubscript{I})\cite{hatakeyama_water_2011,hatakeyama_nanoporous_2010,carter_glycerol-based_2012}
  and the inverted hexagonal (H\textsubscript{II}) \cite{zhou_supported_2005}
  phases. 
  \begin{itemize}
    \item When separating organic solutes from NaCl, Q\textsubscript{I}-phase
    membrane filtration experiments have shown selectivity 2--3 times higher than
    commercial RO and 6--12 times higher than commercial NF membranes.\cite{dischinger_application_2017}
    \item When separating a series of various sized dyes, the 
    H\textsubscript{II}-phase membrane showed complete rejection of dyes bigger
    than 1.2 nm in size \cite{zhou_supported_2005}.
  \end{itemize}

  The H\textsubscript{II}-phase pore geometry (Figure~\ref{fig:assembly}) has a
  higher theoretical capacity for transport than the Q\textsubscript{I} phase.
  \begin{itemize}
	  \item The H\textsubscript{II} phase forms at room temperature in the 
	  presence of ca.~10 wt\% water and consists of hexagonally packed, 
	  hydrophilic pore columns\cite{smith_ordered_1997}. 
	  % BJC: should rethink the following since we think that thermotropic doesn't exist
	  \item In the absence of water, neat monomer will form the same hexagonal
	  columnar structure which, in the literature, has been referred to as the
	  Col\textsubscript{h} thermotropic phase\cite{feng_scalable_2014}.
  \end{itemize}

  Q\textsubscript{I}-phase membranes consist of a tortuous network of three
  dimensionally interconnected pores that prevent optimal through-plane
  transport. 
  \begin{itemize}
    \item In contrast, the densely packed, non-tortuous and uniform sized
    pores of H\textsubscript{II}-phase membranes represent the ideal geometry
    for achieving high solute flux\cite{matyka_tortuosity-porosity_2008}.  
    \item However, the hexagonally packed LC domains of the H\textsubscript{II}-phase
    generally form mutually unaligned domains, which hurts membrane permeability. 
    \item This domain scale misalignment had inhibited further development of 
    this technology, and research efforts were focused on the Q\textsubscript{I}
    phase, whose geometry does not require alignment~\cite{zhou_new_2007}.
  \end{itemize}

  Recently, researchers have learned how to macroscopically align the
  hexagonal domains which has revived research into H\textsubscript{II}-phase LLC
  polymer membranes. 
  \begin{itemize}
    \item In 2014, Feng et al.~showed that one can align Col\textsubscript{h}
    domains, created by the ``dry" monomer Na-GA3C11, using a magnetic field
	with subsequent cross-linking to lock the structure in place\cite{feng_scalable_2014}.
	\item In 2016, Feng et al.~showed that one could also obtain the same result
	by confining the neat monomer between PDMS or glass substrates since hexagonal
	mesophases preferentially anchor perpendicular to both surfaces\cite{feng_thin_2016}.
  \end{itemize}
  
  Unfortunately, reproducing the work of Feng et al. with the H\textsubscript{II}
  has been an experimental challenge. Therefore, the primary focus of experimental
  research efforts has been with the Q\textsubscript{I} phase.

  \section{Project Objectives}\label{section:objectives}  
  
  Our current understanding of the molecular details of LLC membranes'
  nanostructure is not sufficient to be able to precisely design them for
  specific separations. 
  \begin{itemize} 
%    \item Over the past 20 years, H\textsubscript{II}-phase LLC polymer 
%    membrane studies have been limited primarily to the Na-GA3C11 monomer
%    with some characterization done after minor structural modifications.\cite{zhou_supported_2005,resel_structural_2000}
    \item Dischinger et al.~attempted to use an empirical model
    that correlates the physiochemical properties of the counterion used in
    a Q\textsubscript{I}-phase LLC membrane to solute rejection.\cite{dischinger_effect_2017}
    \item Although their model showed some qualitative agreement with experiment, the
    quality of fit of their model was limited due to complex solute-membrane
    interactions that could not easily be modeled. 
  \end{itemize}

  A molecular-level understanding of structure of and transport in LLC polymer 
  membranes, enabled by molecular dynamics (MD) simulations, can provide 
  guidelines to reduce the large chemical space available to design
  monomers for creation of separation-specific membranes. 
  \begin{itemize}
    \item Using a sufficiently accurate molecular model, we can directly 
    observe transport of solutes within LLC membrane nanopores and infer
    mechanisms. 
    \item Based on this information we can intelligently design new membranes
    by screening new liquid crystal monomer designs.
    \item Our most promising designs can be tested experimentally.
  \end{itemize}
  
  \noindent There are five primary objectives of our work.
  \begin{enumerate}
  
    \item Develop techniques to build and understand the nanoscopic structure
    of LLC membranes.
    
    Useful molecular-level modeling should incorporate a detailed picture 
    of the nanoscopic pore structure, which is crucial to understanding
    the role of monomer structure in solute transport and membrane design.
    We will generate simulated X-ray diffraction patterns from MD trajectories
    and compare them to an experimental 2D wide angle X-ray scattering (WAXS)
    spectrum of a Col\textsubscript{h} phase membrane. We will assess the extent
    to which we can apply our understanding to the H\textsubscript{II} phase,
    as well as systems built with alternate monomers.
    
    \item Determine dominant solute-membrane interactions that give rise to
    transport mechanisms.
    
    We will observe transport of a relatively large set of small polar solutes
    placed within the H\textsubscript{II} membrane nanopores. We will observe
    the time series of each solute's position as well as directly measure the
    physical interactions, such as hydrogen bonding and ion coordination, 
    between solutes and LLC monomers.
    
    \item Create a stochastic model which can project long timescale 
    transport behavior.
    
    We will combine our general knowledge of the solute transport mechanisms
    with simulation data in order to inform a stochastic model. This model
    should closely reproduce the time series that we observe in our simulations.
    Due to the low computational cost of a stochastic model relative to 
    MD simulations, we will be able to project long timescale transport behavior
    and make well-converged predictions of macroscopic transport properties.
           
    \item Adapt the same analysis to the Q\textsubscript{I} phase.
    
    Over the course of this project, experimental research surrounding
    LLC membranes has shifted nearly all focus towards the Q\textsubscript{I}
    phase due to its more facile synthesis.	Although most of our work has 
    been applied to the H\textsubscript{II} phase, we expect that much of 
    the same analyses can be applied to the Q\textsubscript{I} phase. The 
    biggest challenge will be adapting our techniques to its more complex
    geometry.
    
    \item Enable easy continuation of our work with a dedicated and 
    well-documented python package.
    
	Although molecular simulations have become popular
	for studying systems at the atomic level, LLCs used in this context
	have not been heavily investigated. Consequently, much of the 
	analysis developed for this project is not widely applied. Therefore, 
	it is important for us to make available the scripts that reproduce
	the exact results presented in our published papers along with detailed
	documentation of the scripts. This will ensure near-seamless continuation
	of this project and accelerate the development of LLC membranes.
    
  \end{enumerate}

  \section{Progress to Date}\label{section:progress}
  
  \textbf{\large Objective 1:} \textit{\large Build and understand nanoscopic structure of LLC membranes} (\textcolor{green!40!olive}{\textbf{Complete}})
  
  \noindent We have developed a procedure for building an LLC membrane unit cell.
  
  \noindent We have created a molecular model that is maximally consistent with
  experimental WAXS data.
  
  \noindent Water is necessary to fully reproduce all features in the WAXS pattern.
  
  \noindent An ensemble of simulation trajectories best capture the system's 
  time averaged behavior.
  
  \noindent \textbf{\large Objective 2:} \textit{\large Determine transport mechanisms} (\textcolor{green!40!olive}{\textbf{Complete}})
  
  \noindent We added additional water to our model to create the H\textsubscript{II} phase
  
  \noindent Water partitions into the distal tail region
  
  \noindent The pore structure changes
  
  \noindent We place 20 small polar solutes 
  
  \noindent The mean squared displacements are not a monotonic function of solute MSD
  
  \noindent Solute behavior is subdiffusive.
  
  \noindent Subdiffusive behavior is a consequence of a hopping and trapping mechanism.
  
  \noindent There are three mechanisms of entrapment.
  
  \noindent \textbf{\large Objective 3:} \textit{\large Create a stochastic model} (\textcolor{blue}{\textbf{In Progress}})
  
  \noindent We are in the process of developing the theory required to build a 
  stochastic model. 
  
  \noindent The distribution of dwell times is power law distributed.
  
  \noindent The distribution of hop lengths is approximately gaussian
  
  \noindent The model will likely have a radial dependence. 
  
  \noindent \textbf{\large Objective 4:} \textit{\large Apply analyses to Q\textsubscript{I} phase} (\textcolor{blue}{\textbf{In Progress}})
  
  \noindent We have developed a procedure to build the Q\textsubscript{I} phase.
  
  \noindent \textbf{\large Objective 5} \textit{\large Create a well-documented python package} (\textcolor{blue}{\textbf{In Progress}})
  
  Python scripts used to simulate X-ray diffraction patterns are available in the following
  GitHub repository: \url{https://github.com/joeyelk/MD-Structure-Factor}. Documentation
  is forthcoming. 
  
  Python scripts used to conduct all other post-simulation trajectory analysis are 
  also available on GitHub: \url{https://github.com/shirtsgroup/LLC_Membranes}. 
  Documentation is a work in progress. It is available for viewing in its 
  current state at \url{https://llc-membranes.readthedocs.io/en/latest/}.

  \section{Timeline for Completion of Objectives}\label{section:timeline}

  A schematic of the estimated timeline that will be followed for the completion of
  tasks pertinent to finishing all objectives is given in Table~\ref{table:timeline}.
  \begin{itemize}
    \item Simulations required to study the structure of the bicontinuous cubic
    phase will be run in parallel while working out the details of our stochastic
    model of transport in the H\textsubscript{II} phase.
    \item Simulations and analysis required for Q\textsubscript{I} phase solute 
    transport studies analogous to those of Objective 2 will be carried out 
    throughout Fall 2019.
    \item Finalization of code documentation and application of a stochastic
    model to the Q\textsubscript{I} phase will be finished by May 2020
  \end{itemize}

  \begin{center}  
  \begin{table}[!htb]
	\renewcommand\arraystretch{1.4}\arrayrulecolor{LightSteelBlue3}
	\captionsetup{singlelinecheck=false, font=blue, labelfont=sc, labelsep=quad}
	\caption{Estimated Timeline for Completion of Objectives}\vskip -1.5ex
	\begin{tabular}{@{\,}r <{\hskip 2pt} !{\foo} >{\raggedright\arraybackslash}p{8cm}}
	\toprule
	\addlinespace[1.5ex]
	August 2019 & Complete Stochastic Model for H\textsubscript{II} phase \\
	September 2019 & Finalize Q\textsubscript{I} phase structure \\
    January 2020 & Finish transport study of Q\textsubscript{I} phase \\
    April 2020 & Complete code documentation \\
    April 2020 & Finish application of stochastic model to Q\textsubscript{I} phase \\
    May 2020 & PhD Defense \\
	\end{tabular}
	\label{table:timeline}
  \end{table}
  \end{center}
  
  \section{Resource Requirements}\label{section:resources}
  %BJC: Not sure what to include here. Hour estimate? Or omit this section.   
  
  The remainder of our work will require the use of high performance computing (HPC)
  resources. 
  \begin{itemize}
    \item We will continue using Bridges, an XSEDE resource as well as Summit, a
    supercomputer located at CU Boulder.
  \end{itemize}
  
  % BJC: add funding?

  \newpage
  \bibliographystyle{ieeetr}
  \bibliography{comps}

\end{document}
