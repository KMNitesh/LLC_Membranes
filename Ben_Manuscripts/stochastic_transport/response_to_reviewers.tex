\documentclass{article}
\usepackage[margin=1in]{geometry}
\usepackage{amsmath}
\usepackage{xr}
\usepackage{xcolor}
\usepackage{siunitx}
\usepackage{gensymb}
\usepackage{graphicx}
\usepackage{wrapfig}
\usepackage{subcaption}
\usepackage{enumitem}
\usepackage{scrextend}
\usepackage[normalem]{ulem}

\externaldocument[S-]{Supporting_Information}
\externaldocument[M-]{Final_Draft}

\begin{document}

\graphicspath{{./figures/}}

\begin{center}
\textbf{Response to reviewers: Capturing Subdiffusive Solute Dynamics and Predicting
Selectivity in Nanoscale Pores with Time Series Modeling} \\
Authors: Benjamin J. Coscia and Michael R. Shirts
\end{center}

We thank the reviewers for carefully reading over our manuscript and providing
helpful comments. We have taken the suggestions into consideration and made
appropriate revisions to the manuscript document. The comments have been reproduced
in italics and all changes to the text have been documented below. 
%BJC: Need to modify this to be more in line with what I did.
In cases where we modified the text we included the original text (denoted by ``Original text:") 
followed by the new text (denoted by ``New text:").

\section*{Response to Reviewer 1}

\begin{enumerate}[label={Comment \theenumi :}, leftmargin=3.9\parindent]

  \item \textit{\textbf{Page 5 line 19:} In this work, we use the output of our MD simulations
        to construct \textbf{two classes} of mathematical models which aim to predict membrane 
        performance while providing quantitative mechanistic insights. \\
        \textbf{Page 5 line 47:} We treat our system in terms of \textbf{two well-known} classes of 
        anomalous subdiffusion: \textbf{fractional Brownian motion} (FBM) subordinate to a 
        \textbf{continuous time random walk (CTRW)}, or subordinated FBM (sFBM) for short. \\ \\
        I am a little confused by these sentences, does the two well-known classes correspond to FBM
        and CTRW, or refer to different classes of modeling approaches used later in the manuscript?}
  
  %BJC: use a different word?
  
  Author reply: We thank the reviewer for identifying our potentially confusing choice of words.
  The first sentence (Page 5 line 19) refers more broadly to the two types of stochastic models 
  formulated in this work: the Anomalous Diffusion model and the Markov state-dependent dynamical
  model. The second quoted sentence (Page 5 line 47) refers to the different types of
  motion (FBM and CTRW) that give rise to anomalous diffusion specifically. The paragraph expands
  upon the introduction given in the previous paragraph. We have modified the second quoted 
  sentence as follows to remove ambiguity.
  
  Original text:
  \begin{quote}
  We treat our system in terms of two well-known classes of anomalous subdiffusion: 
  \end{quote}
  
  New text:
  \begin{quote}
  In this first approach, we treat our system in terms of two well-known types of molecular
  motion that leads to anomalous subdiffusion: 
  \end{quote}
  
  \item \textit{\textbf{MD-section} \\
        \textbf{Page 8 and 9} \\
        The authors refer to ref 13 (where these simulations have been performed before), 
        nevertheless I find that the MD section does not provide sufficient amount of details 
        to be self-contained. Could the authors specify the force field, as well as summarize 
        its reliability for this particular system (often simulations involving changed solutes
        through pores require polarizable force fields)? Next, reading this section I would 
        also like to know how well finite size effects are controlled for with this specific 
        membrane and these specific solutes. Lastly, it would be natural to report the size 
        of the system in this section (I found it eventually in the introduction after 
        searching for it throughout the manuscript).}

  Author reply: We thank the reviewer for pushing us to include more details about our 
  molecular simulations in order to boost reproducibility. The length of the manuscript made
  us wary to include information that could be accessed elsewhere, however there are 
  benefits to keep this very important section self-contained.
  
  We added the following text in order to inform the reader of the system's size:
  
  New text:
  \begin{quote}
  Each unit cell contains $\sim$ 62,000 total atoms. % BJC: do you think they want unit cell dimensions? X x Y x Z? 
  \end{quote}
  
  We added the following text to address comments related to our choice of force field:
  
  New text:
  \begin{quote}
  We parameterized monomers and solutes using the Generalized Amber Force Field (GAFF) and
  water molecules using the TIP3P water model.~\cite{} We made each choice in order to
  ensure that our approach is accessible and reproducible to a wide audience and so that
  it is easy to study a broad range of solutes and monomer chemistries in high throughput. 
  The main goal of this work is to develop stochastic models which can reproduce solute
  behavior on MD time scales regardless of the force field. In future work, it may be 
  beneficial to employ specialized force fields, like polarizable force fields in 
  order to improve the accuracy of selectivity predictions.
  \end{quote}
  
  %BJC: not sure if anything should be added to text
  %BJC: I could use some help on justification
  With respect to finite size effects, we have not made any explicit corrections, but
  feel they are not of great concern for this work. While the tails of the monomers from
  different pores interact across periodic boundaries, we are not interested in their
  diffusion. In our previous work, we have shown that monomer motion is negligible on 
  simulation time scales. Our main focus is on the time series of the solutes in the
  membrane pores. On the 5 $\mu s$ time scales which we simulate, we do not observe any 
  solutes that cross between pores. Therefore, we expect similar solute behavior in larger
  arrays of pores. A study of finite size effects may be interesting in future work, but
  we do not feel it is justified for this study given the computational cost of studying
  larger systems. We also believe it would have little influence on the way in which 
  we fit our models to the time series.
        
  \item \textit{\textbf{Figure 5 page 25:} \\
		First, the symbols within the figure are very small and barely readable, and probably not
		according to authors guidelines for figures. Second, the caption, as with the other captions of
		this manuscript, contain a lot of interpretation that normally are placed in the main text. This
		might be a particular style of the authors that has its benefits, however in this specific case the
		caption fails at explaining what exactly this figure is showing. The way I interpret this figure is:
		the symbol on the y-axis marks a general parameter and the boldface symbol marks the
		parameter shown in the histogram. Am I correct? If so, I would strongly recommend modifying
		y-axis and the caption to make this clear.}
		
		Author reply:
		
		We thank the reviewer for their constructive feedback. The fontsize in the figure is indeed too small.
		We have modified the plots in the figure to have larger, more readable font. We also took this 
		opportunity to increase the font size on Figures 6--8.
		
		The reviewer is correct in their interpretation of the figures. For clarity, we modified the y axes as
		well as the caption in order to further aid the reader. We also made efforts to cut down on the interpretation
		that is already presented in the main text. We made similar modifications to Figure 6 since it is an
		analogous figure. Our caption changes are presented below.
		
		This comment caused us to reconsider the detail in the captions of other figures as well. We feel
		the level of the detail is justified in most cases. However, we decided to remove some redundant
		information from Figure 13, as shown below.
		
		Figure 5, original caption:
		
		\begin{quote}
		
		  The parameters of the one mode model reveal differences in dynamics
	      between solutes. (a) We parameterized Gaussian,
	      $\mathcal{N}(\sigma)$, and L\'evy stable, $L(\sigma, \alpha_h)$,
	      distributions to describe solute hop lengths. We assume the mean
	      ($\mu$) to be zero for these distributions and no
	      skewness ($\beta = 0$) in the L\'evy stable distributions. High
	      values of $\sigma$ and lower values of $\alpha_h$ result in larger
	      hops. (b) We parameterized a pure power law, $P(\alpha)$, and a
	      truncated power law, $P_T(\alpha, \lambda)$, distribution to describe
	      solute dwell times. Lower values of $\alpha$ lead to heavier power
	      law tails and higher values of $\lambda$ truncate the distribution at
	      lower dwell times. (c) Finally, we parameterized the hop
	      autocorrelation function, $\gamma(H)$, to describe the degree of
	      correlation between hops. Simulations with higher values of $H$ display
	      behavior closer to the Brownian limit.
	      
		\end{quote}
		
		Figure 5, modified caption:
		
		\begin{quote}
		
		  The parameters of the one mode model reveal differences in dynamics
	      between solutes. (a) We parameterized Gaussian, $\mathcal{N}(\sigma)$, 
	      and L\'evy stable, $L(\sigma, \alpha_h)$, distributions to describe
	      solute hop lengths. 
		  % BJC: added	      
	      Each bar represents the value of a single parameter,
	      highlighted in bold, of the associated hop distribution.
		  % BJC: removed	      
	      % We assume the mean ($\mu$) to be zero for these distributions and no
%	      skewness ($\beta = 0$) in the L\'evy stable distributions. 
	      In general, higher values of $\sigma$ and lower values of $\alpha_h$ 
	      result in larger hops. 
	      (b) We parameterized 
	      % BJC: removed. I think this was left over from when the pure power law was still in the paper
	      % a pure power law, $P(\alpha)$, and 
	      a truncated power law distribution, $P_T(\alpha, \lambda)$,  to describe
	      solute dwell times. 
		  % BJC: not sure if should reiterate the bold highlights here.	      
	      Lower values of $\alpha$ lead to heavier power
	      law tails and higher values of $\lambda$ truncate the distribution at
	      lower dwell times. (c) Finally, we parameterized the hop
	      autocorrelation function, $\gamma(H)$, to describe the degree of
	      correlation between hops. Simulations with higher values of $H$ display
	      behavior closer to the Brownian limit.
	      
		\end{quote}
		
		Figure 6, original caption:
	 
	    \begin{quote}

		  The two mode model parameterizes solute behavior in the pore and tails
	      separately. We consider solutes to be within the pore region if they
	      are 0.75 nm from a given pore center, otherwise they are in the
	      tails. (a) Generally, movement is much more restricted in the tail
	      region, parameterized by lower $\sigma$ values (smaller hops) for the
	      Gaussian and L\'evy stable distributions. Values of $\alpha_h$ are
	      significantly lower for urea and acetic acid meaning there is a
	      larger probability that they will take large hops. (b) Dwell times
	      are longer in the tails. Lower values of $\alpha$ correspond to power
	      laws with heavier tails and thus higher probabilities of long dwell
	      times. There is no easily discernible trend in $\lambda$ of the
	      truncated power law distribution. Note that we used the same Hurst
	      parameter for both modes (shown in Figure~\ref{M-fig:hurst_barchart})
	      due to a low number of sufficiently long sequences of hops in each
	      mode. (c) Solutes spend various amounts of time in the tail and pore
	      region dependent on their size, shape and chemical functionality.
	      Methanol's small size favors occupation of the much larger accessible
	      volume in the tails. Urea and acetic acid are fairly stable in both
	      regions since they are small and hydrophilic. Ethylene glycol has a
	      slight preference for the pores likely because it is a larger
	      molecule with two hydrophilic hydroxyl groups.
	  
        \end{quote}
	
	    Figure 6, modified caption:
	 
	    \begin{quote}
		  % Figure 6 caption
		  The two mode model parameterizes solute behavior in the pore and tails
	      separately. We consider solutes to be within the pore region if they
	      are 0.75 nm from a given pore center, otherwise they are in the
	      tails. 
	      %BJC: added next two sentences
	      For (a) and (b), each bar represents the value of a single parameter, 
	      highlighted in bold, of the associated hop or dwell time distribution.
	      Parameterizations of solute motion in the pores are visualized as bars
	      in the upper half of the plots, while the parameters describing motion 
	      in the tails are reflected into the lower half of the plots.
	      %BJC: modified this sentence to cut down on detail
	      (a) Generally, movement is more restricted in the tail region, but 
	      with occasionally large hops, as implied by lower $\sigma$ (smaller
	      hops) and $\alpha_h$ (heavier tails of the hop distribution) values.
	      %for the Gaussian and L\'evy stable distributions. 
%	      Values of $\alpha_h$ are significantly lower for urea and acetic acid
%	      meaning there is a larger probability that they will take large hops.
	      (b) Dwell times are longer in the tails as implied by lower values 
	      of $\alpha$.
	      %correspond to power laws with heavier tails and thus higher probabilities of long dwell
	      %times. 
	      There is no easily discernible trend in $\lambda$ of the
	      truncated power law distribution. Note that we used the same Hurst
	      parameter for both modes (shown in Figure~\ref{M-fig:hurst_barchart})
	      due to a low number of sufficiently long sequences of hops in each
	      mode. (c) Solutes spend various amounts of time in the tail and pore
	      region dependent on their size, shape and chemical functionality.
	      %BJC: removing details that is already in the text
%	      Methanol's small size favors occupation of the much larger accessible
%	      volume in the tails. Urea and acetic acid are fairly stable in both
%	      regions since they are small and hydrophilic. Ethylene glycol has a
%	      slight preference for the pores likely because it is a larger
%	      molecule with two hydrophilic hydroxyl groups.
	  
        \end{quote}
        
        Figure 13, modified caption:
        
        \begin{quote}
        
          The selectivity between pairs of species changes monotonically with
	      pore length. The strength of dependence on pore length depends on
	      the difference between $\beta$ values. 
		  %BJC: All of this is in the text.	      
	      \sout{The largest differences in solute
	      flux result in high selectivities at any pore length. This membrane
	      may be a good candidate for the separation of ethylene glycol from
	      acetic acid. Ethylene glycol has the lowest $\beta$ value while
	      acetic acid has the second highest, leading to strong length
	      dependence. Ethylene glycol also has the highest flux and acetic acid
	      has the lowest resulting in relatively high selectivities independent
	      of pore length.}
	  
        \end{quote}
		
  \item \textit{\textbf{Page 6 line 55}: The second measure of success is based on the qualitative
        comparison between individual realizations of solute trajectories generated ...\\ \\
		Here I am a little curious. Are there no measures which can be used to quantify the similarity of
		the trajectories, such as autocorrelation functions etc, or are such measures inputs for the
		parametrization of the model which cannot be used?}
		
		Author reply: In short, we do not believe there are any other suitable methods for comparing
		similarity beyond the checks which we show in this work. Autocorrelation is indeed an input 
		to the anomalous diffusion and Markov state-dependent dynamical model, specifically through
		the Hurst parameter which is generated by fitting to autocorrelation of solute jumps. The
		anomalous diffusion model is only parameterized by the distribution of hop length, the 
		distribution of dwell times and the degree of autocorrelation between hops, all of which
		are measured directly from MD. It may be possible to think of other metrics of success
		for the Markov state-dependent dynamical model, but it fails the qualitative test and
		thus does not merit any further checks.
		
  \item \textit{One question that came to mind reading the paper was: why is the model only in 
  		one dimension and not two (for example using cylindrical coordinates)?}
  		
  		Author reply: We thank the reviewer for their sensible question. The simplest answer
  		to this is that ultimately, we are only concerned with transport along the pore axis. Radial
  		solute transport (with respect to the pore center) is effectively bounded and only 
  		contributes to the axial MSD to the extent that it influences axial transport. We 
  		do acknowledge that the radial location of solutes has an influence on solute
  		hopping behavior which is why attempted to create the two mode anomalous diffusion model.
		
		From a more mathematical standpoint, there are two factors which pose challenges with
		increasing dimensionality. First, we have a relatively low amount of data, and therefore
		it is challenging to adequately parameterize behavior as a continuous function of both 
		$r$ and $z$. Second, it is not trivial to simulate fractional motion in multiple dimensions.
		We were fortunate to find an efficient python implementation of FBM and we needed to
		develop our own approximation of 1D fractional L\'{e}vy motion based on literature. 
		We felt there was little to gain by models more complicated than those showcased in our
		work.
		
		However, we would like to acknowledge that we have a strong desire to incorporate data
		from multiple dimensions. We are actively investigating a new modeling approach which 
		allows us to overcome these issues and to use cylindrical representations with ease.

  \item \textit{\textbf{Page 35 line 12-14}: \\
   		We do not consider the two mode AD model because it has a broken correlation structure \\ \\
        I was a little disappointed to see only the single mode being used. Could the authors 
        elaborate on this point? Exactly how does this broken correlation structure hinder such
        an investigation?}
        
        Author reply: We are pleased with the reviewer's interest in the two-mode model. While it
        would be technically simple to apply the same mathematical framework to study the long time
        scale performance of the two-mode model, we feel it may be misleading to the reader.
        As described in the text, the predicted MSDs of the two-mode model may indeed show better 
        qualitative MSD behavior than the one-mode model. Specifically, the two-mode model becomes
        linear on long timescales which may be more consistent with our MD results. However, this 
        long time scale linear behavior of the two-mode model has only a mathematical basis and
        no physical basis. As discussed in the text, the one-mode model has persistent curvature
        since hop correlation persists indefinitely for fractional motion. However, for the two-mode
        model, each time a solute transitions between the pore and tail regions, we must generate a
        new FBM process because we lack a proven technique for simulating fractional 
        motion that incorporates time-dependent variation in the width of the hop length 
        distributions.
        
        Perhaps, most importantly, Section 3.3 is primarily used to demonstrate our mathematical
        framework. Applying it to the two mode model would be trivial. We chose to use the one-mode 
        model for demonstration because it passes our qualitative and quantitative checks and is
        mathematically sound. As the reviewer points out, it is likely that our approach to 
        modeling the solute time series is not yet suitable to make quantitative predictions of 
        flux and selectivity. But our approach opens the door for applying the same analysis to 
        improved time series modeling techniques.
         
  \item \textit{\textbf{Page 35 line 20-42:}\\
        A pore-length of 50 nm is a little underwhelming compared to 7 micrometers. Reading the paper
		I do not understand what lies behind this limitation - what hinders the sampling one really would
		like to have when doing such modeling? Also, you write about RAM-requirements scaling more
		than linearly (assuming with L). What exactly lies behind this scaling? Finally, just out of
		curiosity, are there ways of coarse-graining/renormalizing the random walk such that larger time
		and length scales can be reached? That being said, I am impressed by fit against the scaling
	  	law.}
	  	  
	  	Author reply: We can appreciate why the reviewer has questioned this limitation.
		When simulating fractional motion, each step must be correlated to all of its previous 
		steps. Therefore the computational cost increases with each added step. We show in the paper
		that the mean first passage time across the pores scales with L$^{\alpha}$, and $\alpha$ is
		always $\geq$ 2. As the degree of anti-correlation between hops increases, the scaling
		parameter also increases. Therefore, we need to generate very long trajectories, with 
		correlation between all data points, in order to sufficiently sample the MFPT distribution 
		before fitting them to Equation 9. This is especially challenging with slow-moving solutes
		like acetic acid. We needed to simulate over 2 million timesteps in order to sufficiently
		sample the MFPT over a 50 nm pore. This procedure becomes computationally expensive very
		quickly, and requires ample RAM to store the trajectories as they are constructed.
		
		Given the quality of our fits in Figure 11b, we felt it was not necessary to carry out the
		calculation for longer pores which would likely offer little extra benefit to the insight we
		already gained.
		 	
	  	This may be an issue that could be overcome by developing specialized algorithms and perhaps 
	  	by using a compiled computer language like C++. 
	  	  
	  	%BJC: Could reasonably go beyond 50 nm but 7 microns becomes tough
	  	%BJC: building in the correlation structure is computationally expensive. Reference a paper describing how FBM is simulated
	  	%BJC: could benefit from optimized libraries, but python isn't fast enough
	  	%BJC: also need to specify the number of steps to take beforehand
	  	%BJC: truncated correlation functions may allow us to surpass the limitation

  \item \textit{In this paper the models are only evaluated by their ability of achieving 
		  self-consistency with the trajectories used as input. This is conceptually not ideal. 
		  Are there no experimental data or mathematical models which act as a reference for 
		  the models? Please elaborate.}
		  
		  Author reply: The reviewer makes a valid point. This work started in parallel with experimental
		  work which promised solute flux and selectivities that we could compare directly to our
		  simulations. However, synthesizing an aligned hexagonal phase membrane has been a practical
		  challenge for our experimental collaborators and much of their focus has shifted to 
		  membranes formed by liquid crystals that self-assemble into bicontinuous cubic architectures
		  which are much simpler to synthesize. We have also started to shift our focus towards models 
		  of the bicontinuous cubic phase. However our research into generated physically accurate 
		  molecular models of that phase lags our work on time series modeling. In future work, we will
		  be able to apply this technique to systems for which we have experimental data.
		  
		  %BJC: I'm not sure this is worth bringing up
		  As far as mathematical models, there are none that we know of to which we can reasonably
		  compare. Some work has been done to apply Donnan steric pore models to experimental data, 
		  but these models are continuum-based and have not been used in a predictive capacity.
		  
  \end{enumerate}

\noindent Optional comment to consider:\\
\begin{addmargin}[5.8em]{0em}
	\textit{Having read the paper thoroughly twice, I still struggle with the acronyms of the different random
	walks. Given that this is not a journal of stochastic processes, it is my opinion that the readability
	and accessibility of the manuscript would be greatly improved by writing the full names of the
	random walks in the text while keeping acronyms in figures. \\}

    %BJC: I'll address this once I incorporate things into the text.
	
\end{addmargin}

\noindent Small typos to consider: \\
\begin{addmargin}[3.3em]{0em}
\begin{itemize}
  \item \textit{Micro in micrometer should be written with upright Greek letter.} \\ \\
  		
  \item \textit{d as “dt” in integrals should be non italic.} \\ \\
  		
  \item \textit{Page 10 line 28: T should be in symbol form.} \\ \\
  		
  \item \textit{Symbol T is used for both time and transition matrix, perhaps use some matrix notation.} \\ \\
  		
  \item \textit{The references should be checked thoroughly by the authors. Ref 13, which is arguably
  the most important reference for this paper is missing the volume number.} \\ \\
\end{itemize}

\end{addmargin}

%BJC: TODO
Author reply: We thank the reviewer for catching these errors and have corrected all of the instances
we could identify.

\section*{Response to Reviewer 2}

\begin{enumerate}[label={Comment \theenumi :}, leftmargin=3.9\parindent]  

    \item \textit{ Abstract: here it is difficult to understand what is meant with "dwell times, hop 
    lengths between dwells and correlation between hops", without reading the whole Introduction (and 
    possibly reference 13, which by the way is not completely listed in the References section). The 
    authors should briefly describe the complex transport behavior observed in their previous study of
    these systems (for example, by including information from the Introduction, i.e., the last paragraph
    of page 3 and first paragraph of page 4).}
    
    Author reply: We thank the reviewer for pushing us to increase the clarity of the abstract. 
    
    Original text:
    
    \begin{quote}
    Mathematically modeling complex transport phenomena can be a powerful tool
    for extracting important physical information from molecular simulations. 
    In this study, we present two new approaches that use stochastic time series
    modeling to predict long time-scale behavior and macroscopic properties 
    from molecular simulation which can be generalized to other molecular systems
    where complex diffusion occurs. Specifically, we parameterize our models using
    long molecular dynamics (MD) simulation trajectories of a cross-linked 
    H\textsubscript{II} phase lyotropic liquid crystal (LLC) membrane in order to
    predict solute mean squared displacements (MSDs), solute flux, and
    solute selectivity in macroscopic length pores.    
    \end{quote}
    
	Modified text (highlighted in red):
	
	\begin{quote}
    Mathematically modeling complex transport phenomena can be a powerful tool
    for extracting important physical information from molecular simulations. 
    In this study, we present two new approaches that use stochastic time series
    modeling to predict long time-scale behavior and macroscopic properties 
    from molecular simulation which can be generalized to other molecular systems
    where complex diffusion occurs. \textcolor{red}{In our previous work, we studied long 
    molecular dynamics (MD) simulation trajectories of a cross-linked 
    H\textsubscript{II} phase lyotropic liquid crystal (LLC) membrane where we
    observed subdiffusive solute transport behavior characterized by intermittent
    hops separated by periods of entrapment. In this work, we use our models to 
    parameterize the behavior of the same systems, so we can generate characteristic
    trajectory realizations that can be used} to predict solute mean squared 
    displacements (MSDs), solute flux, and solute selectivity in macroscopic length pores.
    \end{quote}
  
    \item \textit{Some of the acronyms were not defined (or perhaps I missed them), for example 'AD' 
    (page 7) and 'MLE' (page 11)}
    
    Author reply:

    \item \textit{In the Introduction, specifically the last paragraph of page 3 and first paragraph
    of page 4, the authors should refer to Figure 1 to complement their description of the complex solute
    dynamics.}
    
    Author reply:

    \item \textit{Caption of Figure 7 should refer the reader to Table 2 for completeness}
    
    Author reply:

\end{enumerate}

%\bibliographystyle{ieeetr}
%\bibliography{transport}

\end{document}

