\documentclass{article}
\usepackage[margin=1in]{geometry}
\usepackage{amsmath}
\usepackage{xr}
\usepackage{xcolor}
\usepackage{siunitx}
\usepackage{gensymb}
\usepackage{graphicx}
\usepackage{wrapfig}
\usepackage{subcaption}
\usepackage{enumitem}
\usepackage{scrextend}
\usepackage[normalem]{ulem}
\usepackage{soul}

\externaldocument[S-]{Supporting_Information}
\externaldocument[M-]{Draft}

\begin{document}

\graphicspath{{./figures/}}

\begin{center}
\textbf{Response to reviewers: Statistical Inference of Transport Mechanisms and Long
        Time Scale Behavior from Time Series of Solute Trajectories in Nanostructured Membranes} \\
Authors: Benjamin J. Coscia, Christopher P. Calderon and Michael R. Shirts
\end{center}

We thank the reviewers for carefully reading over our manuscript and providing
thoughtful and constructive feedback. We have taken the suggestions into consideration 
and made appropriate revisions to the manuscript document. The comments have been reproduced
in italics and all changes to the text have been documented below them. Any newly added
references are given in the bibliography of this document and are placed in the main 
text with appropriate numbering.

\section*{Response to Reviewer 1}

\begin{enumerate}[label={Comment \theenumi :}, leftmargin=3.9\parindent]

    %r1_comment1
    \item \textit{The manuscript by Coscia et al. analyzes diffusion and transport of solute molecules
    in a low water content lyotropic liquid crystal using a statistical model developed on the basis 
    of molecular dynamics simulations. The purpose of the statistical model is two-fold; first to 
    extrapolate computed MSDs to longer times to obtain diffusion coefficients, and second to attempt
    to gain increased physical insight on diffusion mechanisms. These issues are important in 
    highly-confined/low-water content nanoporous materials, in which diffusion occurs through complex
    mechanisms and over long timescales. The manuscript will most likely be of interest to the community
    interested in nanoporous materials for separations.}
    
    Author reply: We thank the reviewer for their thoughtful assessment of our work.
    
    %r1_comment2
    \item \textit {Unfortunately, because there is no comparison with alternative techniques, it is 
    very difficult to evaluate this new approach relative to previous work that has examined similar 
    issues. There exist numerous other methods for extrapolating MSDs to longer time, which the 
    authors have not discussed. One such method is the “trajectory-extending kinetic Monte Carlo” 
    (TEKMC) approach of Brown and coworkers see e.g. Macromolecules 2010, 43, 9210-9214, 
    Macromolecules 2015, 48, 7346-7358, and J. Phys Chem. B, 2012, 116, 95-103. The authors should 
    add a comprehensive discussion of how their approach compares to this and similar methods. The 
    paper would be significantly improved if the authors were to perform and present TEKMC results 
    for their system, so that one could better evaluate the performance of their proposed method.}
    
    Author reply: We thank the reviewer for bringing our attention to these other interesting 
    approaches. They certainly merit discussion as alternative approaches for estimating mean 
    squared displacements on long timescales. We have added some discussion of these methods to 
    the introduction.
    
    \begin{quote}
    A number of approaches have been explored to inexpensively extend the relatively short
    trajectories output by molecular simulations which exhibit complex diffusive behaviors
    onto much longer timescales. The trajectory extending kinetic Monte Carlo (TEKMC) approach
    divides a unit cell into a three dimensional grid and quantifies the transition 
    probabilities of particles that can move between each subdivision allowing one to generate
    stochastic trajectories that reach the diffusive regime~\cite{neyertz_trajectory-extending_2010}
    It is possible to parameterize TEKMC models with relatively short simulation trajectories 
    which facilitates systematic studies of small perturbations to material constituents.~\cite{hanson_computer_2012}
    In a related approach, dynamic bond percolation (DBP) theory can be used to quantify the
    rate of transition between sites in a dynamically disordered medium and long timescale
    behavior can be propagated using kinetic Monte Carlo simulations.~\cite{druger_dynamic_1983}
    Webb et al. extended this approach by defining sites based on the molecular details of 
    a polymer electrolyte system, thus allowing them to explain ion diffusivities predicted by
    the model in terms of chemically specific interactions.~\cite{webb_chemically_2015}
    \end{quote} 
    
    We have also modified the first sentence of the paragraph that follows in order to maintain
    good flow.
    
    Original text:
    \begin{quote}
    We have already advanced our understanding of LLC monomer design in our previous work
    by building stochastic time series models based on the chemical intuition gained
    from qualitative MD studies of solute transport in an H\textsubscript{II} phase LLC
    membrane.
    \end{quote}
    
    Modified text:
    \begin{quote}
    Based on the chemical intuition gained from our previous qualitative MD studies of 
    solute transport in an H\textsubscript{II} phase LLC membrane, we have contributed our
    own stochastic modeling approaches with goals similar to both TEKMC and DBP.
    \end{quote}

    %MRS: tweaked below.
    We agree that a full comparison of our method to other methods for computing long timescale 
    diffusion such as TEKMC would be very interesting and instructive but feel it would be better 
    suited as a separate study where we could include a number of approaches, including perhaps
    recent approaches we have developed within our research group.~\cite{coscia_capturing_2020}
    Such a large-scale comparison would take more space and time than would be appropriate for 
    the current paper. Although method such as TEKMC seems a fairly straightforward approach to
    implement, it is not immediately clear how we could optimally adapt the methodology to a 
    monoclinic unit cell with two dimensional order. 
%    the grids that divide the homogeneous cubic unit cells of the referenced TEKMC studies to 
%    our system which uses a monoclinic unit cell and has 2 dimensional order. 
    
    As stated by the reviewer, there are two goals of our work: to obtain long timescale MSDs and
    to understand mechanisms leading to diffusive behavior. TEKMC may be a suitable alternative 
    approach to obtaining our first goal, but it is not clear that it could give the same level
    of mechanistic insight that the HDP-AR-HMM offers. Although the work by Hanson et al.
    (J. Phys Chem. B, 2012, 116, 95-103) is able to provide physical insight into the effect of
    nanoparticles on diffusion of gaseous penetrants in a polystyrene polymer matrix, it does so
    by testing hypotheses motivated by observations of the diffusion constant (predicted by TEKMC)
    in response to different temperatures and nanoparticle concentrations. This is actually pretty
    in line with the type of analysis we performed in our initial LLC transport work (J. Phys Chem. B, 2019, 123, 6314-6330)
    where we used the mean squared displacement and the qualitative characteristics of the solute
    time series in order to hypothesize transport mechanisms. In this work we "re-discover" these
    mechanisms with minimal human intervention and characterize them in detail using the 
    HDP-AR-HMM while \textit{simultaneously} providing a method to estimate diffusion constants.
    
    %r1_comment3
    %BJC: not sure if I should ignore the JCTC part completely or basically just say what the editor said.
    %MRS: my response below (Basically, saying what the editor said, in case it is send back out).
    \item \textit{Much of the writing seems more suitable for JCTC rather than JPC. I encourage the
    authors to better emphasize and discuss “new physical insight” and “chemical details”, and spend
    less time on methodology. For example pages 5-16 are all on methodology, yet it is not until 
    Figure 8 of the paper (page 26) that the authors give any description of what the LLC system 
    they are studying is actually composed of. Presumably the LLC is made up of self-assembled 
    monomers (in water) shown in Figure 8a). I would encourage the authors to describe their 
    system more concretely much earlier in the paper... there is also no information on system 
    size, box dimensions, etc.}
    
    Author reply: We thank the reviewer for encouraging us to include more physical insight and
    further details about our molecular simulations. We received feedback from the editor along 
    with these comments that the scope of JPC has been expanding, and that the methodological 
    details are appropriate. We have studied the structure of and transport of solutes within 
    this particular LLC system extensively in our previous work.~\cite{coscia_chemically_2019}
    We feel strongly that our current work should emphasize the HDP-AR-HMM as a more general 
    tool for understanding complex dynamics in systems beyond our example system. We would prefer
    not to cut any methodological details, as this would hurt the paper's reproducibility, and
    lengthening the paper by including additional physical insight may narrow the audience willing
    to dive in.

    %TODO    
    %BJC: probably could add a picture of the system. Or is it okay to just reference another
    % paper? I would probably use the same pic anyways or something very similar.
    %MRS: never bad to include pictures of the system, so go ahead.
    
    We do agree that there are certain basic simulation details that we have neglected to 
    mention. Therefore we have made the following edits to section 2.1 of the methods:
    
    We added the highlighted text to the first sentence of the section:
    \begin{quote}
      We studied transport of solutes in the H\textsubscript{II} phase using \hl{our
      previously developed} atomistic molecular model of four pores in a 
      \hl{$9.4\times9.4\times7.6$ nm} monoclinic unit cell with 10\% water by weight, 
      \hl{totaling 62,000 atoms}.    
    \end{quote}
    
    We have also added the following text describing the model:
    \begin{quote}
    We modeled water molecules using the TIP3P water model,~\cite{jorgensen_comparison_1983}
    We used the Antechamber module shipped with AmberTools18 to parameterize monomers and 
    solutes using the Generalized Amber Force Field (GAFF).~\cite{wang_development_2004} We 
    assigned charges using the AM1-BCC model as implemented in Antechamber.~\cite{jakalian_fast_2000,jakalian_fast_2002}
    GAFF is very commonly used in simulations with diverse organic chemical functionality, 
    and was developed for consistency within the AMBER ecosystem, which uses TIP3P as the
    water model. We chose these standard force fields and generalizable parameterization 
    approaches so that a range of solutes and monomers could be self-consistently and easily 
    studied. The main goal of this work is to develop stochastic models which can reproduce
    solute behavior on MD time scales, independent of the choice of force field. In future 
    work, it may be beneficial to employ specialized force fields, including polarizable 
    force fields or other higher-order effects in order to improve the accuracy of structure 
    parameters and selectivity predictions.
    \end{quote}

    %r1_comment4
    \item \textit{The authors should more clearly note that their discussion largely applies only 
    to very low water content LLC regimes (such as the 10\% LLC studied). For higher water content 
    LLCs (e.g. $>$ 15-20\%) it is straightforward to obtain the diffusion limit from standard MD 
    simulations. While the authors are surely well aware of this, the reader may not be, and it is
    important to clearly distinguish the regimes.}
    
    Author reply: We thank the reviewer for making this point. 
    %BJC2: modded based on MRS suggestion    
    In systems with less obstructions to motion, such as LLCs with a higher water content, most 
    solutes and water will be relatively unimpeded, and diffusion constants can be obtained by 
    standard techniques.    
    However, we do contend that our method would still be very useful for studying
    solute dynamics in these higher water content systems 
    %BJC2: incorporated MRS suggestion with small modifications    
    by allowing one to study the dynamics of solutes that have longer residence times and interact
    with moieties in a complex medium. %portions of the system away from the pore center.
    %Interactions characterized by different dynamical behavior even on short time scales can
    %still be highly informative. 
    This is the basis for the point we make in the introduction:``...direct calculations of diffusion 
    constants may obscure the molecular mechanisms controlling transport". Therefore we would
    prefer not to limit the range of application of this method. 
    %BJC2: added    
    However, we do agree that it would be helpful to the reader to clarify that the level of modeling 
    which we present in this work is certainly not merited in cases where they are only interested in
    diffusion constants. Therefore, we have modified a paragraph in the introduction in order to
    encourage readers to pursue simpler approaches if the system allows it.
    
    Updated text with additions highlighted in yellow:
    
    In this work, we are particularly interested in the design of lyotropic liquid 
    crystal (LLC) monomers, a class of amphiphilic molecules whose ordered self-assembled
    phases can be cross-linked into mechanically strong membranes capable of highly 
    selective separations. Inverted hexagonal phase, or H\textsubscript{II}, LLC membranes
    are characterized by hexagonally packed, uniform-sized and straight pores, an ideal 
    geometry for high throughput transport. Their pores are lined with the LLC monomer's
    functional groups which can potentially be designed to interact with solutes in a 
    chemically-specific manner. It would be extremely useful to experimentalists if we 
    could connect LLC monomer design with measurable quantities such as diffusivity and
    selectivity.\hl{In many systems, including LLC membranes with water contents higher
    than the 10 wt\% water system studied here, making this connection can be relatively 
    straightforward calculation with standard techniques since small solutes and water 
    can move relatively unimpeded and therefore enter a diffusive regime on timescales 
    accessible to MD. When possible, we encourage the reader to take advantage of the 
    simplest approach possible. However, for the system that has capture our interest,} 
    this goal is complicated by the non-Brownian hopping and trapping behavior induced 
    by the membrane's diverse topology and inhomogeneous structure resulting in subdiffusive
    behavior out to multiple microseconds of simulation.~\cite{coscia_understanding_2019,coscia_chemically_2019}
        
    %r1_comment5
    \item \textit{In the HDP-AR-HMM method, it is not clear to me how the solute molecules are 
    constrained to be confined within the pore radius (which they must be on physical grounds 
    based on steric interactions). From Figure 2, it “looks” like this happens to work out, but
    for more complicated systems (e.g. gyroid LLCs with 3D channels) such a constraint may be 
    necessary and it is unclear how the method would generalize. This should be discussed.}
    
    Author reply: We thank the reviewer for posing this question as this a constraint to which 
    we gave thought but neglected to emphasize in the main text. 
    %BJC: he's not saying that solutes should stay in the aqueous pore region right? We talk pretty
    % extensively about solute behavior in the tails so his steric argument is confusing me.  

    %MRS: It is a little confusing. First, we never use the term "pore
    %radius", since the structure of the pore makes it very hard to
    %define a single pore radius.  The term "pore region" is used once
    %(about line 995), that that should probably be "pore center".
    %BJC2: changed pore region to pore center
    %Second, it's not clear what is meant when they say it's not clear
    %how they are constrained when they must be by steric
    %interactions.  Since steric interactions do NOT keep them in the
    %pore region (at least for the smallest solutes).  Figure 2 is a selection of a trajectory. 

    %    First, we would like to clarify
%    that our simulations suggest that small solutes can partition out of the pores and into
%    the dense tail region. This is not a particularly favorable move as evidenced by our radial
%    distribution functions, but it does happen and could conceivably lead to rare jumps of solutes
%    between pores (although we have not observed any).

    %MRS: how about this?  
    %BJC2: seems good. made some minor edits
    We are grateful for the opportunity to clarify some of the details of the simulation 
    geometries. We avoid defining a ``pore radius'' because the radial distribution of 
    head groups and even tail groups is spread over several nanometers and we have
    observed clear evidence that solutes can similarly partition radially throughout 
    the pores.~\cite{coscia_understanding_2019,coscia_chemically_2019} 
%    This partitioning occurs to the extent that smaller molecules like water and methanol
%    can in fact move between pores in rare instances.
    %BJC2: I don't think I've actually published anything stating or proving the above. 
    %I think it's particularly hard to get by the crosslinks. But I do believe it could happen.
    %rephrased with that in mind.
    Therefore, it may be possible that, in rare instances, smaller molecules like water and
    methanol may move between pores. However the rarity of these events and the fact that 
    they occur orthogonal to the transport dimension means that they do not significantly 
    affect transport, and can be interpreted mathematically as simply going deep into the 
    dense tail region before returning to the same pore.
    
    Because of this, our stochastic simulations are constrained to remain close to a single 
    pore by the trajectory generation procedure. Each state is parameterized by a mean in 
    the radial direction computed based on the solute's distance from the \textit{nearest 
    pore center}, making it impossible for the solute to get further than approximately half
    the pore-to-pore distance away (the exact maximum distance is an an angle-dependent 
    quantity due to the hexagonally packed pore configuration).
    %MRS: though I'm not QUITE sure that's right, since the hexagonal packing means that the transition between pores
    %occurs at different distances depending on the angle. 
    %BJC: Yeah, I know what you mean. I added the bit in parentheses above to acknowledge that.
    Each time a state change occurs, the mean position of the solute in the $r$ dimension 
    jumps to the appropriate mean. However, in the axial direction, we do not parameterize 
    the mean since it should be unbounded in that dimension. The $z$ mean of each segment 
    is set based on the position immediately before the state change occurred.
    
    The generalization of this procedure to three dimensions will be under active investigation 
    in the near future. As mentioned, we are interested in applying this approach to bicontinuous
    cubic phases like the gyroid. Based on our current understanding, we would model solute dynamics
    in three dimensions and set the mean vector, $\mathbf{c}$ equal to $\mathbf{0}$. Of course, this
    may result in solutes exploring a three dimensional space that does not share the geometric 
    features of the studied phase. Our hypothesis is that we can add one or more additional 
    dimensions to the time series passed to the HDP-AR-HMM that includes a distance from the pore centers. 
    We could define the pore center using a spline or a surface and transitions would likely
    favor moves towards states that exist closer to the pore interiors. Alternatively, it may prove
    necessary to constrain solute positions to the means in each dimension. However, these extensions
    are beyond the scope of the current paper.
    
    %BJC: figured I'd get your feedback on above before writing the part that will go into the text.
    
    %r1_comment6
    \item \textit{As stated in the introduction, one motivation for the work is that “...one 
    cannot reliably compute diffusion constants because MSDs are often non-linear on the timescales
    accessible to molecular simulation”. Later on in section 2.4, the authors state “For the 
    purposes of our analysis, we assume ergodicity of the 24 solute trajectories.” These are 
    contradictory statements. If the trajectories are ergodic, then MSDs will be linear by 
    definition The authors should clarify. Similarly, in Figure 7, if the simulations are indeed 
    ergodic the red and blue bars should be proportional to each other.}
    
    Author reply: We thank the reviewer for identifying the potential confusion that could be
    caused by our use of the word 'ergodic'. Figure 7 clearly demonstrates that individual trajectories
    are not by themselves ergodic. Our intention was to communicate our assumption that collectively all 24 
    trajectories visit all possible dynamic states with representative probability. This is 
    necessary in order to make accurate long timescale predictions. The 5 $\mu s$ length of our 
    simulations was intended to increase the validity of this assumption. 
    
    The use of the word `ergodic' is not necessary, therefore we have modified the text as follows
    to remove the source of potential miscommunication:
    
    %r1_comment7
    \begin{quote}
    For the purposes of our analysis, \sout{we assume ergodicity of the 24 solute trajectories. That is,}
    we assume our MD simulations sample all possible states with the correct frequency.
    \end{quote}    
    
    \item \textit{There are no units on Figure 1 y-axis, presumably these are nanometer like the 
    other figures?}
    
    Author reply: We thank the reviewer for catching our oversight. We double-checked the 
    other plots in the paper and found the same mistake on the y-axes of Figure 5. Units of nanometers
    have been added to the y-axes of both plots.
    
    %r1_comment8
    \item \textit{Figure captions need to be improved. Wording such as “we plotted...”, “we 
    calculated...”, “We also analyzed...” should be omitted in figure captions.}  
    % BJC: This seems like the reviewer forcing his own style on us. And seems like a generalization
    % based on only looking at figure 8. I could just modify parts of figure 8 to please him. But I
    % think the rest are fine. What do you think?

    %MRS: So, I think they have a partial point, in that the caption
    %should describe the figure, and the conclusions proved by the
    %figure, not the process used to get to the figure so much.  So maybe figure 8 should be "hydrogen bonds donated to the methanol to the highlighted oxygen atoms in three regions of the monomer are are shown: the carboxylate . . For each type of hydrogen bond in each predominant state, we show the 95th percentile of hydrogen bond lifetimes."  Does that make sense?

\end{enumerate}


\section*{Response to Reviewer 2}

\begin{enumerate}[label={Comment \theenumi :}, leftmargin=3.9\parindent]  

    %r2_comment1
    \item \textit{This paper presents an approach to study transport mechanisms of solutes within
    membranes by applying the sticky hierarchical Dirichlet process autoregressive hidden Markov model
    (HDP-AR-HMM) to molecular dynamics simulation trajectories spanning long times (5 microseconds). 
    Using this method, the authors reveal different dynamic states associated with the transport of 
    solute across a membrane, and interpret them on the basis of solute-membrane interactions. The 
    paper is clear, well written, and relevant to the Journal of Physical Chemistry. I only have a 
    couple of suggestions and minor corrections:}
    
    Author reply: We thank the reviewer for their kind assessment of our work.

    %r2_comment2
    \item \textit{A motivation for the study is providing a computational tool to guide experimentalists
    in designing membrane materials. However, given the length of the molecular dynamics simulations 
    required, it is presumably impractical to carry a large-scale screening in this way. It would be 
    useful if the authors were more explicit about the role of this approach in guiding experiments, 
    and what would be its advantages and limitations compared to a purely experimental search.}
    
    Author reply: We thank the reviewer for pushing us to further evaluate the practicality of our
    method as a tool for experimentalists. In short, there are many ways we can envision that 
    would make this approach more amenable to high throughput simulations. The length of our 
    simulations in this paper are indeed quite long. The dynamics of the solutes are somewhat 
    slow and we wanted to ensure we gathered enough data for a sufficient parameterization. 
    Putting aside brute force high throughput simulations that could be enabled by the 
    ever-increasing availability of HPC resources, it would be to our benefit to use the system
    in this work as a benchmark to explore much less expensive models.
    %MRS: maybe focus more on understanding mechanisms, rather than screening?  So the length is not SO impractical.
    We propose some ideas to this end in the text we have added to the end of Section 2.1, where
    our simulations are described:
    \begin{quote}
    In consideration of our ultimate goal, which is to use high-throughput simulations to 
    facilitate our understanding of LLC membrane design for solute-specific separations, the size 
    of our system and length of the accompanying molecular simulations appears to be somewhat 
    impractical.
    We would like to emphasize that our large, fully atomistic models will be
    extremely valuable as a benchmark systems used to explore less expensive models. For example, we 
    could explore systems built with fewer LLC monomers by making the membrane pores shorter. We 
    could also build systems with higher concentrations of solutes in order to gather more 
    dynamical data on shorter timescales. Using what we have learned about the radial distribution
    functions of these solutes in this and our previous work~\cite{coscia_chemically_2019}, it may be possible to 
    learn how to create initial configurations with solutes initially distributed in a way that
    is close to their equilibrium distribution, allowing solutes to thoroughly explore the 
    accessible membrane structural space more quickly and reducing the amount of unequilibrated
    data that must be discarded. % could mention GCMC?
    %MRS: sure.
    Finally, it may be possible to speed up our simulations by 
    reducing the number of atoms using united-atom representations of the liquid crystal monomers.
    %BJC: How strong of an influence would hydrogen mass repartitioning have on hydrogen bonding?
    %, or by using hydrogen mass repartitioning to allow us to increase the MD time step. 
    %MRS: zero on hydrogen bonding strength, which is thermodynamics.  It may affect H-bonding kinetics by
    % increasing the timescales of tumbling (larger rotational inertia, slower rotational diffusion).  If molecules aren't free to tumble anyway, though,
    % then the kinetics may be affected less. 
    \end{quote}
    
    %r2_comment3
    \item \textit{Have the authors attempted to compare their results directly to experimental data? 
    It would be useful to clarify whether the predicted selectivities are expected to be quantitatively
    accurate or just reflect trends.}
		 
    %BJC: This is copied straight from the reviewer response for the previous paper.
    Author reply: The reviewer makes a valid point. This work started in parallel with experimental
    work which promised solute flux and selectivities that we could compare directly to our
	simulations. However, synthesizing an aligned hexagonal phase membrane has been a practical
	challenge for our experimental collaborators and much of their focus has shifted to 
	membranes formed by liquid crystals that self-assemble into bicontinuous cubic architectures
	which are much simpler to synthesize. We have also started to shift our focus towards models 
	of the bicontinuous cubic phase, which do have much more experimental data. However, our 
	research into generating physically accurate molecular models of that phase lags our work on
	time series modeling. In future work, we will be able to apply this technique to these 
	bicontinuous systems for which we have experimental data.

    %r2_comment4    
    \item \textit{As two very minor corrections, in page 15 line 8 it should say "Method 1", and in 
    page 17 line 5 the angle should be H-D-A.}
    
    % BJC: reviewer is right, VMD docs are misleading: https://www.ks.uiuc.edu/Research/vmd/plugins/hbonds/
    Author reply: We thank the reviewer for pointing out our potential errors. For the hydrogen bond
    criteria we have updated the text from ``D-H-A" to ``H-D-A". Regarding page 15 line 8 (``Method 2:
    Since trajectories generated by Method 2..."), changing ``Method 2" to ``Method 1" would actually be 
    incorrect. At the beginning of Section 2.4, we defined two techniques for generating stochastic 
    trajectories. ``Method 2'' on page 15 line 8 describes the bootstrapping procedures for Method 2 of
    trajectory generation. This is a clear lack of clarity on the author's part and therefore we 
    have made an effort to modify our wording to avoid confusion.
    
    First, we added additional clarification before describing the bootstrapping procedure (added text
    is highlighted in red):
    \begin{quote}
    Our bootstrapping procedure varied dependent on the method of trajectory generation. \textcolor{red}{
    The following procedures correspond to the methods described above:}
    \end{quote}
    
    We have also modified the text on page 15 line 8 from the original text:
    \begin{quote}
    Since trajectories generated by Method 2 are based on the clustered parameter set used to define a
    single model, we can bootstrap on trajectory ensembles of arbitrary size.
    \end{quote}
    to:
    \begin{quote}
    Since in our second approach to stochastic trajectory generation described above we use clustered 
    parameter sets to define each model, we can bootstrap on trajectory ensembles of arbitrary size.
    \end{quote}

\end{enumerate}

\bibliographystyle{ieeetr}
\bibliography{hdphmm}

\end{document}

%MRS: additional thoughts:

%1. Dynamic Bond Percolation theory (both the original version and the
%chemically specific one) an and TEKMC are based on spatial clustering
%and coarse-graining. Our approach is based on clustering and
%coarse-graining the dynamics.  Can be more powerful, doesn't assume
%(or minimally assumes - in some cases we assume pore vs. nono-pore)
%which configurations are important, which matters when there are a
%diversity of different types of bound states, and diversity of
%dynamics (in the pore, associated with the head groups, etc).

%2. Dynamical behavior is more complex with our solutes than ion
%interactions because of the richer range of interactions involved
%(ionic binding PLUS hydrophobic binding, i.e. more types of bound
%states; it's not just binding and hopping, and thus something

%3. Configurational behavior is more complex with the nanostructures,
%because one cannot simply divide the structure into cubes
%isotropically.  It may be easier to use our approach with more
%diverse configurations, because we don't need to worry about
%partitioning correctly.  This does push some of the difficulty into
%choosing the coordinate system for the dynamics, but we have a lot
%more freedom about choosing coordinate systems for the model than one
%would using a uniform (or even nonuniform) partitioning of the
%system. Yes, there are difficult choices to make but the model has
%the freedom to make these choices.

%4. One reason our simulations are longer because the timescales of
%events observed generally are longer, and the number of solutes we
%can put into the simulation is smaller (since the total pore volume
%is smaller than it would be in an amorphous polymer).

%5. We could probably construct our model it with a larger number of
%more shorter simulations as well. Rare events require long timescale
%sisulations.  However, the equilibration times for these simulations
%(since we build from geometry, not from amorphous melts) are such
%that we probably get more efficient use of simulations by running
%fewer, longer simulations.


%6. So it's not clear that this method is better than DBP or TEKMC
% (especially since the results are a big ambiguous); one could define
% bound states, and look at hops betwen then.  But it's definitely
% DIFFERENT, it's not really the same thing at all, and we should push
% back that it is similar.  We can argue that our approach is more
% suited to our systems than the two configuration-based approaches -
% one COULD try to use those approaches and they could be made to
% work, but it's not a trivial thing to do, and we wanted to try
% something where the data informed our model more completely and we
% wouldn't have to try those modifications.


% LocalWords:  Subdiffusive Solute Nanoscale Coscia leftmargin FBM vy
% LocalWords:  subdiffusion CTRW sFBM ref solutes antechamber solute
% LocalWords:  polarizable LLC intramolecular quasiharmonic BJC JCTC
% LocalWords:  tex fontsize evy Methanol's selectivities MSDDM arXiv
% LocalWords:  parametrization MSDs MSD timescales renormalizing MFPT
% LocalWords:  bicontinuous Donnan steric em dt lyotropic Calderon al
% LocalWords:  subdiffusive Nanostructured nanoporous TEKMC Phys Chem
% LocalWords:  Venkat Ganesan DBP timescale Webb diffusivities HDP AR
% LocalWords:  HMM nanoparticles penetrants nanoparticle JPC pic LLCs
% LocalWords:  diffusivity gyroid solute's nanometer nanometers HPC
% LocalWords:  carboxylate Dirichlet unequilibrated GCMC VMD docs
