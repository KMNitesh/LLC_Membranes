\documentclass{article}
\usepackage{graphicx}
\usepackage{wrapfig}
\usepackage{subcaption}
\usepackage[margin=1in]{geometry}
\usepackage{amsmath} % or simply amstext
\usepackage{amssymb}
\usepackage{siunitx}
\usepackage{booktabs}
\usepackage[export]{adjustbox}
\newcommand{\angstrom}{\textup{\AA}}
\newcommand{\colormap}{jet}  % colorbar to use
\usepackage{cleveref}
\usepackage{booktabs}
\usepackage{gensymb}
\usepackage{float}

%MRS: better to have the first outline shorter to iterate more quickly!
%BJC1: right, this one has been in the works for a while now. Wrote the anomalous diffusion methods a long time ago

%MRS: better to have a title that described a finding, rather than the
%approach used.  But we may need to wait a bit further until we can decide on a finding.
%MRS: also 'stochastic modeling' is pretty vague.  should be more precise. 
\title{Using Stochastic Modeling to Predict Long Timescale Transport Behavior of Solutes in an H\textsubscript{II} Phase Lyotropic Liquid Crystal Membrane}
%MRS: we should discuss whether Chris Calderon would be on this one.  I would guess no, but I don't know for sure. 
%BJC1: Yea maybe Chris as an acknowledgement on this one. He didn't directly contribute anything other than pointing
%me towards Greg Morrin who pointed me towards useful anomalous diffusion literature.
\author{Benjamin J. Coscia \and Michael R. Shirts} 

\begin{document}

  \graphicspath{{./figures/}}
  \maketitle

  \section{Introduction}
  %BJC: obviously this section will get beefed up
  We need highly selective membranes in order to perform efficient separations. \\

  \noindent Amphiphilic molecules are capable of self-assembling into ordered nanostructures.\\

  Lyotropic liquid crystals are a class of amphiphilic molecules that can be cross-linked
  into mechanically strong membranes.
  \begin{itemize}
  	\item H\textsubscript{II} phase lyotropic liquid crystals have densely packed, uniform
	sized pores and have the potential to disrupt conventional membrane separation
	techniques by being selective based not only on size and charge, but on chemical
	functionality as well.
	\item Q\textsubscript{I} phase LLCs consist of a tortuous network of 3D interconnected
	pores. They are easier to make.
  \end{itemize}

  We can only learn so much from experiment. MD can give us mechanistic insights with
  atomistic resolution so that we can intelligently design new membranes for 
  solute-specific separations.\\

  \noindent In our previous work, we studied the transport of 20 small polar molecules
  in an H\textsubscript{II} phase LLC membrane.
  \begin{itemize}
    \item In general, we observed subdiffusive transport behavior characterized by 
    intermittent hops separated by periods of entrapment.
    \item We identified three mechanisms responsible for the solute trapping behavior:
    entanglement among monomer tails, hydrogen bonding with monomer head groups, and
    association with sodium counter ions.
  \end{itemize}

  Unfortunately, the timescales that we can simulate with MD are insufficient to be
  able to make well-converged predictions of macroscopic transport properties 
  traditionally used to characterize membranes in the lab.
  \begin{itemize}
    \item However, if we use descriptive stochastic models that can capture solute
    dynamics, then we could project long timescale behavior in addition to gaining
    a deeper understanding of solute behavior on short timescales.
  \end{itemize}
  
  In our previous work, we designed two different approaches which used
  solute time series in order to parameterize stochastic models that could be used
  to project transport on much longer timescales.
  \begin{itemize}
  	\item Brief description of anomalous diffusion
  	\item Brief descritpion of MSDDM
  \end{itemize}
  
  Although both models had reasonable success at predicting solute MSDs on simulation
  timescales, they had shortcomings.
  \begin{itemize}
  	\item Why MSDDM failed
  	\item Why anomalous diffusion model could be better.
  \end{itemize}
  
  % BJC: Not sure whether to call it the infinite hidden markov model or hierarchical dirichlet process hidden markov model
  % The former is definitely simpler.
  In this work, we apply the infinite hidden markov model, a modeling that is agnostic
  to the source of time series data, in order to detect and parameter an unknown number
  of autoregressive modes.
  
    
  \section{Methods}
    
  We ran all MD simulations and energy minimizations using GROMACS 2018. We  % needs gromacs citations
  performed all post-simulation trajectory using python scripts which are available
  %MRS: eventually want to say what scripts were used to do what, but that is not at the outline stage!
  %BJC1: I usually put a table like that in the SI. Is that sufficient?
  online at \\ \texttt{https://github.com/shirtsgroup/LLC\_Membranes}.

  \subsection{Molecular Dynamics Simulations}

  We studied transport of solutes in the H\textsubscript{II} phase using an
  atomistic molecular model of four pores in a monoclinic unit cell with 
  10 \% water by weight. 
  \begin{itemize}
    \item Approximately one third of the water molecules occupy the tail region 
    with the rest near the pore center.
    \item We chose to study the 10 wt \% water system because solutes move 
    significantly faster than in the 5 wt \% system studied previously.
    \item Appropriate stochastic modeling requires that solutes sample the 
    accessible mechanisms with representative probability.  %BJC1: incorporated suggested phrasing
  \end{itemize}
  
  We chose to study a subset of 4 of the fastest moving solutes from our previous
  work: methanol, acetic acid, urea and ethylene glycol.
  \begin{itemize} 
    \item In addition to exploring membrane structural space the most, these solutes
    have a relatively diverse set of chemical functionality.   
    %MRS: do the solutes also have some diversity of mechanisms (that were observed previously?)
    %BJC1: Yes. Need to put following into words
    %BJC1: Ranking of time spent in pore region: AcOH > EG > URE > MeOH  (spans 4th most to 3rd least)
    %BJC1: All have high hop frequency which is consistent with fast motion
    %BJC1: Hbonding: Urea barely hbonds. AcOH and EG hbond a lot. AcOH has highest hbond lifetime of all 20 solutes. MeOH on par with all other simple alcohols
    %BJC1: Sodium Association: Urea coordinates with sodium the most of all 20 solutes. AcOH, EG, MeOH spread out of rest of range. Lifetime similar 
    \item For each solute we created a separate system and to each system we
    added 6 solutes per pore for a total of 24 solutes.
    \item This number of solutes per pore provides a balance of a low 
    degree of interaction between solutes and sufficient amount of data from
    which to generate statistics on the time scales which we simulate.
    \item Further details on the setup and equilibration of these systems can
    be found in our previous work.\cite{coscia_chemically_2019}
  \end{itemize}
  
  \noindent We extended the 1 $\mu$s simulations of our previous work to 5 $\mu$s in order
  to collect ample data.
  \begin{itemize}
    \item We simulated the system with a time step of 2 fs at a pressure of 1 bar
    and 300 K controlled by the Parinello-Rahman barostat and the v-rescale thermostat
    respectively.
    \item We recorded frames every 0.5 ns
  \end{itemize}

  % BJC: can ignore this section for now. 
  \subsection{The Infinite State Hidden Markov Model}\label{method:iHMM}  %BJC: potentially a separate paper
  
  Hidden Markov models (HMMs) are a useful and widely used technique
  for modeling sequences of observations where the probability of the next observation
  in a sequence depends only on a previous unobserved, or hidden, state.~\cite{beal_infinite_2002}
  \begin{itemize}
    \item In the context of our simulations, the observations correspond to 
    the center of mass coordinates of the solutes versus time, and the states
    correspond to the dynamical behavior which give rise to those types
    of observations.
    \item Unfortunately, standard HMMs require the number of hidden states to be known
    a priori.
    \item One can partially overcome this by testing a range of numbers of 
    hidden states and determining which is the best representation of their
    data.
  \end{itemize}
  
  The infinite-state HMM overcomes this drawback by placing a hierarchical
  Dirichlet process (HDP) prior on the transition probabilities.
  \begin{itemize}
    \item Using some base probability distribution, H, a Dirichlet process 
    (DP) generates distributions over a countably infinite number of 
    probability measures:
    \begin{equation}
      G_0 = \sum_{k=1}^{\infty} \beta_k \delta_{\theta_k} ~~ \theta_k \sim H, \beta \sim GEM(\gamma)
    \end{equation}
    where the $\theta_k$ are values drawn from the base distribution and the
    weights $\beta_k$ come from a stick-breaking process parameterized by the concentration 
    parameter $\gamma$ (equivalently referred to as GEM($\gamma$)). 
    \item The concentration parameter expresses one's confidence in H relative to the posterior 
    and is closely related to the number of data observations.
    \item Each row, $G_j$, of the transition matrix is produced by drawing from a DP specified 
    using the $\beta$ vector as a discrete base distribution and a separate concentration
    parameter, $\alpha$.
    \begin{equation}
      G_j = \sum_{k=1}^{\infty} \pi_{jk} \delta_{\theta_k} ~~ \pi_j \sim DP(\alpha, \beta)
    \end{equation}
    \item This hierarchical specification ensures that the transition probabilities in 
    each row share the same support points \{$\theta_1$, ..., $\theta_k$\}.
    \item Once the model has converged only a finite number of states will have significant
    sampling.
  \end{itemize}
   
  \noindent We describe the dynamics of each state using a vector autoregressive (VAR) model. 
  \begin{itemize}
  	\item A VAR process is characterized by a vector of observations in a time series 
  	that are dependent on $r$ previous values of the time series vector, weighted by a
  	coefficient matrix $A_i$ in addition to a white noise term $\mathbf{e}_t$:
  	\begin{equation}
  	\mathbf{y}_t = \sum_{i=1}^r A_i\mathbf{y}_{t-i} + \mathbf{e}_t~~~~\mathbf{e}_t \sim N(0, \Sigma)
  	\end{equation}
  	\item We assumed multivariate Gaussian noise and limited our analysis to an 
  	autoregressive order of $r=1$.
  	\item We used a conjugate matrix-normal inverse-Wishart prior on parameters
  	$A$ and $\Sigma$ in order to analytically draw from the posterior.
  \end{itemize}   
  
  Based on the VAR parameters and matrix of transition probabilities, we calculated
  the most likely sequence of hidden states.
  \begin{itemize}
    \item We repeated this process iteratively until we reached convergence % BJC: need to define convergence criteria
    \item Our python implementation of this process is heavily adapted from the MATLAB code
    of Fox et al.~\cite{fox_sticky_2007} 
    \item We refer the interested reader to much more extensive descriptions of 
    this process and its implementation. 
    ~\cite{beal_infinite_2002,teh_hierarchical_2006,van_gael_beam_2008,fox_nonparametric_2009,fox_bayesian_2010}
  \end{itemize}
  \break
  \section{Results and Discussion}
 
  \section{Conclusion}
  
  We have tested two different mathematical frameworks for describing solute
  motion in an H\textsubscript{II} phase LLC membrane.
  \begin{itemize}
    \item Markov state modeling with predefined states gives a nice description
    of transitions between observed states as well as the type of stochastic 
    behavior shown in each state. However, it doesn't accurately portray correlated
    time series behavior leading to overpredicted MSDs.
    \item Subordinated fractional Brownian motion has a nice theoretical foundation
    in the anomalous diffusion literature. A two mode model that describes dynamics
    based on whether a solute is in or out of the pore region leads to MSDs fairly
    consistent with MD simulated trajectories.
  \end{itemize}
 
  \section*{Supporting Information}

  Detailed explanations and expansions upon the results and procedures mentioned in
  the main text are described in the Supporting Information. This information is
  available free of charge via the Internet at http://pubs.acs.org.

  \section*{Acknowledgements}

  Molecular simulations were performed using the Extreme Science and
  Engineering Discovery Environment (XSEDE), which is supported by National
  Science Foundation grant number ACI-1548562. Specifically, it used the Bridges
  system, which is supported by NSF award number ACI-1445606, at the Pittsburgh
  Supercomputing Center (PSC). This work also utilized the RMACC Summit supercomputer,
  which is supported by the National Science Foundation (awards ACI-1532235 and
  ACI-1532236), the University of Colorado Boulder, and Colorado State
  University. The Summit supercomputer is a joint effort of the University of
  Colorado Boulder and Colorado State University.

  \clearpage

  \bibliographystyle{ieeetr}
  \bibliography{stochastic_transport}

  \newpage

  \section*{TOC Graphic}

\end{document}
