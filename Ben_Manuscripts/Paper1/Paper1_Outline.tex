\documentclass{article}
\title{Predicting Transport in Lyotropic Liquid Crystal Membranes with Molecular Dynamics Simulations -- Outline}
\author{Benjamin J. Coscia \and Douglas L. Gin \and Richard D. Noble \and Michael R. Shirts} % usually the PI would be the last author.
\begin{document}
	\bibliographystyle{ieeetr}
	\maketitle
	\section{Introduction}
%MRS: make sure the introduction asks all the questions that will be answered by the results presented, and doesn't raise questions that are never brought up again. 	
%MRS: possibly we want to look at the more detailed X-ray diffraction patterns, but that may not be feasible on the timescale.
	Nanostructured membrane materials have become increasingly popular in desalination and wastewater treatment applications because they offer the ability to control membrane architecture at the atomic scale.
	\begin{itemize}
		\item Current state-of-the-art reverse osmosis membranes are dense and unstructured with tortuous and polydisperse pores which lead to inconsistent performance
		\item Tortuosity and polydispersity drive up energy requirements which strain developing regions and contribute strongly to CO2 emissions
		\item With nanostructured materials, solute rejecting pores can be tuned uniformly -- drives down energy requirements
	\end{itemize}
	
	Development of nanostructured materials has been limited by the ability to synthesize and scale various fundamentally sound technologies.
	\begin{itemize}
		\item Leading technologies and their limitations:
		\begin{itemize}
			\item Graphene sheets -atomically thick which gives excellent permeability but defects during manufacturing severely impact performance
			\item Carbon Nanotubes - MD studies are promising but synthetic techniques unable to achieve necessary alignment and pore monodispersity
		\end{itemize}
	\end{itemize} 
	Self assembling lyotropic liquid crystals (LLCs) share the characteristic ability of nanostructured membrane materials to create highly ordered structures with the benefits of low cost and synthetic techniques feasible for large scale production.
	\begin{itemize}
		\item Forms lamellar, bicontinuous cubic and hexagonal phase based on solution composition
		\item My monomer form two types of self assembled LLCs - thermotropic (Colh) and lyotropic (HII) - explain difference 
		\item Hydrophilic regions point towards pore centers
		\item Until recently, they could not be aligned - hindered progress
		\item Yale aligns them
		\item LLC HII phase membranes offer potential for high permeability and selectivity which equals low energy consumption
		\item This paper will focus on the development of a model of the colh phase since it is a simpler starting point and has just as much experimental data
	\end{itemize}
%MRS: might want to include effective pore radius.
%BJC: I'm not sure how to include the pore radius since we don't have a clear method for its measurement (experimentally and consequently
%via simulations). I don't know that we can answer that question which is why I didn't pose it in the intro	
%MRS2: if no measurement experimentally, don't need one for simulation at this point!
	Constructing a molecular model may give a clearer picture of the microscopic structure of LLC membranes which will change how we think about their design.
	\begin{itemize}
		\item The arrangement of sodium ions in the channels is thought to be confined to the pore walls. It is possible they are arranged more randomly
		\begin{itemize}
			\item This could change how one thinks about molecules diffusing through membrane
			\item Could also be a difference between lyotropic and thermotropic phases
		\end{itemize}
		%BJC - I'm likely going to cut this since I cannot answer this question definitively - unless we take an approach based on enthalpy differences and structural comparisons (pore-to-pore distance and x-ray scattering data). We probably have enough information for a speculative answer, but I don't know if that should be included in the paper
                %MRS: I think it's probably useful to report which arrangements are stable and give consistent pore-to-pore distances. Also can help answer the questions whether there are distinct well-separated ``layers'', or if it's more complicated. 
		\item How many monomers are in a layer?
		\begin{itemize}
			\item There is no clear answer to this question in literature. I'd like to confirm or deny past predictions
		\end{itemize}
	\end{itemize}
	
	A molecular level understanding of solute transport in LLC membranes will accelerate development efforts by providing guidelines for monomer design.
	\begin{itemize}
		\item Two types of self assembled LLCs have been made - thermotropic and lyotropic -- explain difference -- shift focus to Colh
		\item Colh phase studies currently limited mostly to one monomer with minor variations
		\item Current optimization efforts performed through trial and error
		\item Macroscopic models are the only source of predictive modeling
		\item An atomistic understanding of the mechanism of solute transport can identify
		performance bottle necks and direct design of future monomers/membranes
%MRS: Are we going to restrict this paper to dry membranes?  That might be the best way to have a reasonable length paper out in a reasonable amount of time. Follow up with water paper. Donnan exclusion. 
%BJC: I think restricting the paper to dry membranes (Colh phase) is a good idea, especially since I don't have much to say about the HII phase yet. 
		\begin{itemize}
%MRS: breaking into Coulombic vs. Lennard-Jones doesnt' really help that much.  We can brainstorm other ways to examine it. The cleanest is to run separate simulations with electrostatics turned off, because it's the free energy that matters, not different components of the potential energy
%BJC: Ok, I removed mention of that - We won't study electrostatic/coulombic interactions until we are looking in depth at transport mechanism anyways - right?
			\item Do ions have trouble getting through because of interactions with other things in the pores (e.g. ions, carbonyl groups, benzene rings) -- related to ionic conduction
			\item Does concentration of ions in the pore repel incoming ions?
%MRS: we might want to restrict to what happens INSIDE the ring rather than entering/exiting, since the entrance/exit will differ a lot between pores with and without water.
%BJC: Ok, removed the point about end effects
		\end{itemize}
	\end{itemize}
	A physically accurate molecular model with easily modified structural features will enable a qualitatively different type of membrane design aimed optimization of self assembled LLC membranes.
	\begin{itemize}
		\item The paper will illustrate the development of a predictive molecular model
		\item Measurements made to validate model
		\item Observations of transport mechanism
		\item The model is consistent with experiments and will be used in solvent/solute flux and solute rejection predictions
	\end{itemize}
	
	\section{System Setup}
	
	HII monomers were parameterized using the Generalized Amber Forcefield with the Antechamber package provided with AmberTools16. All molecular dynamics simulations were run using Gromacs version 5.1.2.

	An ensemble of characteristic, low-energy vacuum monomer configurations were constructed by applying a simulated annealing process to a parameterized monomer.
	\begin{itemize}
		\item Initially parameterized monomer gives unfavorable structure
		\item Structure cooled from 1000 to 50 K over 10 nanoseconds
		\item Result not global minimum but close enough for structure building
		\item Multiple configurations saved from annealing trajectory to prove independence of starting config
                 %MRS: say what you mean by reparameterized.  Charges recalculated? Are the charges different from configuration to configuration (hopefully not).  Were multiple configurations used in the reparameterization?
		\item New configurations reparamaterized
	\end{itemize}
	
	The timescale for self assembly of monomers into the hexagonal phase is unknown and likely outside of a reasonable simulation length. 
	\begin{itemize}
		\item Work done that shows coarse grain model self assembly in ~1000 ns , Citation: J. Phys. Chem. B 2013, 117, 4254-4262
		\item Attempts with Colh system not fruitful  
                  %MRS: will eventually report what didn't work . 
                  %BJC Will we not report it here?
                  %MRS2: no, I meant you eventually will put here what didn't work.
		\item Wrote own code to assemble monomers into HII configuration (described in intro) close to what is expected -- equilibration simulations allow structure to relax into expected configuration 
		\item Code has adjustable parameters: Pore-to-Pore distance, pore radius, number of monomer layers (justification for parameters chosen in following paragraphs)
		\item This is the thermotropic phase, information used depends on thermotropic experimental data
%MRS: somewhere in the outline there should be a description of the difference between thermotropic and lyotropic phases.
%BJC: added in the intro around line 28 
%MRS2: the two are blended together a bit making it hard to see what is investigated where. We can discuss this today.
	\end{itemize}
	
	After some experimentation, it was found that twenty monomer layers per pore provided a balance of structural accuracy and computational efficiency. 
%MRS: worth it to describe the extent to which the final structure is independent of parameters.  
%BJC: So, mention how starting pore-to-pore affects the equilibrium structure, how different crosslinking parameters affect crosslinked structure etc.? Or are you just talking about crosslinking?
%MRS2: I'm just talking about building the system (before crosslinking) - distance between starting config, length of time equilibrated.  Because this is how you are showing that the builds procedure is reasonable and leads to similar structures independently of how you build it. Crosslinking later. 
	\begin{itemize}
		\item Space between membrane layers in z direction - semi-isotropic sims fix z box dimension
		\item Small number of layers create micellar structure
		\item Beyond 20 layers is unnecessary to get expected configuration
		\item layers spaced far enough apart to avoid large energy repulsions but close enough to prevent disks slipping between one another ~ 5 angstroms works.
	\end{itemize}
	
	\noindent The remaining parameters were chosen based on experimental data. 
	\begin{itemize}
		\item SAXS to get Pore-to-Pore
		\item TEM images and rejection give a pore size estimate
	\end{itemize}
	
	\noindent NPT simulations were run at 300 K and 1 bar for system equilibration.
	\begin{itemize}
		\item Monitored 'hexagonality' - i.e. how the distance between pores compares to an ideal hexagon
		\item Varied initial pore-to-pore distance and ran very long simulations
		\item Starting pore-to-pore distances of 4.0 and 4.5 converge to same average with longer sims 
	\end{itemize}
	
	After an equilibrated structure was created, a crosslinking procedure was performed in order to better parallel synthetic procedures. 
	\begin{itemize}
		\item Crosslinking maintains alignment of cylindrical mesophases
		\item head to tail addition
		\item racemic mixture 
		\item Details of crosslinking algorithm (refer to appendix or supplemental info)
	\end{itemize}  
	
	\section{Results and Discussion}
	
	The validity of our model can be tested based on its agreement with experimental structural and material property data
	\begin{itemize}
		\item Our simulations should mimic the real system as best as we can within the limits imposed by choice of force field
		\item Tougher with a large system for which no force fields have been specifically developed
		\item Direct comparison of conductivity measurements to experimental results
		\item SAXS structural comparisons
%MRS: this seems a bit repetitive - SAXS structural comparisons and conductivity are brought up in a few places.  Want to make it a cleaner outline. 
%BJC: Okay, I can probably delete this paragraph from this section. Maybe the structural stuff is better worked into the methods section since it is really about model development. 
%BJC: The ionic conductivity measurement and reason for them will be made clear once I actually start talking about it 
	\end{itemize} 
	
	A structurally accurate and reproducible model of the Colh phase LLC membrane has been developed
	\begin{itemize}
		\item Visual perspective - show top view and cross section
		\item System validated using multiple monomer configurations (working on this now)
		\item Pore to pore distance
		\begin{itemize}
			\item We know from SAXS data what the distance between pores should be
			\item We require long simulation times to reach an equilibrated structure with the correct dimensions
			\item Effect of crosslinking
		\item Pore Radius - a less reliable validation because we don't have an agreed upon way to measure this parameter experimentally
		\item Density - there are crude lab measurements which my model is in agreement with (no one has reported any values since it wasn't relevant to them)
		\end{itemize}
	\end{itemize}
	
	The monomers are arranged into disks containing XXXX monomers in each layer. (Or are there defined layers -- I need to test this idea a bit more) 
	%BJC: Do you think this point is even worth attacking at the moment?
        %MRS2: we sort of need to do this to convince ourselves that this approach is reasonable, so we should include it and show others it is reasonable.
	\begin{itemize}
		\item Hypothesis: There are an average of 7 monomers per layer when defined per unit volume but there are not well-defined layers as pictured in the literature. While long range order is maintained, hexagonal mesophases are disordered within their hydrophilic and hydrophobic domains. Staying completely ordered, stacked on top of each other is not entropically favorable.
%MRS: one thing we could to to test to see if pi-pi stacking could order them.  We introduce an explicit dipole in the middle of the aromatic ring, all aligned (I can show you how to set this up using the virtual_sites command). We see how strong it neds to be in order to make the rings line up - or if any strength will really create that ordering. One could imagine that when the dipole gets strong enough, it just pulls ions out of the channel.
%BJC: I remember us talking about this during one of our meetings. I assume this shouldn't be too hard to set up and simulate before the paper is submitted?
%MRS2: we can talk about what would need to be done.  We need to get the dipole in the right direction for each monomer. 
		\item We can understand how monomer shape/size/length influence pore density
%MRS: I'm not quite sure how the next sentence fits in. You mean we want to design a membrane that has more pores, and so we want materials with larger pores?
%BJC: What I'm trying to say is that if we can make the membrane more porous - closer together pores mostly - then we can presumably increase ionic conductivity and permeability
%MRS2: OK, you want to put this where it is connected directly with the other information. Perhaps the thesis of it's own paragraph, with supporting sentences about how one could use MD simulations to do this?
		\item Ideally we'd want as many pores as possible per membrane area. More monomers per layer means more material needed and less pores formed. 
	\end{itemize}
	Sodium counterions are distributed randomly within the hydrophilic pore regions. (I should actually come up with a distribution e.g. gaussian)
	\begin{itemize}
		\item Past literature reports ions arranged in a circle about pore walls implying some kind of void space as a pore. Simulations show that they are not
%MRS: I think we need to be a litte careful. If we see that the pores form this way under a variety of conditions (just the ones we've looked at - different numbers of monomers per layer, different spacing, then that provides strong evidence that the more aligned form is less probable. 	
%BJC: So by be careful, you mean word it a little more delicately, for example: "Simulations under a variety of conditions suggest that ions prefer disorder within the pores"
%MRS2: yes.
%MRS2: below -- hinder transport of what?  Both other ions and water?  Probably need to explain more clearly what people though before about whether they hindered transport or not in order to show that the role is ``different than previously thought''
		\item Ions may play a different role in transport than previously thought. They might hinder transport by slowing diffusion of other sodiums and molecules.
%MRS2: should be careful, since they are not necessarily predicting flux yet.  We do know that they have good ionic conductivity. 
		\item  Experimentalists may be over-predicting flux through pores because they view it as a hollow channel.
        \end{itemize}	
	The model gives reasonable estimates of ionic conductivity.
	\begin{itemize}
		\item There are a few ways to estimate ionic conductivity as seen in literature. We prefer a method which can extract an estimate based purely on an equilibrium trajectory
		\item We must also be sure that our analysis of results is consistent with the method use for experimental evaluation (i.e. AC impedance spectroscopy)
%MRS2: clarify in what situations the azimuthal distribution is necessary.
		\item We must also account for the azimuthal distributions, linking our perfectly straight microscopic system to the not-so-straight macroscopic system
		\item Two methods used to for prediction
		\item Nernst Einstein Relation:
		\begin{itemize}
			\item Widely used equation for estimating ionic conductivity
			\item Estimates DC ionic conductivity -- Frequency used during AC impedance slow enough to be approximated by dc at short enough timescales
			\item Relates the diffusive motion of ions in the membrane to the membrane's ionic conductivity
			\item Concentration is concentration of ions in the whole membrane, not just channels
		\end{itemize}
		\item Collective Diffusion:
		\begin{itemize}
			\item Defines a collective coordinate, Q (charge), to quantify the amount of charge transfer through the system
			\item In the limit of infinite time, the MSD of Q can be used to formulate a diffusion coefficient of Q that can be related to ionic conductivity
			\item The model is valid for non-equilibrium and equilibrium simulations. Our analysis is based on the latter
			\item A similar model has been derived and validate to predict water permeability using equilibrium simulations
			\item The pore region is defined as the entire membrane system since lab IC measurements are done on bulk membrane rather than on individual pores. One would expect single channel IC to be much larger than the bulk membrane
			\item Changing the valence of the counterion may also increase ionic conductivity (I need to actually test this by increasing the charge on sodium for a calculation)
                \end{itemize}                    
%MRS: increasing charge on the sodium alone probably won't work that well unless you also increase the charge on the carboxylate, since you will then have a lack of electroneutrality.
%BJC: right, that makes sense. In which case, the fluctuations might cancel and give similar conductivity - maybe I'll just stay away from it for now. I just though it might be nices to have some hypotheses
%about how to improve ionic conductivity based just on the modeling done so far.
	\end{itemize}

	Our model can be used to evaluate other liquid crystalline assemblies
		\begin{itemize}
			\item Liquid crystal used - or a mixture of different liquid crystals
			\item Minor structural variations (e.g. 7, 8, 9, 10, 11 ... CH2's in the tails)
			\item Counterion -- Size/valence of counterion
			\item Functional head groups
			\item How does varying these things effect pore separation, pore size, phase stability, and transport and why are these effects observed
		\end{itemize} 
		
	\section{Conclusion}

	In this work, a molecular model has been developed intended to predict transport in a thermotropic liquid crystal membrane. 
	\begin{itemize}
		\item Results presented for Colh phase monomer but can be adapted to other LCs
%MRS: will need to describe more what the differences are.  Is it just whether it's solvated or not? (i.e. different compositions, but structures are similar?)
%BJC: I'm not sure I follow, differences between what exactly? All I am trying to say is that we can stick other monomers in place of NAGA3C11 and see what happens.
%MRS2: got it.
		\item Model can be used for prediction of transport properties in new membranes
		\item We will solvate the system to establish procedures for HII phase prediction
	\end{itemize}
	
	
\end{document}
