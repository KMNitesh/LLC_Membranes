\documentclass{article}
\title{Predicting Transport in Lyotropic Liquid Crystal Membranes with Molecular Dynamics Simulations -- Outline}
\author{Benjamin J. Coscia, Michael R. Shirts, Douglas L. Gin, Richard D. Noble}
\begin{document}
	\bibliographystyle{ieeetr}
	\maketitle
	\section{Introduction}
	
	Nanostructured membrane materials have become increasingly popular in desalination and wastewater treatment applications because they offer the ability to control membrane architecture at the atomic scale.
	\begin{itemize}
		\item Current state-of-the-art reverse osmosis membranes are dense and unstructured with tortuous and polydisperse pores which lead to inconsistent performance
		\item Tortuosity and polydispersity drive up energy requirements which strain developing regions and contribute strongly to CO2 emissions
		\item With nanostructured materials, solute rejecting pores can be tuned uniformly -- drives down energy requirements
	\end{itemize}
	
	Development of nanostructured materials has been limited by the ability to synthesize and scale various fundamentally sound technologies.
	\begin{itemize}
		\item Leading technologies and their limitations:
		\begin{itemize}
			\item Graphene sheets -atomically thick which gives excellent permeability but defects during manufacturing severely impact performance
			\item Carbon Nanotubes - MD studies are promising but synthetic techniques unable to achieve necessary alignment and pore monodispersity
			\item Zeolites - Good rejections/selectivities, but low flux and high cost of manufacturing. Generally limited to applications needing thermal/chemical resistance
		\end{itemize}
	\end{itemize} 
	
	Lyotropic Liquid Crystals (LLCs) share the characteristic ability of nanostructured membrane materials to create highly ordered structures with the benefits of low cost and synthetic techniques which are feasible for large scale production.
	\begin{itemize}
		\item Forms lamellar, bicontinuous cubic and hexagonal phase based on solution composition
		\item My monomer forms HII phase consisting of vertically aligned, monodisperse, straight pores in a hexagonal array
		\item Hydrophilic regions point towards pore centers
		\item Until recently, they could not be aligned - hindered progress
		\item Talk about Yale alignment techniques
		\item LLC HII phase membranes offer potential for high permeability and selectivity which equals low energy consumption
	\end{itemize}
	
	Constructing a molecular model may give a clearer picture of the microscopic structure of LLC membranes which will change how we think about their design.
	\begin{itemize}
		\item The arrangement of sodium ions in the channels is thought to be confined to the pore walls. It is possible they are arranged more randomly
		\begin{itemize}
			\item This could change how one thinks about molecules diffusing through membrane
			\item Could also be a difference between lyotropic and thermotropic phases
		\end{itemize}
		\item How many monomers are in a layer?
		\begin{itemize}
			\item There is no clear answer to this question in literature. I'd like to confirm or deny past predictions
		\end{itemize}
		\item What is the equilibrium water content in lyotropic phase?
		%BJC: There is a paper which synthesizes the membrane with different starting water content in solution. They find no difference in the final structure of the of the membrane with water contents ranging from 7 - 10 %. I want to see what the water content is at equilibrium as a way to inform experimentalists who are doing sythesis -- It's not super important but it addresses a question that has been asked in the past.
	\end{itemize}
	
	A molecular level understanding of solute transport in LLC membranes will accelerate development efforts by providing guidelines for monomer design.
	\begin{itemize}
		\item HII phase studies currently limited mostly to one monomer with minor variations
		\item Current optimization efforts performed through trial and error
		\item Macroscopic models are the only source of predictive modeling
		\item An atomistic understanding of the mechanism of solute transport can identify
		performance bottle necks and direct design of future monomers/membranes
		%MRS: this is getting to be a lot of points under this.  Talking about the specifically understood mechanisms should probably be a separate thesis, perhaps outlining the specific questions that are not known about these membranes.
		%BJC: There isn't much in the literature about specifically understood mechanisms for the HII phase. All of the discussion pertains to the bicontinuous cubic phase. The Donnan model applies to both in theory so I could talk more about that. There are also the generic membrane transport equations relating flux to applied pressure, pore size, membrane thickness etc. based on Hagen-Poiseuille flow in the pores.
		\item Will we observe Donnan exclusion or some variation on it at the pore entrance
		\item What are the relative weights of steric and electrical effects that deterimine selectivity -- is there an optimal balance
		\begin{itemize}
			\item Look at Coulombic vs. Lennard Jones interactions
			\item Do ions or water have trouble getting through because of interactions with other things in the pores (e.g. ions, water, carbonyl groups, benzene rings)
			\item Does concentration of ions in the pore repel incoming ions?
		\end{itemize}
	\end{itemize}
	
	A physically accurate molecular model with easily modified structural features will launch a new phase of research into directing design of an optimal LLC membrane.
	\begin{itemize}
		\item The paper will illustrate the development of a predictive molecular model
		\item Measurements made to validate model
		\item Observations of transport mechanism
		\item The model is consistent with experiments and ready to extend to other systems
	\end{itemize}
	
	\section{System Setup}
	
	HII monomers were parameterized using the Generalized Amber Forcefield with the Antechamber package provided with AmberTools16. All molecular dynamics simulations were run using Gromacs version 5.1.2.
	
	The most suitable monomer structural configuration was decided by applying a simulated annealing process to the parameterized monomer. 
	\begin{itemize}
		\item Initially parameterized monomer gives unfavorable structure
		\item Structure cooled from 1000 to 50 K over 10 nanoseconds
		\item Result not global minimum but close enough for structure building
		\item Multiple configurations saved from annealing trajectory to prove independence of starting config
	\end{itemize}
	
	The timescale for self assembly of monomers into the hexagonal phase is unknown and likely outside of a reasonable simulation length. 
	\begin{itemize}
		\item Work done that shows coarse grain model self assembly in ~1000 ns
		\item Attempts with HII system not fruitful
		\item Wrote own code to assemble monomers into HII configuration (described in intro) close to what is expected -- equilibration simulations allow structure to relax into expected configuration 
		\item Code has adjustable parameters: Pore-to-Pore distance, pore radius, number of monomer layers (justification for parameters chosen in following paragraphs)
		\item This is the thermotropic phase, information used depends on thermotropic experimental data
	\end{itemize}
	
	After some experimentation, it was found that twenty monomer layers per pore provided a balance of structural accuracy and computational efficiency. 
	\begin{itemize}
		\item Space between membrane layers in z direction - semi-isotropic sims fix z box dimension
		\item Small number of layers create micellar structure
		\item Beyond 20 layers is unnecessary to get expected configuration
	\end{itemize}
	
	The starting distance between layers was chosen so that energy minimization occured with no errors due to overlaps. 
	\begin{itemize}
		\item Layers collapse to favorable layer-to-layer distance within first few picoseconds of equilibration 
	\end{itemize}
	
	\noindent The remaining parameters were chosen based on experimental data. 
	\begin{itemize}
		\item SAXS to get Pore-to-Pore
		\item TEM images estimate pore size
	\end{itemize}
	
	\noindent NPT simulations were run at 300 K and 1 bar for system equilibration. 
	\begin{itemize}
		\item Monitored 'hexagonality'
		\item Pores don't show perfect hexagonal character for short sims (~20 ns) 
		\item We see a smaller average pore-to-pore distance than reported
		\item Varied initial pore-to-pore distance and ran very long simulations
		\item Starting pore-to-pore distances of 4.0 and 4.5 converge to same average with long\textbf{er} sims 
	\end{itemize}
	
	After an equilibrated structure was created, a crosslinking procedure was performed in order to better parallel synthetic procedures. 
	\begin{itemize}
		\item Crosslinking maintains alignment of cylindrical mesophases
		\item head to tail addition
		\item Details of crosslinking algorithm (refer to appendix or supplemental info)
	\end{itemize}
	
	\noindent The crosslinked system was then solvated with water to mimic a membrane in water.
	\begin{itemize}
		\item x ns sims run to allow full water penetration into pores
		\item simulated isotropically with enough water in between periodic images
	\end{itemize}  
	
	\section{Results and Discussion}
	
	The final, equilibrated system had pores which were spaced 3.5 nm apart. This is the thermotropic phase though. More to come once system is solvated. 
	\newline 
	\newline \noindent Questions to be answered and experiments to test them:
	
	\begin{itemize}
		\item Question: How many monomers are in each layer? Do we even see defined layers?
		\begin{itemize}
			\item \textbf{Why this is important:} 
				\item We can understand how monomer shape/size/length influence pore density
				\item Ideally we'd want as many pores as possible per membrane area. More monomers per layer means more material needed and less pores formed.
				\item We can see whether the effects of pi stacking are as pronounced as hypothesized in other papers.
					\item fixed-charge models may underpredict pi-stacking -- searching for a reference to confirm this although that would make sense. How would this limit what we can say about the pore structure?
			\item \textbf{Experiments to be run:}
				\item Use membrane thickness and approximate pore size to define cylindrical disk regions. Measure number of monomers in these regions
				\item Visualization will help too
			\item \textbf{Hypothesis:} There are an average of 7 monomers per layer when defined per unit volume but there are not well-defined layers as pictured in the literature. While long range order is maintained, hexagonal mesophases are disordered within their hydrophilic and hydrophobic domains. Staying completely ordered, stacked on top of each other is not entropically favorable.
		\end{itemize}
		\item What is the mechanism of sodium transport through the pores?
		\begin{itemize}
			\item \textbf{Why this is important:}
			\item Past literature reports ions arranged in a circle about pore walls leaving void space as a pore. Simulations show that they may be arranged gaussian-like	
			\item Ions may play a different role in transport than previously thought. They might hinder transport by slowing diffusion of other sodiums and molecules.
			\item  Experimentalists may be over-predicting flux through pores because they view it as a hollow channel.
			%MRS: should make it clearer why each of these questions matter - how would knowing the answers improve the ability to make predictions about macroscopic membrane properties?
			\item \textbf{Experiments to be run and what they can tell us:} 
				\item Measure timescales of counter-ions binding to carbonyl groups. The residence time of a sodium ion -- get scripts from Drew.
				\item Observe positions of bulk sodium ions over time.
					\item Do they go all the way through or do they cause some sort of momentum transfer that pushes a different ion out of the pore on the opposite side? I would expect higher rejections if the sodium ion is required to diffuse all the way through
				\item See if we can observe ion diffusion through hydrophobic matrix. Is it negligible? If it's not negligible, there is whole new issue to consider
				\item Observe influence of chlorine ion in NaCl? The hydrated NaCl has to fit in the pore. Does it ion exchange with bound sodiums.
			\item \textbf{Hypothesis:} Sodium ions that are already acting as counterions stay mostly bound to their functional groups. Sodium ions diffusing through from bulk salt water need to diffuse all the way through and drag along a chlorine atom to maintain charge neutrality. Sodium ions tethered to chloride may exchange with sodium ions bound to monomer. There is negligible diffusion of sodium chloride through hydrophobic matrix.
		\end{itemize}
		\item What is the mechanism of water transport through the pores?
		\begin{itemize}
			\item textbf{Why this is important:}
			\item Bottle necks in membrane design which limit water transport will directly affect membrane performance with regard to selectivity and rejection. 
			\item The configuration/structure of water molecules in confined regions is not well understood
			\item \textbf{Experiments to be run and what they can tell us:}
			\item Observe hydration shell around entering salt. Do waters get stripped from hydration shell?
			\item Quantify how much free water goes through vs. how much gets through as a part of hydration shell. More free water would mean higher water flux and higher rejection. 
			\item Measure water transport down the center of the pore vs. on the pore walls. Do water molecules orient themselves in a specific way at the pore walls? If water primarily passes through the pore center, maybe there is a modification that can be made so the walls are 'slippery-er' and more water can get through. 
			\item \textbf{Hypothesis}: Water molecules will travel with through the pores with a parabolic laminar-like trajectory. Water molecules along the wall will be hindered as they get stuck in the hydrophilic head groups causing slower transport at the walls -- This hypothesis will depend on what I see once I solvate and equilibrate the system with water. I think the arrangement of sodium ions will change when solvated (That could be an intersting conclusion in itself). No water will be transported through hydrophobic regions.    
		\end{itemize}
		\item What are the relative rates of diffusion of water and sodium? 
		\begin{itemize}
			\item \textbf{Why this is important:}
			\item The relative rates of diffusion are ultimately what characterizes a membrane's performance. These are what must be optimized.
			\item \textbf{Experiments to be run:}
			\item Calculate diffusion of water and sodium. 
			\item Use these values to calculate selectivity and rejection
		\end{itemize}
		\item How can we validate the model?
		\begin{itemize}
			\item textbf{Why this is important:}
			\item Our simulations should mimic the real system as best as we can within the limits imposed by choice of force field
			\item \textbf{Experiments to Run:}
			\item Direct comparison of conductivity measurements to experimental results
			\item SAXS structural comparisons
		\end{itemize} 
	\end{itemize}
	\section{Future Work}
	We can vary the following to make predictions about new systems:
		\begin{itemize}
			\item Liquid crystal used - or a mixture of different liquid crystals
			\item Minor structural variations (e.g. 7, 8, , 9, 10, 11 ... CH2's in the tails)
			\item Counterion -- Size/valence of counterion
			\item Functional head groups
			\item How does varying these things effect pore separation, pore size, phase stability, and transport and why are these effects observed
		\end{itemize} 
	\section{Conclusion}
	
	In this work, a molecular model has been developed which can predict transport in an lyotropic liquid crystal membrane. 
	\begin{itemize}
		\item Results presented for HII phase monomer but can be adapted to other LLCs
		\item Model can be used for prediction of transport properties in new membranes
	\end{itemize}
	
\end{document}
