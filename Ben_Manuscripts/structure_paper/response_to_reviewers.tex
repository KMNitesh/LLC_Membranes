\documentclass{article}
\usepackage[margin=1in]{geometry}
\usepackage{amsmath}
\usepackage{xr}
\usepackage{xcolor}
\usepackage{siunitx}
\usepackage{gensymb}

\externaldocument[S-]{supporting_information}
\externaldocument[M-]{Final_Draft}

\begin{document}

%MRS: making the title a bit smaller, removing date.
\begin{center}
\textbf{Response to reviewers: Understanding the Nanoscale Structure of Inverted Hexagonal Phase Lyotropic Liquid Crystal Polymer Membranes}: \\
Authors: Benjamin J. Coscia, Joseph Yelk, Matthew A. Glaser, Douglas L. Gin, Xunda Feng, and Michael R. Shirts
\end{center}

We thank the reviewers for carefully reading over our manuscript and providing
helpful comments. We have taken the suggestions into consideration and made
appropriate revisions to the manuscript document. All changes to the text have
been documented below. In cases where we made major revisions, we included the
original text (denoted by ``Original text:") followed by the new text (denoted by
``New text:". If we only made additions or small changes to the existing text, 
we highlighted them in red (denoted by ``Modified text:").

\section{Response to Reviewer 1}

\begin{enumerate}
	
    \item \begin{quote} \textit{This manuscript describes atomistic molecular
	    dynamics simulations and diffraction analysis of hexagonally-arranged pores,
	    each of which is made of many columns of the monomer Na-GA3C11. This physical
	    system is of interest because such nanopores have potential to transform
	    filtration applications. This work is of potential interest to the JPC
	    readership because it provides atomistic insight into the origins of
	    diffraction reflections observed in experiments, which could be used to resolve
	    discrepancies about what the structure of these hexagonal arrays are, or used
	    to engineer better filters. Overall, the manuscript provides comprehensive
	    descriptions of metrics used to evaluate the simulated structures against
	    experiments, but the number of things described and length of these
	    descriptions obscures whether significant new physical insight into this system
	    is made.} 
    \end{quote}
	
    Author reply: We thank the reviewer for their constructive feedback and aim to 
    address all of their concerns. 
	
    \item \begin{quote}
    \textit{As written, it is not clear if the authors are presenting this
	    study as a validation of GAFF and the present equilibration techniques, or
	    whether this study is providing significant new physical insight into these
	    lyotropic liquid crystals. The paper reads a bit more like the former, with a
	    comprehensive discussion of the degree to which five scattering features
	    expected in experiments are predicted in simulations. However, at the same
	    time, the authors make arguments including (1) pores are made from five columns
	    of monomers, (2) that water is present in the ``dry" structures, and (3) that a
	    scattering feature comes from tail packing, not tilt: These observations
	    *could* be new physical insight into lyotropic liquid crystal structure, but as
	    presented seem more like hypotheses than conclusions:}

    \end{quote}

    Author reply: We thank the reviewer for offering their interpretation of
    the purpose of our study, as it is important that we properly convey our
    message. The main purpose of our work is to give a detailed description of the
    structure of a lyotropic liquid crystal membrane backed by experimental data.
    This work lays a foundation for future transport studies. By matching 2D-WAXS
    data, we have attempted to unveil the most experimentally-consistent structure
    possible. To clarify this purpose, we have modified and added text to the
    paragraph at the end of page~\pageref{M-rework:purpose} of the introduction, 
    in which we state our perceived importance of knowing the most 
    experimentally-consistent structure:

    Original text: ``A molecular-level understanding of LLC polymer membrane
    structure, enabled by molecular dynamics (MD) simulations, can provide
    guidelines to reduce the large chemical space available to design monomers for
    creation of separation-specific membranes. Useful molecular-level modeling
    should incorporate a detailed picture of the nanoscopic pore structure, which
    is crucial to understanding the role of monomer structure in solute transport
    and membrane design. 

    New text: ``A molecular-level understanding of LLC polymer membrane
    structure, enabled by molecular dynamics (MD) simulations, will enhance our
    understanding of the role of monomer structure in solute transport and provide
    guidelines to reduce the large chemical space available to design monomers for
    creation of separation-specific membranes. It is important that we model the
    system with a nanoscopic pore structure that is maximally consistent with
    experimental data in order to best approximate the chemical environment experienced
    by solutes in future simulated transport studies."

    We do not intend for this study to be a validation of GAFF for this
    particular system. We believe we would come to similar conclusions using other
    force fields, however validation of this hypothesis is left for a future study.
    We chose GAFF because it is widely used to simulate organic molecules and there
    are no catch-all force fields built for liquid crystal systems. We are interested in
    what information we can gain from a widely-used `out-of-the-box' force field and in future
    work, we will look at sensitivity of the structural results to force field parameters.
    To clarify our use of GAFF, we have added the following text to 
    Secton~\ref{M-method:parameterization} of the Methods on
    page~\pageref{M-method:parameterization}.

    Modified text: ``We parameterized the interaction potential for the liquid crystal
    monomer Na-GA3C11 using the Generalized AMBER Force Field (GAFF) $^{26}$ with the
    Antechamber package $^{27}$ provided with AmberTools16.$^{28}$ {\color{red} We 
    chose GAFF because it has been parameterized for use with organic molecules. 
    Although researchers have optimized force fields, including GAFF, for use with 
    liquid crystals, they are often tuned in order to reproduce one or more experimental
    observables for specific types of liquid crystal molecules.$^{29,30}$ Na-GA3C11
    does not have a structure characteristic of more commonly studied liquid crystal
    monomers so we chose to stick with a widely used force field. If we can reproduce
    experimental trends using a single `out-of-the-box' force field, then there will 
    be no need to complicate our studies with customized parameter sets.}

    We have added the following citations:
    \begin{itemize}
    \item Cook, M. J.; Wilson, M. R. Development of an All-Atom Force Field for the Simulation
    of Liquid Crystal Molecules in Condensed Phases (LCFF). Mol. Cryst. Liq. Cryst.
    2001, 357, 149–165.
    \item Boyd, N. J.; Wilson, M. R. Optimization of the Gaff Force Field to Describe Liquid
    Crystal Molecules: The Path to a Dramatic Improvement in Transition Temperature
    Predictions. Phys. Chem. Chem. Phys. 2015, 17, 24851–24865.
    \end{itemize}

    Since our ultimate goal is to design new monomers for solute-specific
    separations, we will be employing the same techniques used in this work to
    unsynthesized monomers. In most cases, we will not have an experimental dataset
    to use for comparison. Despite all of the structural intricacies which we have
    revealed in this study, one of our main conclusions is that for any of the 
    systems we studied in this work, the radial distribution of monomer components
    stays consistent. From this, we can hypothesize that we will extract similar
    transport results regardless of which configuration we use to conduct
    transport studies. This is an extremely important finding which may have
    been undersold in the original draft of the manuscript. We have added the
    following concluding paragraph to the end of
    Section~\ref{M-section:pore_composition} on
    page~\pageref{M-addition:pore_composition} in order to drive the point home:

    New text: ``Our observations suggest that the details of transport may be
    relatively independent of the structural differences between different possible
    structures studied here. Despite the structural intricacies that give rise to
    differences between the various metastable basins, all pores are characterized
    by dense, primarily hydrophilic cores with a gradient towards a primarily
    hydrophobic tail region far from the pore center. This implies that we might
    see the same trends in transport properties from any of the systems studied so
    far. Since the ultimate goal of our work is to design new monomers using high
    throughput simulations, achieving that goal becomes tractable without the need
    to optimize every system with an experimental dataset."
	
    We also modified a paragraph in our conclusion on page~\pageref{M-rework:conclusion}
    to better emphasize our point:

    Original text: ``We characterized the environment centered around the membrane
    pores and learned that the pores are generally filled with monomer head groups
    and sodium ions. All dry systems studied showed a similar distribution of
    sodium, head groups and tails while the wet system shows evidence of slight
    swelling, with minor changes in the distributions due to the presence of water
    molecules. We also observed that there is not a hard partition between
    hydrophilic and hydrophobic regions; instead, there is a gradient of chemical
    constituents. This finding raises questions about the nature of size-exclusion
    separations in systems without a well-defined pore size, which potentially
    could enable separations that vary with chemical identity as well as size. 	

    New text: ``Although exactly reproducing the experimental 2D-WAXS pattern
    required attention to minute structural details, we found that all systems
    showed similar radial distribution functions characterized by a gradual radial
    transition from a dense hydrophlic core to hydrophobic tails. This observation
    enables us to study new systems that have not been experimentally characterized
    with the expectation that any additional structural optimization will not
    greatly influence trends in transport property predictions. The compositional
    gradient itself raises questions about the nature of size-exclusion separations
    in systems without well-defined pore boundaries, which potentially could enable
    separations that vary with chemical identity as well as size."
	
    \item \begin{quote}
    \textit{1. For the simulations performed here, the authors do make a strong
	    case that 5 columns per core gets the best pore-to-pore spacing (Figure 5).
	    However, insufficient time is accessed to see whether the 4-, 5-, and 6-column
	    initial conditions move towards or away from the expected pore spacing over
	    many-microseconds trajectories. That said, for the quenches performed here,
	    5-columns per core seems more plausible than 4.}
    \end{quote}
	
    Author reply: We agree with the reviewer that 5 columns-per-core is more
    plausible than 4 and that much longer simulations could provide further
    supporting evidence. However, there is very little motion on the hundreds of
    nanoseconds timescale that we simulated, which suggests that we will not gain
    any significant new insight by extending our simulations into the single microsecond
    regime. It is not feasible at the present time to push the simulations at least
    an order of magnitude longer.

    Some confusion may have been generated by agreement of the disordered basin
    6 column-per-pore system with experimental pore spacing. As mentioned in the
    text on page~\pageref{M-6colperpore}, the vertical stacking distance between 
    monomers in that system is far too large. The columns are `stretched' in the
    $z$-direction allowing the pores take up less space on the $xy$ plane and 
    thus pack closer together. We have modified the text starting at the very 
    bottom of page~\pageref{M-6colperpore} for clarification:
    
    Original text: ``Monomers in disordered basin systems built with 6 
    columns-per-pore agree with experimental pore-to-pore distances within error, 
    but stack too far apart. 6 column-per-pore disordered sandwiched and 
    disordered parallel displaced configurations stack $\sim$~4.87 and 4.94 
    \AA~apart respectively, which is $\sim$ 1.2~\AA~further apart than 
    suggested by experiment."
    
    New text: ``Monomers in disordered basins systems stack too far apart,
    therefore they are not suitable candidate structures. Although their pore-to-pore
    spacing is in close agreement with experiment, the 6 column-per-pore 
    disordered sandwiched and disordered parallel displaced configurations 
    stack $\sim$~4.87 and 4.94 \AA~apart respectively, which is $\sim$ 
    1.2~\AA~further apart than suggested by experiment. The monomer columns
    in these systems are effectively stretched in the $z$-direction which 
    allows them and the pores to pack closer together on the $xy$ plane."

    \item \begin{quote}    
    \textit{Is this significant physical insight? It is insight, but the
	    authors do not make a compelling case for how 5 columns would change how we
	    think about these crystals, or how it would impact membrane engineering (or
	    more importantly, whether these 5-column structures assemble in the solution
	    phase, or are only locally stable).}
    \end{quote}
    
    Author reply: We thank the reviewer for pushing us to provide more
    significant physical insight based on the number of columns-per-pore. The
    number of columns-per-pore influences the pore radius which is important for
    determining the limits of size-exclusion-based separations. It also tells us
    the density of interacting sites within the pore which may play a role in small
    molecule transport. We have added the following text to the main document at
    the end of Section~\ref{M-section:mon_per_pore} on page~\pageref{M-addition:pore_radius}:
    
    New text: ``The number of columns-per-pore dictates the density of interaction sites
    within each pore and the pore radius. A higher number of interaction sites will
    surely play a role in transport of molecules that have an affinity for the
    monomer head groups.  The pore radius increases with the number of columns per
    pore (See Figure~\ref{S-fig:pore_radius}). This radius is likely related to the
    size of the molecules which can be excluded by this type of membrane. One may be
    able to control LLC membrane pore size by selecting monomers that are stable in
    configurations with more columns-per-pore."
    
    We are unable to definitively answer the question of whether the 5-column 
    structures are most likely to assemble in solution phase. The system does
    not self-assemble on a timescale reasonable for simulation. See 
    Section~\ref{S-section:self_assembly} of the supporting information where
    we attempted self-assembly. These sorts of predictions are left for 
    a separate effort that will use coarse grain models.
    
    \item \begin{quote}
    \textit{2. The authors make a pretty compelling case that water causes the
    R-double scattering feature by showing its absence when water is added, but
    that the R-double appears and persists for 200ns after the water makes its way
    into the crystal. However, the authors also show that nonuniformly spaced
    monomers can give rise to the R-Double feature, a seeming counterexample to the
    necessity of water.  Is the R-double feature with water because the monomers
    are now irregularly spaced? What happens when the waters are removed and the
    scattering re-performed in Fig 17c(bottom)?}
    \end{quote}
    
    Author reply: We thank the reviewer for exposing a potentially unclear
    explanation of the origin of R-double. R-double appears in a dry system with
    non-uniformly spaced monomers \textit{only} if $z$-direction position
    restraints are applied to monomer head groups in order to maintain unevenly
    spaced stacking. As soon as the restraints are released, R-double fades and the
    monomer spacing becomes uniform. We modified the text in the first full 
    paragraph of page~\pageref{M-modification:rdouble} in order to more clearly
    convey this message:
 
    Original text: ``We can also produce R-double if the LLC monomers are not
    uniformly spaced in the $z$-direction, but instead form pairs that stack less
    than 3.7~\AA~apart, with COMs that are spaced 7.4 \AA~ from neighboring pairs
    of monomers (Figure~\ref{M-fig:staggered_rzplot_norestraints}). Our force field
    causes our system to tend towards uniformly spaced layers. Simulations of
    unevenly spaced systems are only stable if position restraints are applied to
    heavy atoms of the phenyl rings. Additionally, there is little evidence from QM
    studies of stacked $\pi-\pi$ systems that such uneven stacking could be
    energetically stable.$^{52}$%.~\cite{tauer_beyond_2005}"

    New text: ``We can also produce R-double if the LLC monomers are not
    uniformly spaced in the $z$-direction, but instead are placed in pairs that
    stack less than 3.7~\AA~apart, with COMs that are spaced 7.4 \AA~ from
    neighboring pairs of monomers (Figure~\ref{M-fig:staggered_rzplot_norestraints}).
    Simulations of unevenly spaced systems are only stable while position
    restraints are applied to heavy atoms of the phenyl rings. There is little
    evidence from quantum mechanical studies of stacked $\pi-\pi$ systems that such
    uneven stacking could be energetically stable.$^{52}$ As soon %~\cite{tauer_beyond_2005} As soon
    as we remove position restraints, our system immediately moves towards uniformly spaced
    monomers."

    R-double is only long-term stable without position restraints if a small
    amount of water is present. We believe that water facilitates the appearance
    of R-double by stabilizing non-uniform spacing through hydrogen bonding, as the
    reviewer suggests. We clarified the text on page~\pageref{M-modification:rdouble2}
    in order to make this conclusion clear:

    Original text: ``When two vertically stacked monomer head groups hydrogen bond
    with a shared water molecule, the monomers are drawn closer together (as
    illustrated in Figure~\ref{M-fig:hbond_visualization}), which creates an
    asymmetry that allows R-double to appear. If a monomer head group shares a
    hydrogen-bonded water molecule with a head group above itself, it will be less
    likely to share a water molecule with a head group below it due to geometric
    constraints. The monomer head group below can just as easily share a water
    molecule with a head group below itself. In this scenario, the COMs of each
    pair are 2 times the $\pi$-stacking distance apart which would lead to R-double
    (much like the configuration in
     Figure~\ref{M-fig:staggered_rzplot_norestraints}). There are a modest number of
    occurrences of this scenario, which we quantify in further detail in the SI,
    Section \ref{S-section:rdouble}."

    New text: ``The addition of a small amount of water to the pores stabilizes
    non-uniform stacking of head groups much like that shown in
    Figure~\ref{M-fig:staggered_rzplot_norestraints}. When two vertically stacked
    monomer head groups hydrogen bond with a shared water molecule, the monomers
    are drawn closer together (as illustrated in
    Figure~\ref{M-fig:hbond_visualization}), which creates the asymmetry that allows
    R-double to appear. If a monomer head group shares a hydrogen-bonded water
    molecule with a head group above itself, it will be less likely to share a
    water molecule with a head group below it due to geometric constraints. The
    monomer head group below can just as easily share a water molecule with a head
    group below itself. There are a modest number of occurrences of this scenario,
    which we quantify in further detail in the SI, Section
    \ref{S-section:rdouble}."

    The reviewer asked us to re-perform the scattering analysis on the same
    system with water molecules removed. We would like to emphasize that we
    attempted to make a similar argument by removing just the head group atoms from
    the trajectory and re-performing the scattering analysis
    (Figure~\ref{M-fig:rdouble_nophenyls}). In the absence of head groups, R-double
    is not present, implying that their structure (which is influenced by the
    presence of water molecules) is responsible for the R-double reflection. For
    completeness, we performed the scattering analysis on the same system with
    water molecules removed, as suggested by the reviewer, and R-double is still
    present. This is the expected result since the structure of water itself does
    not give rise to R-double. We've added the resulting simulated XRD pattern to
    the Supporting Information, Figure~\ref{S-fig:rdouble_water_removed}. 
   
    \item \begin{quote}
    \textit{3. The authors make the case that the R-spots feature comes from the 
    structure of the hexagonal tails, not their tilt. The authors should perform the
    scattering analysis to check if the R-spots are present on the tilted-tails they
    constrain: This quick experiment would nail down their interpretation here, or 
    show that the tilted tails are also plausible (but not observed in this model).}
    \end{quote}

    Author reply: We thank the reviewer for their suggestion. We do not deny
    that the tilt of the monomer tails is a plausible explanation for the
    appearance of reflections that resemble R-spots in many liquid crystal systems.
    While the reviewer's suggested experiment is a good idea, its results in this
    case are ambiguous for a couple reasons. First, the tails may still pack
    hexagonally in the tilted configuration which would independently give rise to
    R-spots. Second, it is necessary to impose high force constant ($>$ 1000 kJ mol$^{-1}$
    nm$^{-2}$) position restraints in order to maintain a tilt angle close to that
    implied by the 2D WAXS data. The resulting high degree of order in the system 
    gives way to reflections which are easily misinterpretted. We have added the
    following text, highlighted in red, to the text shown in the first full paragraph of
    page~\pageref{M-addition:rspots} in order to acknowledge that tail tilt is a 
    valid explanation of R-spot-like reflections: 
 
    Modified text: ``Previous literature has attributed the R-spots reflection
    in this particular WAXS dataset as the result of tilted alkane
    chains $^{16}$. {\color{red} This explanation is not  % feng_scalable_2014
    unfounded as there are examples in literature of systems where tilted liquid
    crystals give rise to reflections that resemble R-spots.$^{45,46}$} We
    measured the tilt angle of the alkane chains ..."
    
    We have added the following citations in support of the above statement:
    \begin{itemize}
    
    \item Percec, V.; Imam, M. R.; Peterca, M.; Wilson, D. A.; Graf, R.; Spiess, H. W.; Balagu-
    rusamy, V. S. K.; Heiney, P. A. Self-Assembly of Dendronized Triphenylenes into Helical
    Pyramidal Columns and Chiral Spheres. J. Am. Chem. Soc. 2009, 131, 7662–7677.

    \item Gearba, R. I.; Anokhin, D. V.; Bondar, A. I.; Bras, W.; Jahr, M.; Lehmann, M.;
    Ivanov, D. A. Homeotropic Alignment of Columnar Liquid Crystals in Open Films by
    Means of Surface Nanopatterning. Adv. Mater. 2007, 19, 815–820.
    
    \end{itemize}

    \item \begin{quote}
    \textit{These arguments would be stronger if there weren't so many
	    qualifications about simulation timescales and GAFF, in particular its ability
	    to predict the shortest-scale feature (pi-stacking) for this system. How does
	    everything else change if the force-field were tuned to get that feature
	    correct?}
    \end{quote}

    Author reply: We thank the reviewer for their constructive criticism. Many
    of our findings may be relatively independent of the force fields used, but we
    feel that we should include these qualifications in order to not overstate our
    case. It is common for researchers to run simulations for an insufficient
    length of time and to make bold claims. While we feel that we have run our 
    simulations sufficiently long, we still acknowledge the shortcomings of our
    techniques in order to inspire other researchers to build on our initial work. 

    We agree with the reviewer that we could tune our force field in order to
    exactly reproduce the $\pi-\pi$ stacking distance seen experimentally. Surely
    there will be some structural differences. There are many ways to tune a force
    field in order to achieve this. As stated earlier, we prefer to work with an
    `out-of-the-box' force field since they are widely used and can be applied  
    transferably to a large number of systems. The ultimate goal of our work is to create new monomers
    that can achieve specific separations. If we can gain experimentally-consistent
    insights using our model simulated with GAFF, then we can have increased
    confidence in future predictions generated from systems built with
    unsynthesized monomers.  With that said, we think that it would be interesting
    to repeat our analysis using different and/or optimized force fields in a
    future study in order to comment on the signficance of any differences.

    We have added to the paragraph (highlighted in red) of our conclusion which describes future
    work (bottom of page~\pageref{M-addition:forcefield-structure}) in order to acknowledge
    the possibility that a more tuned force field could influence our results:

    Modified text: ``Future work, based on what has been learned in this study, may
    help further improve the structural agreement between experiment and
    simulation {\color{red} and test the sensitivity of our current conclusions to a more
    accurate structure.}"

    \item \begin{quote}
    \textit{The MD-Structure-Factor repository is not available at the provided
	    github link, nor does it appear to be online anywhere. So, the correctness of
	    that code cannot be checked.}
    \end{quote}
    
    Author reply: We thank the reviewer for pointing this out and apologize for
    our mistake. The referenced GitHub repository was set to private due to
    miscommunication with our collaborators but it has now been made public. 
    We have performed checks to ensure that the code properly handles monoclinic
    unit cells. For example, we generated test trajectories containing thousands
    of frames made up of configurations of randomly placed point scatterers in a
    monoclinic unit cell. In this case, the simulated diffraction pattern will 
    be Gaussian. This exact result is shown in the Figure~\ref{S-fig:xrd_noise}
    of the SI. With improper treatment of the monoclinic geometry, the gaussian
    shape is skewed into an elliptical shape. 
    
    \item \begin{quote}
    \textit{One might want to check that code based on Figure 7f, where
	    presumably spherical scatterers generate reflections with anisotropy (diamonds
	    instead of squares or circles). This kind of an artefact is plausible if there
	    are errors handling the non-cubic box, or with scattering angles that aren't
	    parallel to a box axis. llcsim/analysis/correlation.py looks like it has the
	    machinery for doing the scattering analysis.}
    \end{quote}
	
    Author reply: We thank the reviewer for their astute observation and
    the resulting improvements to our manuscript. We generated Figure 7f by
    calculating the structure factor of a 2-dimensional array of hexagonally-packed
    scatterers. It is meant to be a simplified representation of the scattering
    exhibited by hexagonally packed alkane tails so that we can qualitatively
    understand where we would expect R-spots to appear if tails indeed pack in that
    way. Since it is possible to generate a rectangular unit cell which contains
    hexagonally packed point scatterers, there are no artifacts due to improper
    treatment of periodicity.  Although we believe our calculations are correct, we
    originally used only eight total scatterers in the 2D unit cell and then
    plotted the structure factor with a contour plot
    (\texttt{matplotlib.pyplot.contourf}) which does not interpolate the data. We
    re-created Figure 7f by simulating the structure factor of 100 hexagonally
    packed scatterers, in order to get higher fourier space resolution, and then
    plotting the result with \texttt{matplotlib.pyplot.imshow} and Gaussian
    interpolation. We have added the code used to generate this figure to the
    \texttt{llcsim} repository which can be accessed at
    \texttt{llcsim/analysis/hexagonal\_packing.py}.
    Additionally, we have added an entry to Table~\ref{S-table:python_scripts} in 
    order to make it clear where the code can be found.

    \item \begin{quote}	
    \textit{Regarding the ion conductivities, how do the numbers presented here
	    compare against ion conductivities predicted with GAFF and other polarizable
	    and nonpolarizable force fields? If GAFF is systematically too high in other
	    cases, it could help explain the order-of-magnitude discrepancy observed here.}
    \end{quote}

	Author reply: We thank the reviewer for bringing up a good point. There are few studies
	with experimental comparisons of sodium ion conduction. Conduction of potassium in 
	ion channels is quite common. The ionic conductivity is directly related to the 
	diffusivity of the ion of interest, so if the dynamics of sodium are overpredicted
	in our model, this could give rise to the discrepancy between experiment and simulation.
	The diffusivity of sodium ions in water has been overpredicted with multiple models, 
	including AMBER where the diffusivity is about 2x too high. We have modified the text 
	on page 49 to include additional reasoning for the 
	discrepancy between the simulated and experimental ionic conductivity:
	
%	Original text: ``The calculated value of ionic conductivity is 5 times higher than experiment 
%    likely because we simulated infinitely long, aligned pores. The ionic conductivity 
%    measurement to which we are comparing was done with an \SI{80}{\micro\metre}-thick film, 
%    nearly 10,000 times thicker than our simulated system. The thick film is likely 
%    imperfectly aligned and has defects leading to non-contiguous pores. It has been shown
%    that there is a large dependence of ionic conductivity on the alignment of the pores.
%    The ionic conductivity of an isotropically aligned film is ca. 85 times lower than that
%    of the nearly aligned film to which we are comparing.$^{16}$ We 
%    hypothesize that a thin, perfectly aligned film would have a value of ionic conductivity
%    in closer agreement with our model."
    
    Modified text: ``The calculated value of ionic conductivity is 5 times higher than
    experiment likely because we simulated infinitely long, aligned pores {\color{red} and
    because our model over-predicts the diffusivity of sodium}. The ionic conductivity 
    measurement to which we are comparing was done with an \SI{80}{\micro\metre}-thick film, 
    nearly 10,000 times thicker than our simulated system. The thick film is likely 
    imperfectly aligned and has defects leading to non-contiguous pores. It has been shown
    that there is a large dependence of ionic conductivity on the alignment of the pores.
    The ionic conductivity of an isotropically aligned film is ca. 85 times lower than that
    of the nearly aligned film to which we are comparing$^{16}$. 
    {\color{red} Additionally, sodium ions parameterized with AMBER parameters exhibit bulk
    water diffusion coefficients that are two times greater than experiment.$^{56}$ We cannot 
    definitively say how this ratio changes in the confined LLC pore environment, but
    it is likely that sodium diffuses faster in our system which contributes to a higher 
    ionic conductivity than experiment.}"
	
	% Amber parameters used for sodium
	% most studies with conductivity look at potassium
	% look at diffusivities

	\item \begin{quote}
    \textit{If the authors either (a) strengthen the cases for their proposed structural
    insights being  highly likely (not just plausible), or (b) make a stronger case for
    why these particular insights have important ramifications for our understanding of 
    LLC's (their significance), then it would be easier to publish this immediately. This
    reviewer's opinion is that this work would find a broader interested audience with a more
    concise focus on these physical insights, which could be accomplished by moving more
    content (e.g., dihedral correlations) to the SI.}
    \end{quote}
    
    Author reply: We feel that the additions and modifications to the sections cited above 
    have made the intent of our work clearer and helped communicate the significance of 
    our learned insights for future work on this topic. We have clarified the purpose of our work
    and what we expect to learn from it. We have commented on the significance of knowing the
    number of columns-per-pore. We have emphasized the observation that the radial distribution
    of monomer components is relatively independent of the system's initial configuration
    which is important for future high-throughput work. We have clarified and strengthened our
    reasoning for the existence of the R-double feature.
    
    We sympathize with the reviewer's desire for a more concise manuscript but feel that most 
    of the information presented is necessary in order to come to a full understanding of the 
    work we've performed (an opinion shared by Reviewer 2). We have attempted to satisfy both
    viewpoints, by carefully examining the manuscript for figures and discussions which draw
    the reader's attention from the important physical insights, and moving these to the SI. 
    %MRS: could you clarify this?  Maybe just state how many pages were moved. 
    Including additions made above, we have reduced the total length of the body of the 
    manuscript by 3 pages. The following sections were moved to the SI, with only the key
    information left in the main paper:
    \begin{itemize}
    
    \item We moved the plots of the 1D Correlation functions, $g(z)$, used to calculate 
    correlation lengths from Section~\ref{M-section:rpi} to the SI, now Figure~\ref{S-fig:correlation}.
    The most important information, the correlation lengths, is already presented in 
    Table~\ref{M-table:correlation_length}.
    
    \item We moved the discussion of the influence of system size on $g(z)$ 
    from Section~\ref{M-section:rpi} to the Supporting Information, 
    Section~\ref{S-section:correlation_functions}. We supplemented that discussion in
    the main text with the following text:    
    
    ``We show that the system size in the $z$-direction does not significantly 
    alter $g(z)$ in Section~\ref{S-section:correlation_functions} of the SI."
    
    \item We removed the discussion of a system that we annealed from 500 K to 300 K
    from Section~\ref{M-section:rpi} because it is described again (with better context)
    in Section~\ref{M-section:slow_dynamics}.
    
	\item We moved the discussion of a correlation function, $g(z)$, that averages
	all slices of the 3D correlation function and the associated figure from 
	Section~\ref{M-section:rpi} to Section~\ref{S-section:correlation_functions}.
    
    \item We've supplemented the discussion of the influence of increased noise in the 
    $z$-direction and in the $xy$ plane on the shape and intensity of R-$\pi$ with the
    following paragraph:
    
    ``We also used our simplified systems in order to study the influence of increased
    disorder in each dimension. Increasing $z$-directional disorder reduces the
    intensity of R-$\pi$ without changing the shape of its $q_r$ cross-section.
    Increasing disorder on the $xy$ plane, somewhat counterintuitively, reduces
    the FWHM of the $q_r$ cross-section of R-$\pi$ with an insignificant effect
    on its intensity. We explain these points in more depth in 
    Section~\ref{S-section:simplified_sf}."
    
    The full discussion, with figures, has been moved from Section~\ref{M-section:rpi}
    to Section~\ref{S-section:simplified_sf}. The subsection of Section~\ref{M-section:rpi}
    starting on page~\pageref{M-section:rpi_xsection} of the main text now has a primary
    focus on the relationship between R-$\pi$ and inter-column correlation.
    
    \item We moved the plots of dihedral autocorrelation functions to the SI, 
    Figure~\ref{S-fig:dihedral}.
    
    \item We moved the figure associated with physical property changes after cross-linking
    to the SI, Figure~\ref{S-fig:xlink}, since its main findings are summarized sufficiently
    in the text.
    
    \end{itemize}
	
\end{enumerate}

\section{Response to Reviewer 2}

\begin{enumerate}
	
	\item \begin{quote} \textit{The authors report a computer simulation study of the detailed structure
	of several systems, which are similar to the cross-linked inverted hexagonal phase of self-assembled
	lyotropic liquid crystals (LLCs) used to develop porous polymer membranes for separations. Specifically,
	the authors studied the Col-h thermotropic (solvent-free) phase formed by the assembly of the Na-GA3C11
	monomer, which is similar to the LLC phases used in the development of polymeric membranes. The Col-h 
	phase has been characterized experimentally using wide-angle X-ray scattering (WAXS), but nonetheless 
	important gaps in its structure still remains. Using all-atom classical molecular dynamics simulations,
	the authors performed a very careful and very detailed study of the structure of the Col-h phase formed
	by Na-GA3C11. The authors compared their simulated X-ray diffraction (XRD) patterns with existing 2D-WAXS
	experimental data, carefully discussed and explained the observed differences between simulations and 
	experiments, and provided molecular-level details of the structure of this system that cannot be obtained
	from experiments. This study is very detailed; all relevant information is provided for interested readers
	to replicate results from this study (including links to Python scripts used to set up the simulations, 
	analyze MD trajectories, and obtain simulated XRD patterns). The simulation results has been analyzed and
	discussed comprehensively (perhaps too comprehensively, as this paper is 63 pages long, has 21 figures and
	its Supporting Information file has 19 additional pages; however, I don’t think the authors should make any
	efforts to shorten the paper, as I feel every presented component is required to fully understand the 
	contributions of this study) Overall this is an extremely solid manuscript}\end{quote}
	
	Author reply: We thank the reviewer for the kind assessment of our work. We also acknowledge that this is a long article
	
	\item \begin{quote}
	
	\textit{I have no suggestions to further improve this paper, other than fixing a few small typos:}
	
    \textit{(1) Page 10, first paragraph: remove the ‘page 58’ associated with reference 25}

    \textit{(2) Page 54: reference to the paper of Feng et al is missing}

	\end{quote}
    
    Author reply: We thank the reviewer for pointing out these minor errors in our main text. These issues have been corrected in
    the revised main text file. % need to add page numbers once finished
    
    \item \begin{quote} \textit{(3) Page 54: mention examples of force fields that explicitly include pi-pi interactions
    } \end{quote}
    
    Author reply: We have adjusted our language in order to specifically mention and cite polarizable force fields as follows: 
    
    Modified text: ``{\color{red} Polarizable} force fields {\color{red} such as AMOEBA which}
    explicitly include $\pi-\pi$ interactions {\color{red} between aromatic functional groups}
    may be able to draw stacked monomers closer together.{\color{red}$^{57-59}$}"
     
    We have added citations with relevant force field examples, including AMOEBA:
     
    \begin{itemize}
     
    \item Shi, Y.; Xia, Z.; Zhang, J.; Best, R.; Wu, C.; Ponder, J. W.; Ren, P. Polarizable Atomic
    Multipole-Based AMOEBA Force Field for Proteins. J. Chem. Theory Comput. 2013,
    9, 4046–4063.
    
    \item Kaminski, G. A.; Stern, H. A.; Berne, B. J.; Friesner, R. A. Development of an Accu-
    rate and Robust Polarizable Molecular Mechanics Force Field from ab Initio Quantum
    Chemistry. J. Phys. Chem. A 2004, 108, 621–627.
    
    \item Lopes, P. E. M.; Lamoureux, G.; Roux, B.; MacKerell, A. D. Polarizable Empirical
    Force Field for Aromatic Compounds Based on the Classical Drude Oscillator. J. Phys.
    Chem. B 2007, 111, 2873–2885.
    
    \end{itemize}

\end{enumerate}

\end{document}
