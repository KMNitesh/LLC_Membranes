\documentclass{article}
\usepackage[margin=1in]{geometry}
\usepackage{amsmath}
\usepackage{xr}

\externaldocument[S-]{supporting_information}
\externaldocument[M-]{Final_Draft}

\title{Response to reviewers: Understanding the Nanoscale Structure of Inverted Hexagonal Phase Lyotropic Liquid Crystal Polymer Membranes}
\author{Benjamin J. Coscia \and Joseph Yelk \and Matthew A. Glaser \and Douglas L. Gin \and Xunda Feng \and Michael R. Shirts}

\begin{document}
 
\maketitle

We thank the reviewers for carefully reading over our manuscript and providing helpful comments. We have taken the suggestions into consideration and made appropriate revisions to the manuscript document. All changes to the text have been documented below. 

\section{Response to Reviewer 1}

\begin{enumerate}
	
	\item \begin{quote} \textit{This manuscript describes atomistic molecular dynamics simulations and
	diffraction analysis of hexagonally-arranged pores, each of which is made of many columns of the 
	monomer Na-GA3C11. This physical system is of interest because such nanopores have potential to 
	transform filtration applications. This work is of potential interest to the JPC readership because
	it provides atomistic insight into the origins of diffraction reflections observed in experiments, 
	which could be used to resolve discrepancies about what the structure of these hexagonal arrays are,
	or used to engineer better filters. Overall, the manuscript provides comprehensive descriptions of
	metrics used to evaluate the simulated structures against experiments, but the number of things
	described and length of these descriptions obscures whether significant new physical insight into
	this system is made.} \end{quote}
	
	Author reply: We thank the reviewer for their constructive feedback and aim to address all of their concerns. 
	
	\item \begin{quote}
	\textit{As written, it is not clear if the authors are presenting this study as a validation of 
	GAFF and the present equilibration techniques, or whether this study is providing significant new
	physical insight into these lyotropic liquid crystals. The paper reads a bit more like the former,
	with a comprehensive discussion of the degree to which five scattering features expected in 
	experiments are predicted in simulations. However, at the same time, the authors make arguments
	including (1) pores are made from five columns of monomers, (2) that water is present in the "dry"
	structures, and (3) that a scattering feature comes from tail packing, not tilt: These observations
	*could* be new physical insight into lyotropic liquid crystal structure, but as presented seem more
	like hypotheses than conclusions:}
	
	%BJC: want to convery that we want to reproduce all of the experimental features as best as possible
	% using an out-of-the-box force field. That way we know we are approximating the chemical environment
	% as close to the experimental system as possible. The goal is to eventually test large numbers of 
	% monomers without the needing to dive into the excruciating detail shown here. 
	
	\end{quote}
	
	\item \begin{quote}
    \textit{1. For the simulations performed here, the authors do make a strong case that 5 columns per
    core gets the best pore-to-pore spacing (Figure 5). However, insufficient time is accessed to see 
    whether the 4-, 5-, and 6-column initial conditions move towards or away from the expected pore 
    spacing over many-microseconds trajectories. That said, for the quenches performed here, 5-columns
    per core seems more plausible than 4.}
	\end{quote}
	
	Author reply: We agree with the reviewer that 5-columns per core is more plausible than 4 and that much
    longer simulations may provide further supporting evidence. However, there is very little motion on
    the hundreds of nanoseconds timescale which we simulated which suggests that we will not gain any 
    significant new insight by extending our simulations into the microsecond regime. It is not feasible 
    for us to carry out these simulations as it would consume our computational resources with little reward.
    %BJC: The following might apply:
    Some confusion may have been generated by agreement of the disordered basin 6 column-per-pore system
    with experimental pore spacing. As mentioned in the text, the vertical stacking distance between monomers
    in that system is far too large. The columns are `stretch' in the $z$-direction allowing the pores take
    up less space on the $xy$ plane and thus pack closer together.  
    
	\item \begin{quote}    
    \textit{Is this significant physical insight? It is insight, but the
    authors do not make a compelling case for how 5 columns would change how we think about these crystals,
    or how it would impact membrane engineering (or more importantly, whether these 5-column structures
    assemble in the solution phase, or are only locally stable).}
    \end{quote}
    
    Author reply: We thank the reviewer for pushing us to provide more significant physical insight based
    on the number of monomers per column. The number of columns-per-pore influences the pore radius which
    is important for determining the limits of size-exclusion-based separations. It also tells use the 
    density of interacting sites within the pore which may play a role in small molecule transport. We have
    added the following text to the main document at the end of Section~\ref{M-section:mon_per_pore} on 
    page _:
    
    ``The number of columns per pore dictates the density of interaction sites within each pore and the 
    pore radius. A higher number of interaction sites will surely play a role in transport of molecules
    that have an affinity for the monomer head groups. 
    %We measured the pore radius as the average distance from the pore center to each head group COM. 
    The pore radius increases with the number of columns per pore (See Figure~\ref{S-fig:pore_radius}. 
    This radius is likely related to the size of the molecules which can be excluded by this type of 
    membrane. We may be able to control LLC membrane pore size by selecting monomers that are stable
    in configurations with more columns-per-pore."
    
    \item \begin{quote}
    \textit{2. The authors make a pretty compelling case that water causes the R-double scattering feature
    by showing its absence when water is added, but that the R-double appears and persists for 200ns after
    the water makes its way into the crystal. However, the authors also show that nonuniformly spaced
    monomers can give rise to the R-Double feature, a seeming counterexample to the necessity of water.
    Is the R-double feature with water because the monomers are now irregularly spaced? What happens when
    the waters are removed and the scattering re-performed in Fig 17c(bottom)?}
    \end{quote}
    
    Author reply: The reviewer makes good points about R-double that need to be clarified in the main text.
    R-double appears in a dry system with non-uniformly spaced monomers. However, such a system is only 
    stable, as we've modeled it, if $z$-direction position restraints are applied to monomer head groups. As
    soon as the restraints are released, R-double fades and the monomer spacing becomes uniform. We believe 
    that water facilitates the appearance of R-double by stabilizing non-uniform spacing through hydrogen bonding,
    as the reviewer suggests. We re-performed the scattering analysis on the same system with water molecules
    removed, as suggested, and R-double is still present.
    
    % BJC: mention that this plot is added to supporting info and that the "head group removed" figure further
    % supports this conclusion.
    
	\item \begin{quote}
    \textit{3. The authors make the case that the R-spots feature comes from the structure of the hexagonal
    tails, not their tilt. The authors should perform the scattering analysis to check if the R-spots are
    present on the tilted-tails they constrain: This quick experiment would nail down their interpretation
    here, or show that the tilted tails are also plausible (but not observed in this model).}
    \end{quote}
    
    \item \begin{quote}
    \textit{These arguments would be stronger if there weren't so many qualifications about simulation 
    timescales and GAFF, in particular its ability to predict the shortest-scale feature (pi-stacking) for 
    this system. How does everything else change if the force-field were tuned to get that feature correct?}
    \end{quote}

	\item \begin{quote}
    \textit{The MD-Structure-Factor repository is not available at the provided github link, nor does it appear
    to be online anywhere. So, the correctness of that code cannot be checked.}
	\end{quote}
    
	Author reply: We thank the reviewer for pointing this out and apologize for this mistake. The referenced
	GitHub repository was set to private by our collaborators but has now been made public. Please feel 
	free to read through the code. 
    
    \item \begin{quote}
    \textit{One might want to check that code based on Figure 7f, where presumably spherical scatterers 
    generate reflections with anisotropy (diamonds instead of squares or circles). This kind of an artefact
    is plausible if there are errors handling the non-cubic box, or with scattering angles that aren't parallel
    to a box axis. llcsim/analysis/correlation.py looks like it has the machinery for doing the scattering analysis.}
	\end{quote}
	
	Author reply: We thank the reviewer for their astute observation and the resulting improvements to 
	our manuscript.	We generated Figure 7f by calculating the structure factor of a 2-dimensional array of 
	hexagonally-packed scatterers. It is meant to be a simplified representation of the scattering exhibited
	by hexagonally packed alkane tails so that we can qualitatively understand where we would expect R-spots to appear if 
	tails indeed pack in that way. Since it is possible to generate a rectangular unit cell which contains 
	hexagonally packed point scatterers, there are no artifacts due to improper treatment of periodicity. 
	Although we believe our calculations are correct, we originally used only eight total scatterers in the 
	2D unit cell and then plotted the structure factor with a contour plot (\texttt{matplotlib.pyplot.contourf})
	which does not interpolate the data. We re-created Figure 7f by simulating the structure factor of 100 
	hexagonally packed scatterers, in order to get higher fourier space resolution, and then plotting the
	result with \texttt{matplotlib.pyplot.imshow} and Gaussian interpolation. We have added the code used to
	generate this figure to the \texttt{llcsim} repository which can be accessed at 
	\texttt{https://github.com/bencoscia/llcsim/blob/master/analysis/hexagonal\_packing.py}.
	
	% Add code and clarify main text

	\item \begin{quote}	
    \textit{Regarding the ion conductivities, how do the numbers presented here compare against ion conductivities
    predicted with GAFF and other polarizable and nonpolarizable force fields? If GAFF is systematically too high
    in other cases, it could help explain the order-of-magnitude discrepancy observed here.}
    \end{quote}

	\item \begin{quote}
    \textit{If the authors either (a) strengthen the cases for their proposed structural insights being  highly 
    likely (not just plausible), or (b) make a stronger case for why these particular insights have important 
    ramifications for our understanding of LLC's (their significance), then it would be easier to publish this 
    immediately. This reviewer's opinion is that this work would find a broader interested audience with a more
    concise focus on these physical insights, which could be accomplished by moving more content 
    (e.g., dihedral correlations) to the SI.}
    \end{quote}
	
\end{enumerate}

\section{Response to Reviewer 2}

\begin{enumerate}
	
	\item \begin{quote} \textit{The authors report a computer simulation study of the detailed structure
	of several systems, which are similar to the cross-linked inverted hexagonal phase of self-assembled
	lyotropic liquid crystals (LLCs) used to develop porous polymer membranes for separations. Specifically,
	the authors studied the Col-h thermotropic (solvent-free) phase formed by the assembly of the Na-GA3C11
	monomer, which is similar to the LLC phases used in the development of polymeric membranes. The Col-h 
	phase has been characterized experimentally using wide-angle X-ray scattering (WAXS), but nonetheless 
	important gaps in its structure still remains. Using all-atom classical molecular dynamics simulations,
	the authors performed a very careful and very detailed study of the structure of the Col-h phase formed
	by Na-GA3C11. The authors compared their simulated X-ray diffraction (XRD) patterns with existing 2D-WAXS
	experimental data, carefully discussed and explained the observed differences between simulations and 
	experiments, and provided molecular-level details of the structure of this system that cannot be obtained
	from experiments. This study is very detailed; all relevant information is provided for interested readers
	to replicate results from this study (including links to Python scripts used to set up the simulations, 
	analyze MD trajectories, and obtain simulated XRD patterns). The simulation results has been analyzed and
	discussed comprehensively (perhaps too comprehensively, as this paper is 63 pages long, has 21 figures and
	its Supporting Information file has 19 additional pages; however, I don’t think the authors should make any
	efforts to shorten the paper, as I feel every presented component is required to fully understand the 
	contributions of this study) Overall this is an extremely solid manuscript}\end{quote}
	
	Author reply: We thank the reviewer for the kind assessment of our work. We also acknowledge that this is a long article
	
	\item \begin{quote}
	
	\textit{I have no suggestions to further improve this paper, other than fixing a few small typos:}
	
    \textit{(1) Page 10, first paragraph: remove the ‘page 58’ associated with reference 25}

    \textit{(2) Page 54: reference to the paper of Feng et al is missing}

	\end{quote}
    
    Author reply: We thank the reviewer for pointing out these minor errors in our main text. These issues have been corrected in
    the revised main text file. % need to add page numbers once finished
    
    \item \begin{quote} \textit{(3) Page 54: mention examples of force fields that explicitly include pi-pi interactions
    } \end{quote}
    
    Author reply: We have adjusted our language in order to specifically mention and cite polarizable force fields as follows: 
    
    ``Polarizable force fields explicitly including $\pi-\pi$ interactions may be able to draw stacked
    monomers closer together"

\end{enumerate}

\end{document}