\documentclass{article}
\usepackage{graphicx}
\usepackage{wrapfig}
\usepackage{subcaption}
\usepackage[margin=1in]{geometry}
\usepackage{amsmath} % or simply amstext
\usepackage{siunitx}
\usepackage{booktabs}
\usepackage[export]{adjustbox}
\newcommand{\angstrom}{\textup{\AA}}
\newcommand{\colormap}{jet}  % colorbar to use
\usepackage{cleveref}
\usepackage{booktabs}
\usepackage{gensymb}
\usepackage{float}
\title{Intuitive explanations of X-ray diffraction patterns}
\author{Benjamin J. Coscia \and Michael R. Shirts} 
\begin{document}
  \bibliographystyle{ieeetr}
  \graphicspath{{./figures/}}
  \maketitle

  \section{Topics to explore}

  Topics here are trivial to people who have long been immersed in crystallography or a related field. This stuff is far from intuitive 

  Reference Atlas of optical transforms
  \begin{itemize}
  	\item 2D transforms of 2D arrays explored in detail
	\item Here we look at slices of 3D transforms created from 3D coordinates
  \end{itemize}

  \begin{itemize}
	\item Narrow to structures with periodic order (eliminate proteins, amporphous materials)
	\item Will not talk about instrumentation
	\item Common crystal lattices
	\begin{itemize}
		\item 1D and 2D patterns (potentially 3D)
		\item Perfect lattices
		\item Sample orientation 
		\item Add noise in each dimension
	\end{itemize}
	\item Common liquid crystal phases
	\begin{itemize}
		\item show 1D and 2D patterns
		\item Nematic
		\item Columnar
		\item Bicontinuous Cubic
		\item Smectic
	\end{itemize}
	\item Other shapes
	\begin{itemize}
		\item helices + screw axes
	\end{itemize}			
	\item Effect of tortuosity
	\item Effect of misaligned domains - angle averagin
	\item Intensity of subharmonics

\end{document}
