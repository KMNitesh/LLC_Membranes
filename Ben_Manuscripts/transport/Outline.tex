\documentclass{article}
\usepackage{graphicx}
\usepackage{wrapfig}
\usepackage{subcaption}
\usepackage[margin=1in]{geometry}
\usepackage{amsmath} % or simply amstext
\usepackage{siunitx}
\usepackage{booktabs}
\usepackage[export]{adjustbox}
\newcommand{\angstrom}{\textup{\AA}}
\newcommand{\colormap}{jet}  % colorbar to use
\usepackage{cleveref}
\usepackage{booktabs}
\usepackage{gensymb}
\usepackage{float}

\title{The Transport Mechanisms of Polar Solutes in a Cross-linked H\textsubscript{II} Phase Lyotropic Liquid Crystal Membrane}
\author{Benjamin J. Coscia \and Douglas L. Gin \and Richard D. Noble \and Joe Yelk \and Matthew Glaser \and Xunda Feng \and Michael R. Shirts} 

\begin{document}
%  \bibliographystyle{ieeetr}
  \graphicspath{{./figures/}}
  \maketitle
  \section{Introduction}
  
  We need highly selective membranes in order to perform efficient separations.

  H\textsubscript{II} phase lyotropic liquid crystals have densely packed, uniform
  sized pores and have the potential to disrupt conventional membrane separation
  techniques by being selective based not only on size and charge, but on chemical
  functionality as well.


  We can only learn so much from experiment. MD can give us mechanistic insights with
  atomistic resolution so that we can intelligently design new membranes for 
  solute-specific separations.

  In previous work, we determined the most likely structure of the hexagonal phase 
  formed by the monomer Na-GA3C11.
  \begin{itemize}
  	\item We developed techniques for equilibrating the hexagonal phase made by
	neat monomer as well as with varying amounts of water in the pores.
  \end{itemize} 

  In this work, we have determined the transport mechanisms and macroscopic transport 
  properties exhibited by a number of polar solutes with varying size, chemical 
  functionality and hydrophilic character.
  \begin{itemize}
	\item Many of the separations we are interested in involve polar organic 
	compounds.
  \end{itemize} 

  We have compared our calculated diffusion coefficients with experimental measurements
  made using DOSY NMR. 

  \section{Methods}

  \subsection{Molecular Dynamics Simulations}
  
  \subsubsection*{System Setup}

  There is a broad range of water concentrations which will form a stable 
  H\textsubscript{II} phase with Na-GA3C11. 
  \begin{itemize}
	\item In the literature this system is typically synthesized with close
	to 10 wt \% water
        \item However, Resel et al. noted that the system is likely fully 
	hydrated with less than 7 wt \% water.
	\item We decided to test two different levels of water content: 5 and 10 wt \%
  \end{itemize} 

  We observed that water partitions into the tail region of our system and therefore
  built our initial configurations with water in both regions close to the expected
  equilibrium value.
  \begin{itemize}
	\item There is about 2:1 water in the pores versus in the tails for the 10 wt \% system.
	\item We adjusted the pore radius in our systems so that the right amount of water
	fits in the pores without any vacuum using \texttt{gmx solvate}.
	\item We placed water molecules in the tail region one at a time in random locations
	with short energy minimizations between insertions.
  \end{itemize}

%  We equilibrated the initial configuration using the `wet' equilibration procedure
%  described in our previous work (reference to structure paper).
  %BJC: not sure I need to go into any details describing that procedure
%  \begin{itemize}
%	\item Series of NVT simulations with force constants on carbon atoms of aromatic
%	ring in head group
%	\item Force constants reduced according to the sequence: 1000000, 3162,
%	56, 8, 3, 2, 1, 0 kJ mol$^{-1}$ nm$^{-2}$ 
%  \end{itemize}

%  We cross-linked the equilibrated solvated configuration using the cross-linking procedure
%  described in our previous work. 

  We equilibrated our initial configuration before adding solutes.
  \begin{itemize}
	\item We equilibrated the initial configuration using the `wet'
	equilibration procedure described in our previous work (reference to structure
	paper).
	\item We cross-linked the equilibrated solvated configuration using the
	cross-linking procedure described in our previous work. 
  \end{itemize}

  We added 6 solute molecules to each pore of the equilibrated cross-linked
  configuration.
  \begin{itemize}
	\item We equally spaced each solute in the pore
	\item 6 solutes per pore provided a balance of a useful amount of data
	for generating statistics and a low degree of interaction between solutes
	\item At each insertion point we placed a randomly oriented solute molecule
	then ran a short energy minimzation.
	\item We allowed the solutes to equilibrate for 5 ns using berendsen 
	pressure control
	\item We collected transport data using long simulations, on the order of
	1 microsecond, with pressure controlled by the Parrinello-Rahman barostat.
  \end{itemize}
  
  \subsubsection*{Modeling the Continuous Time Random Walk}

  A continuous time random walk (CTRW) is a stochastic process characterized by a 
  distribution of hop lengths and dwell times.
  \begin{itemize}
	\item We use the \texttt{ruptures} python package in order to identify
	breakpoints in solute trajectories. (See Supporting Information for more
	details on chosen parameters. i.e. type of cost function, cost function penalty
	tolerance, number of dimensions used)
	\item We have limited data, so we attempted to approximate these distributions
	with known continuous probability densities.
	\item For solutes in our system the distribution of hop lengths is
	well-represented by a Gaussian distribution while the distribution of dwell
	times fits relatively well to an exponential distribution characteristic of a
	Poisson process. (See supporting information)
  \end{itemize}

  \begin{equation}
	Ae^{-x}
  \end{equation}

  We calculated macroscopic diffusion coefficients by simulating trajectories orders of
  magnitude longer than our molecular simulations. 
  \begin{itemize}
	\item We constructed trajectories by generating sequences of dwell times 
        and hop lengths randomly sampled from our fit distributions.
	\item For time scales much greater than the characteristic time of our
	exponential dwell time distribution, solutes exhibit Brownian motion. 
	\item The MSD curve becomes linear
	\item We fit a line to the MSD curve and used the Einstein relationship
	to relate its slope to the macroscopic diffusion coefficient.
	\item We found convergence of the value of the diffusion coefficient 
	calculated in this way afer x steps (See Supporting Information)
  \end{itemize}

  \subsection{Experimental}
  % BJC: For Greg to write if things work out
   
  \section{Results and Discussion}
  
 
  \section{Conclusion}
  
 
  \section*{Supporting Information}

  Detailed explanations and expansions upon the results and procedures mentioned in
  the main text are described in the Supporting Information. This information is
  available free of charge via the Internet at http://pubs.acs.org.

  \section*{Acknowledgements}

  Molecular simulations were performed using the Extreme Science and
  Engineering Discovery Environment (XSEDE), which is supported by National
  Science Foundation grant number ACI-1548562. Specifically, it used the Bridges
  system, which is supported by NSF award number ACI-1445606, at the Pittsburgh
  Supercomputing Center (PSC). This work also utilized the RMACC Summit supercomputer,
  which is supported by the National Science Foundation (awards ACI-1532235 and
  ACI-1532236), the University of Colorado Boulder, and Colorado State
  University. The Summit supercomputer is a joint effort of the University of
  Colorado Boulder and Colorado State University.

  \clearpage
  \bibliography{llc}

  \newpage

  \section*{TOC Graphic}

\end{document}
