\documentclass{article}
\usepackage{graphicx}
\usepackage{wrapfig}
\usepackage{subcaption}
\usepackage[margin=1in]{geometry}
\usepackage{amsmath} % or simply amstext
\usepackage{siunitx}
\usepackage{booktabs}
\usepackage[export]{adjustbox}
\newcommand{\angstrom}{\textup{\AA}}
\newcommand{\colormap}{jet}  % colorbar to use
\usepackage{cleveref}
\usepackage{booktabs}
\usepackage{gensymb}
\usepackage{float}

\title{The Transport Mechanisms of Polar Solutes in a Cross-linked H\textsubscript{II} Phase Lyotropic Liquid Crystal Membrane}
\author{Benjamin J. Coscia \and Douglas L. Gin \and Richard D. Noble \and Joe Yelk \and Matthew Glaser \and Xunda Feng \and Michael R. Shirts} 

\begin{document}
%  \bibliographystyle{ieeetr}
  \graphicspath{{./figures/}}
  \maketitle
  \section{Introduction}
  
  We need highly selective membranes in order to perform efficient separations.

  H\textsubscript{II} phase lyotropic liquid crystals have densely packed, uniform
  sized pores and have the potential to disrupt conventional membrane separation
  techniques by being selective based not only on size and charge, but on chemical
  functionality as well.


  We can only learn so much from experiment. MD can give us mechanistic insights with
  atomistic resolution so that we can intelligently design new membranes for 
  solute-specific separations.

  In previous work, we determined the most likely structure of the hexagonal phase 
  formed by the monomer Na-GA3C11.
  \begin{itemize}
  	\item We developed techniques for equilibrating the hexagonal phase made by
	neat monomer as well as with varying amounts of water in the pores.
  \end{itemize} 

  In this work, we have determined the transport mechanisms and macroscopic transport 
  properties exhibited by a number of polar solutes with varying size, chemical 
  functionality and hydrophilic character.
  \begin{itemize}
	\item Many of the separations we are interested in involve polar organic 
	compounds.
  \end{itemize} 

  We have compared our calculated diffusion coefficients with experimental measurements
  made using DOSY NMR. 

  \section{Methods}

  \subsection{Molecular Dynamics Simulations}
  
  \subsubsection*{System Setup}

  There is a broad range of water concentrations which will form a stable 
  H\textsubscript{II} phase with Na-GA3C11. 
  \begin{itemize}
	\item In the literature this system is typically synthesized with close
	to 10 wt \% water
        \item However, Resel et al. noted that the system is likely fully 
	hydrated with less than 7 wt \% water.
	\item We decided to test two different levels of water content: 5 and 10 wt \%
  \end{itemize} 

  We observed that water partitions into the tail region of our system and therefore
  built our initial configurations with water in both regions close to the expected
  equilibrium value.
  \begin{itemize}
	\item There is about 2:1 water in the pores versus in the tails for the 10 wt \% system.
	\item We adjusted the pore radius in our systems so that the right amount of water
	fits in the pores without any vacuum using \texttt{gmx solvate}.
	\item We placed water molecules in the tail region one at a time in random locations
	with short energy minimizations between insertions.
  \end{itemize}

%  We equilibrated the initial configuration using the `wet' equilibration procedure
%  described in our previous work (reference to structure paper).
  %BJC: not sure I need to go into any details describing that procedure
%  \begin{itemize}
%	\item Series of NVT simulations with force constants on carbon atoms of aromatic
%	ring in head group
%	\item Force constants reduced according to the sequence: 1000000, 3162,
%	56, 8, 3, 2, 1, 0 kJ mol$^{-1}$ nm$^{-2}$ 
%  \end{itemize}

%  We cross-linked the equilibrated solvated configuration using the cross-linking procedure
%  described in our previous work. 

  We equilibrated an initial solvated configuration before adding solutes.
  \begin{itemize}
	\item We equilibrated the initial configuration using the `wet'
	equilibration procedure described in our previous work (reference to structure
	paper).
	\item We cross-linked the equilibrated solvated configuration using the
	cross-linking procedure described in our previous work. 
  \end{itemize}

  We added 6 solute molecules to each pore of the equilibrated cross-linked
  configuration.
  \begin{itemize}
	\item We equally spaced each solute in the pore
	\item 6 solutes per pore provided a balance of a useful amount of data
	for generating statistics and a low degree of interaction between solutes
	\item At each insertion point we placed a randomly oriented solute molecule
	then ran a short energy minimzation.
	\item We allowed the solutes to equilibrate for 5 ns using berendsen 
	pressure control
	\item We collected transport data using long simulations, on the order of
	1 microsecond, with pressure controlled by the Parrinello-Rahman barostat.
  \end{itemize}
  
  \subsubsection*{Modeling the Continuous Time Random Walk}\label{method:CTRW}

  A continuous time random walk (CTRW) is a stochastic process characterized by a 
  distribution of hop lengths and dwell times.
  \begin{itemize}
	\item We use the \texttt{ruptures} python package in order to identify
	breakpoints in solute trajectories. (See Supporting Information for more
	details on chosen parameters. i.e. type of cost function, cost function penalty
	tolerance, number of dimensions used)
	\item We have limited data, so we attempted to approximate these distributions
	with known continuous probability densities.
	\item For solutes in our system the distribution of hop lengths is
	well-represented by a Gaussian distribution while the distribution of dwell
	times fits relatively well to an exponential distribution characteristic of a
	Poisson process. (See supporting information)
  \end{itemize}

  \begin{equation}
	Ae^{-x}
  \end{equation}

  We calculated macroscopic diffusion coefficients by simulating trajectories orders of
  magnitude longer than our molecular simulations. 
  \begin{itemize}
	\item We constructed trajectories by generating sequences of dwell times 
        and hop lengths randomly sampled from our fit distributions.
	\item For time scales much greater than the characteristic time of our
	exponential dwell time distribution, solutes exhibit Brownian motion. 
	\item The MSD curve becomes linear
	\item We fit a line to the MSD curve and used the Einstein relationship
	to relate its slope to the macroscopic diffusion coefficient.
	\item We found convergence of the value of the diffusion coefficient 
	calculated in this way afer x steps (See Supporting Information)
  \end{itemize}

  \subsubsection*{Radial Distribution Functions}

  We measured the average radial distance of each solute of interest from the pore
  centers.
  \begin{itemize}
	\item We binned the radial distances and then normalized by the volume
	of the annulus defined by the bin edges.
	\item Although the pores are often described as straight, they have a
	small degree of tortuosity which disrupts the RDF calcuation 
	\item We obtain the best RDF by constructing splines that run through the
	pore centers.
	\item We construct the splines by dividing the membrane into 20 slices
	in the $z$-direction. Within each slice, we calculate the location of 
	the pore centers based on the average location of the aromatic rings
	that make up the monomer head groups.
	\item When calculating the RDF, the radial distance from the pore center
	is based on the distance between the solute center-of-mass and the ($x$, $y$)
	coordinates of appropriate point on the spline.
  \end{itemize}

  \subsection{Experimental}
  % BJC: For Greg to write if things work out
   
  \section{Results and Discussion}

  \subsection{Macroscopoic Diffusion Coefficients}

  \subsection{Simulated Predictions}\label{section:D_macro}
 
  Polar molecules generally exhibit a hop diffusion mechanism. 
  \begin{itemize}
	\item Each trajectory is characterized by an exponential distribution of
	dwell times between hops of random lengths from a Gaussian distribution.
  \end{itemize}

  We fit a CTRW model (see Section~\ref{method:CTRW}) to these 
  trajectories and predicted macroscopic diffusion coefficients which are
  presented in Table TBD.

  % BJC: figure with selected trajectories for each molecule.

  % BJC: table with finalized diffusion coefficients

  \subsubsection*{The Influence of Water Content on Macroscopic Diffusion Coefficients}

  \subsection{Experimental Measurements}

  % For Greg

  \subsection{Transport Mechanisms}

  In order to truly understand the molecular origins of the macroscopic
  diffusion coefficients in Section~\ref{section:D_macro}, we studied the
  microscopic interactions between solutes and the membrane that lead to the
  observed dwell time and hop distributions.
  \begin{itemize}
	\item We studied the radial distribution functions of solutes as a
	function of distance from the pore center
	\item We looked at hydrogen bonding patterns
	\item Order parameters
  \end{itemize}

  The radial distribution function of each solute studied is shown in 
  Figure TBD.

  \subsubsection*{Transport of Water}

  All water molecules exhibit hop diffusion.

  Even in the center of the pore, where the density of water molecules is
  highest, individual water molecules exhibit hop diffusion as they create a
  tight hydrogen bond network.
  %TODO: the following is a hypothesis
  \begin{itemize}
	\item Water sticks to pore walls
	\item Dwell times are short
	\item Water tumbles across pore for a while until it sticks again. 
	\item Water gets caught in h-bonds with other water molecules away
	from pore center.
  \end{itemize}

  In this confined environment, the diffusion constant is x times lower than
  expected bulk diffusion coefficient of tip3p water.

  %TODO: study diffusion mechanism of water
  
  \subsubsection*{Transport of Alcohols}

  The hydroxyl functional group of alcohol molecules is a hydrogen bond donor
  and prefers to donate it's hydrogen to more strongly polarized carboxylate head
  groups.

  As alcohol groups increase in hydrophobic character, they are more inclined
  to stick to the outside edges of the pore. 
  \begin{itemize}
	\item The alkane tails prefer to stay at the edges of the pore.
	\item Radial distribution functions show peaks at pore edges, however
	smaller alcohols have high densities near the pore center. 
	\item They tend to get trapped between monomers and closer to 
	the tail region. 
	\item The entrapment is further stabilized by hydrogen bonds with
	ether oxygens connecting the monomer's alkane tails to the head groups.
	%TODO: quantify h-bonds with ether oxygens and carbonyl oxygens
  \end{itemize}

  The diffusion coefficient of simple alcohols increases as the length of 
  the alkane chain increases.
  \begin{itemize}
	\item The dwell times increase as the oily tails become more
	entrapped in monomer tails.
  \end{itemize}

  Ethylene glycol, a diol, has two hydrogen bond donor groups.
  \begin{itemize}
	\item Can hydrogen bond with same moeity.
	\item Can hydrogen bond with different moeities in the same 
	vicinity. 
	\item Dwell times tend to be shorter. If one hydroxyl group is bound
	with a hydrogen bond, the other unbound hydroxyl group may form a hydrogen bond
	elsewhere and effectively pull the other bound hydroxyl group along with it. 
  \end{itemize}

  \subsubsection*{Transport of Acetone}

  The carbonyl group of Acetone is a hydrogen bond acceptor and therefore only
  form hydrogen bonds with water molecules in the pore. 

  The hydrophobic character of the two methyl groups on acetone causes the methyl
  groups to gravitate towards the outside of the pore, while the carbonyl group
  reaches towards the pore center in order to hydrogen bond with water molecules.
  \begin{itemize}
	\item Order parameter defined between vector along carbonyl and vector extending
	from acetone COM to pore center is non-zero.
  \end{itemize} 

  \subsubsection*{Transport of Acetic Acid}

  Acetic acid, since we modeled it solely in its protonated state, has hydrogen
  bond donor and acceptor groups. 

  \subsubsection*{Transport of Ions} % probably just sodium

   
  \section{Conclusion}

  We have examined the transport characteristics of a series of small polar
  molecules in our model of the H\textsubscript{II} phase formed by 
  Na-GA3C11.

  We calculated the macroscopic diffusion coefficients of each solute as 
  approximated by a CTRW model and validated our estimates using experimental
  DOSY NMR measurements.

  We have studied the influence of water content on the diffusion coefficients.

  We showed that hydrogen bonding between solutes and Na-GA3C11 monomers plays
  a major role in mechanism by which molecules traverse the nanopores. 

  We can use this intuition in order to modify our monomers for a specific 
  separation.
  \begin{itemize}
	\item Increase number of h-bond sites to increase selectivity towards water 
	over polar molecules
	\item Also separate acetone (things with only h-bond accepting groups) in this way
  \end{itemize}
  
 
  \section*{Supporting Information}

  Detailed explanations and expansions upon the results and procedures mentioned in
  the main text are described in the Supporting Information. This information is
  available free of charge via the Internet at http://pubs.acs.org.

  \section*{Acknowledgements}

  Molecular simulations were performed using the Extreme Science and
  Engineering Discovery Environment (XSEDE), which is supported by National
  Science Foundation grant number ACI-1548562. Specifically, it used the Bridges
  system, which is supported by NSF award number ACI-1445606, at the Pittsburgh
  Supercomputing Center (PSC). This work also utilized the RMACC Summit supercomputer,
  which is supported by the National Science Foundation (awards ACI-1532235 and
  ACI-1532236), the University of Colorado Boulder, and Colorado State
  University. The Summit supercomputer is a joint effort of the University of
  Colorado Boulder and Colorado State University.

  \clearpage
  \bibliography{llc}

  \newpage

  \section*{TOC Graphic}

\end{document}
